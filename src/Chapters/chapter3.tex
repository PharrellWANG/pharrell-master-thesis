\chapter{Prepare the Data for Deep Learning}\label{ch:chapter3} % For referencing the chapter elsewhere, use \ref{Chapter1}

Deep learning heavily relies on the size of available data.
It only works when a considerably large set of data can be provided.
Moreover, since we are using supervised learning which is the most
popular form of deep learning for the time being, our datasets
must be well labeled.
A large dataset can contain enormous and different classes,
with each class has its own label.
The labels are used to adjust the inner parameters of a pre-constructed
deep model according to the pre-defined loss function during the training
process.
We need to prepare a large set of labeled data before starting the
model training.
In this chapter, we start with the data collection, in which the data source
and method for collecting data are shown in detail.
After that the necessary pre-processing for the collected data are described.
Furthermore, plenty of visualizations for the collected raw data are shown with
discussion explaining the reason for the data pre-processing.

\section{Data Collection}\label{sec:data-collection}
To collect the data for training a deep model to predict the most probable
intra angular directions for depth blocks, we need to identify two questions:
Firstly, where does the data come from?
Secondly, how do you collect the data from the source?
In this subsection, the two questions are answered one by one.

\subsection{Source of Data}\label{subsec:source-of-data}
The data are collected from four video sequences as shown in
Table~\ref{tab:data-source}.
\begin{table}[t]
    \caption{Source of data for deep learning}
    \bigskip
    \label{tab:data-source}
    \centering
    \begin{tabular}{c c c c c}
        \hline
        No. & Name of the Sequence & Resolution & Usage & Number of Frames\\
        \hline
        1 & Balloons & 1024x768 & train,test,validate & 300\\
        2 & Kendo & 1024x768 & train,test,validate & 300\\
        3 & Poznan Street & 1920x1088 & train,test,validate & 250\\
        4 & Undo Dancer & 1920x1088 & train,test,validate & 250\\
        \hline
    \end{tabular}
\end{table}
Balloons sequence and Kendo sequence are of the resolution 1024 by 768 while
Poznan Street sequence and Undo Dancer sequence are of the resolution 1920
by 1088.
Both of the former two sequences have 300 frames, all of which are used to
collect the data.
The latter two sequences both have 250 frames, and all the frames are involved
in the data collection.
The collected data for each sequence will be separated into three sets
for training, testing and validating.
The training data sets are used for the deep model to learn the best
representations by the back-propagation algorithm~\parencite{RN204},
during which the inner parameters typically weights and bias are adjusted
along the gradient as instructed by the back-propagation.
After the learned model will be obtained, the validating datasets are used to
fine turn the hyper-parameters.
With the reasonably adjusted hyper-parameters, the training process will be
performed again.
After certain loops of the train-validate circle, the testing datasets will
be used to evaluate the final learned model which by then
will not be further turned any more.
After learned model will be applied to the testing datasets, the performance
results of evaluation can indicate the generalization of the learning model.

\subsection{Method for Collecting Data}\label{subsec:collecting-method}
We collect data by encoding the four video sequences shown in
Table~\ref{tab:data-source}.

\begin{algorithm}[H]
\SetAlgoLined
\KwResult{Five csv files for each sequence will be obtained}
% initialization\;
% \tcc{iDISFlag and iParNum and iDir[4] are initialized}
 \For{Every CU}{
  \eIf{DISFlag}{
   pass\;
   }{
   \eIf{iPartNum == 4}{
    collect one dimensional depth luma value for each sub parts along with the label\;
   }{
    collect one dimensional depth luma value for a single block\;
    }
  }
 }
% \label{alg:first}
 \caption{How to collect data}
\end{algorithm}

%See Algorithm~\ref{alg:first}.

\section{Data Pre-processing}\label{sec:data-preprocessing}

\section{Data Visualization}\label{sec:data-visu}
%Welcome to this \LaTeX{} Thesis Template, a beautiful and easy to use template for writing a thesis using the \LaTeX{} typesetting system.
%
%If you are writing a thesis (or will be in the future) and its subject is technical or mathematical (though it doesn't have to be), then creating it in \LaTeX{} is highly recommended as a way to make sure you can just get down to the essential writing without having to worry over formatting or wasting time arguing with your word processor.
%
%\LaTeX{} is easily able to~\parencite{RN93} professionally typeset documents that run to hundreds or thousands of pages long. With simple mark-up commands, it automatically sets out the table of contents, margins, page headers and footers and keeps the formatting consistent and beautiful. One of its main strengths is the way it can easily typeset mathematics, even \emph{heavy} mathematics. Even if those equations are the most horribly twisted and most difficult mathematical problems that can only be solved on a super-computer, you can at least count on \LaTeX{} to make them look stunning.
%
%%----------------------------------------------------------------------------------------
%
%\section{Welcome and Thanku}\label{sec:welome}
%Welcome to this \LaTeX{} Thesis Template, a beautiful and easy to use template for writing a thesis using the \LaTeX{} typesetting system.
%
%If you are writing a thesis (or will be in the future) and its subject is technical or mathematical (though it doesn't have to be), then creating it in \LaTeX{} is highly recommended as a way to make sure you can just get down to the essential writing without having to worry over formatting or wasting time arguing with your word processor.
%
%\LaTeX{} is easily able to professionally typeset documents that run to hundreds or thousands of pages long. With simple mark-up commands, it automatically sets out the table of contents, margins, page headers and footers and keeps the formatting consistent and beautiful. One of its main strengths is the way it can easily typeset mathematics, even \emph{heavy} mathematics. Even if those equations are the most horribly twisted and most difficult mathematical problems that can only be solved on a super-computer, you can at least count on \LaTeX{} to make them look stunning.
%
%%----------------------------------------------------------------------------------------
%
%\section{Welcome and ThYou}\label{sec:weome}
%Welcome to this \LaTeX{} Thesis Template~\parencite{Reference1}, a beautiful and easy to use template for writing a thesis using the \LaTeX{} typesetting system.
%
%If you are writing a thesis (or will be in the future) and its subject is technical or mathematical (though it doesn't have to be), then creating it in \LaTeX{} is highly recommended as a way to make sure you can just get down to the essential writing without having to worry over formatting or wasting time arguing with your word processor.
%
%\LaTeX{} is easily able to professionally typeset documents that run to hundreds or thousands of pages long. With simple mark-up commands, it automatically sets out the table of contents, margins, page headers and footers and keeps the formatting consistent and beautiful. One of its main strengths is the way it can easily typeset mathematics, even \emph{heavy} mathematics. Even if those equations are the most horribly twisted and most difficult mathematical problems that can only be solved on a super-computer, you can at least count on \LaTeX{} to make them look stunning.
%
%%----------------------------------------------------------------------------------------
%
%\section{Welcome and Thau}\label{sec:welcoe}
%Welcome to this \LaTeX{} Thesis Template, a beautiful and easy to use template for writing a thesis using the \LaTeX{} typesetting system.
%
%If you are
%\begin{table}
%
%    \label{tab:treatments}
%    \centering
%%    \begin{tabular}{l l l}
%%        \toprule
%%        \tabhead{Groups} & \tabhead{Treatment X} & \tabhead{Treatment Y} \\
%%        \midrule
%%        1 & 0.2 & 0.8\\
%%        2 & 0.17 & 0.7\\
%%        3 & 0.24 & 0.75\\
%%        4 & 0.68 & 0.3\\
%%        \bottomrule\\
%%    \end{tabular}
%    \begin{tabular}{c r @{.} l}
%        Pi expression       &
%        \multicolumn{2}{c}{Value} \\
%        \hline
%        $\pi$               & 3&1416  \\
%        $\pi^{\pi}$         & 36&46   \\
%        $(\pi^{\pi})^{\pi}$ & 80662&7 \\
%    \end{tabular}
%    \caption{The effects of treatments X and Y on the four groups studied.}
%\end{table}
%writing a thesis (or will be in the future) and its subject is technical or mathematical (though it doesn't have to be), then creating it in \LaTeX{} is highly recommended as a way to make sure you can just get down to the essential writing without having to worry over formatting or wasting time arguing with your word processor.
%
%\LaTeX{} is easily able to professionally typeset documents that run to hundreds or thousands of pages long. With simple mark-up commands, it automatically sets out the table of contents, margins, page headers and footers and keeps the formatting consistent and beautiful. One of its main strengths is the way it can easily typeset mathematics, even \emph{heavy} mathematics. Even if those equations are the most horribly twisted and most difficult mathematical problems that can only be solved on a super-computer, you can at least count on \LaTeX{} to make them look stunning.
%
%%----------------------------------------------------------------------------------------
%
%\section{Welcome and Tnk You}\label{sec:wlcome}
%Welcome to this \LaTeX{} Thesis Template, a beautiful and easy to use template for writing a thesis using the \LaTeX{} typesetting system.
%
%If you are writing a thesis.
%
%%\begin{verbatim}
%\begin{figure}
%    \centering
%    \includegraphics{Figures/Electron}
%    %    \decoRule
%    \caption[An Electron]{An electron (artist's impression).}
%    \label{fig:Electron}
%\end{figure}
%%\end{verbatim}
%(or will be in the future) and its subject is technical or mathematical (though it doesn't have to be), then creating it in \LaTeX{} is highly recommended as a way to make sure you can just get down to the essential writing without having to worry over formatting or wasting time arguing with your word processor.
%
%\LaTeX{} is easily able to professionally typeset documents that run to hundreds or thousands of pages long. With simple mark-up commands, it automatically sets out the table of contents, margins, page headers and footers and keeps the formatting consistent and beautiful. One of its main strengths is the way it can easily typeset mathematics, even \emph{heavy} mathematics. Even if those equations are the most horribly twisted and most difficult mathematical problems that can only be solved on a super-computer, you can at least count on \LaTeX{} to make them look stunning.
%
%%----------------------------------------------------------------------------------------