\chapter{Introduction}\label{ch:chapter1} % For referencing the chapter elsewhere, use \ref{Chapter1}

%----------------------------------------------------------------------------------------
Video is the medium to record, copy, playback, broadcast
and display the motion images in an electronic style~\parencite{RN190}.
Watching videos is becoming an important way for our entertainment as well
as education.
The high definition (HD) and ultra high definition (UHD) video
are increasingly demanding nowadays.
People prefer videos with higher definitions than those with lower
resolutions because the former one provides much better viewing experience.
However, challenges emerged for delivering videos with high definition.
HD videos typically contain much more information in every picture frame than the
standard definition videos.
More data needs to be squeezed into the same capacity for transmission.
For example, the uncompressed video with the dimension 720 x 480 at 30 frames
per second requires 0.03 gigabytes per second, while the uncompressed video with
the dimension 2880 x 2048 at 120 frames per second requires 2.12 gigabytes per
second.
Since bit rate is proportional to system bandwidth for
transmission~\parencite{RN191}, and expanding the bandwidth in a large scale is
too expensive, the significantly increased bit rate
for transmitting the video data is becoming one of the
major obstacles for HD video services.\\
\newline
To cope with the growing need for higher compression of moving
pictures~\parencite{RN193}, Joint Collaborative Team on Video
Coding (JCT-VC)~\parencite{RN192} has developed the High Efficiency Video
Coding standard which is the newest international video coding standard for
substantially ameliorate the compression performance against the previous
standards.
Comparing with the H.264 Advanced Video Compression Standard~\parencite{RN194},
the H.265 High Efficiency Video Coding Standard provides fifty percent bit rate
reduction while maintaining the objective video quality at the same level.\\
\newline
While Two-dimensional video is the most common video type,
Three-dimensional (3D) video has been brought to market via lots of ways,
including Blu-Ray disc, cable and satellite transmission, terrestrial
broadcast, and streaming or downloading from the Internet~\parencite{RN118}.
3D video provides the perception of depth information which augments
the vividness of the video contents.
Currently most 3D videos in the market are using stereo display technology.
Two similar views, one for left eye, the other for right eye, are presented
at the same time with the multiplexing techniques enabling the
adjustments of video geometry information~\parencite{RN196} to provide
the 3D effect.
Figure~\ref{fig:stereo-display} illustrates the typical system structure for
transmitting videos targeting stereo display.
\begin{figure}
    \centering
    \includegraphics[width=\textwidth,height=\textheight,keepaspectratio]{Figures/StereoDisplay}
%        \decoRule
    \caption[System Structure for transmitting videos targeting stereo display]{System Structure for transmitting videos targeting stereo display.}
    \label{fig:stereo-display}
\end{figure}
It can be observed that there exists a displacement between the
two views.
The green vertical left margins of the red rectangles in the two views
at encoder side are different.
Such a displacement is the visual disparity for 3D perception.
Stereoscopic videos~\parencite{RN153} have
achieved great profitability for movie theatres in recent years.
For example, IMAX 3D has became the most popular one that offering
the immersing multimedia experiences around the world.
Special 3D glasses are needed for watching the IMAX 3D movies.
The current 3D film industry is very successful in terms of attracting
customers, however, it is not the end of the story.
Myopic people do not like to wear one more pair of glasses when
watching 3D movies.
Some people will experience discomfort after wearing the 3D glasses for a
period of two hours.
To get rid of the undesired 3D glasses,
autostereoscopic multi-view technology~\parencite{RN153} is coming to
our rescue.
The two major different characteristics between stereo display and
autostereoscopic display are
listed in Table~\ref{tab:diff-stereo-autostereo}~\parencite{RN44}.
\begin{table}[b]
    \caption{Characteristics Comparison of Stereoscopic Display and Autostereoscopic Display}
    \bigskip
    \label{tab:diff-stereo-autostereo}
    \centering
    \begin{tabular}{c c c}
        \hline
        Characteristic & Stereo Display & Autostereoscopic Display\\
        \hline
        Glass-Free & No & Yes \\
        Multiple Stereo Pairs & No & Yes \\
%        Number of Views & two views & more than two views \\
%        Overall Display Resolution & High & Low \\
%        Perceivable Scene Depth Quality & High & Low \\
        \hline
    \end{tabular}
\end{table}
The impact of different view numbers for autostereoscopic display is shown in
Table~\ref{tab:autostereo-less-views-more-views}~\parencite{RN44}.
\begin{table}
    \caption{Impact of Available View Amount for Autostereoscopic Display}
    \bigskip
    \label{tab:autostereo-less-views-more-views}
    \centering
    \begin{tabular}{c c c}
        \hline
        Characteristic & Small Number of Views & Large Number of Views \\
        \hline
%        Overall Display Resolution & High & Low \\
        Seamless View Transition  & No & Yes \\
        High Quality of Scene Depth & No & Yes \\
        \hline
    \end{tabular}
\end{table}
Comparative ease can be brought to the 3D video audience
since they do not need to wear 3D glasses for watching autostereoscopic videos.
At each different view position, scenes with minor differences are available
from multiple stereo pairs which are provided by autostereoscopic
display~\parencite{RN44}.
As a result, when audience make a move for various view positions, scenes
not viewable from the previous locations are revealed during the movement.
The autostereoscopic multi-view display demands more than two views.
With a sufficient amount of views present in autostereoscopic display, the
disparities between every two adjacent views can be small enough to offer
seamless transitions from scene to scene, such that when multiple views
meet eyes sequentially, the scenes as a whole can be gorgeous.
The visual quality of the autostereoscopic display is highly proportional to
the number of available views.
Due to limited available bandwidth, transmitting arbitrary number of views
is not practical.
Researchers have proposed a new format which only requires limited number
of view and their associated depth maps for the capability of
generating arbitrary amount of views theoretically.
The typical system structure for using this new format to compress and supply 3D video
resources is shown in Figure~\ref{fig:SS-MVD}.
An enormous amount of views in the medium positions which are able to
guarantee the high quality of the 3D video can be synthesized from
\begin{figure*}[!b]
    \centering
    \includegraphics[width=\textwidth,height=\textheight,keepaspectratio]{Figures/SystemStructureOf3DEncoder}
%        \decoRule
    \caption[System Structure for transmitting videos of Multi-view Plus Depth format]{System Structure for transmitting videos of Multi-view Plus Depth format.}
    \label{fig:SS-MVD}
\end{figure*}
the decoded texture frames in combination with decoded depth maps.\\
%The multi-view plus depth format provides the functionality of synthesizing
%required number of views from texture views and associated depth maps.\\
\newline
To employ multi-view plus depth format for 3D video, efficient compressing
methods are desired, which has led to the 3D Video Coding Extension of the
High Efficiency Video Coding Standard (3D-HEVC) by the Joint Collaborative Team
on 3D Video Coding Extension Development (JCT-3V)~\parencite{RN195}.
The 3D Extension of the HEVC standard gives extra coding efficiency
for encoding a few texture views along with the corresponding depth maps by
using new tools which exploit the redundancies amongst
texture and depth views, and pay attention to the unique characteristics of
the depth maps, such as large homogeneous
regions separated by sharp boundaries~\parencite{RN47}.\\
\newline
Depth information measures of the distance between the object in the far position
and the object in the near position from a static viewpoint,
which is expressed in the format of depth map.
\begin{figure*}[!t]
    \centering
    \includegraphics[width=\textwidth,height=\textheight,keepaspectratio]{Figures/wedgelet}
%        \decoRule
    \caption[Wedgelet partition illustration]
    {Example of wedgelet partition in a block of size 16 by 16 in depth map
    from Shark video sequence.
    }
    \label{fig:wedgelet-partition}
\end{figure*}
Instead of presenting depth maps directly to the viewer, views in the medium
positions are generated by Depth-Image-Based Rendering (DIBR) technique.
The qualities of the depth maps are vital to the DIBR process.
Corona artifacts (a.k.a. ringing artifacts) can be discovered in synthesized
views if the edge sharpness in depth maps can not be well
preserved.
Therefore, retaining the edge sharpness in depth map is the key to avoid the
artifacts in the synthesized views.
In 3D-HEVC, new intra-picture prediction tools and residual coding methods
have been applied to preserve the special properties of depth maps.
Depth Modelling Mode (DMM) which is one of the new intra-picture
prediction tools, is designed to provide much more granularity for
encoding the depth maps than the normal angular intra prediction modes.
DMM is more capable of approximating the depth maps to be encoded due to
the fact that it provides a vast amount of non-rectangle partitions.
Figure~\ref{fig:wedgelet-partition} presents an example of the wedgelet
partition from the depth map in Shark video sequence.
The small block highlighted by blue color amongst the blocks
separated by the red grid is magnified at the right-bottom position in
Figure~\ref{fig:wedgelet-partition}.
A straight line is used for the partition in wedgelet mode.
Figure~\ref{fig:contour-partition} shows a sample of the contour partition
from the same depth map as Figure~\ref{fig:wedgelet-partition}.
The partition pattern comprises contour lines instead of one single
straight line.
Wedgelet partition and contour partition for depth maps
are enabled by DMM1 and DMM4 separately.
%The wedgelet partition is shown in Figure.
%The contour partition is shown in Figure.
%~\parencite{RN197}.
\begin{figure}
    \centering
    \includegraphics[width=\textwidth,height=\textheight,keepaspectratio]{Figures/contour}
%        \decoRule
    \caption[Contour partition illustration]
    {Example of contour partition in a block of size 16 by 16 in depth map
    from Shark video sequence.
    }
    \label{fig:contour-partition}
\end{figure}
%introduce a little about depth map and their usage.
%mentioning iphonex true depth camera.
%draw the picture

%----------------------------------------------------------------------------------------

\section{Motivation and Contribution}\label{sec:motivation_and_contribution}
The idea of this work originates from the discovery of the computational
complexity of the wedgelet searching process in depth modelling modes.
The immense complexity for searching the best wedgelet candidate lead to
the strongly marked increase of encoding time.
The time consumed for compressing a single depth map in 3D-HEVC encoder is
roughly a sixfold increase relevant to the encoding time of a single texture
frame.
Thus we designed a deep neural network architecture which is trained
subsequently for predicting the most probable wedgelet candidates.
The learned model achieves 92.2\% to 97.3\% top-16 accuracy for various
block sizes.
The inference engine is integrated into the reference software
(HTM16.2) of 3D-HEVC.
The learned models reduce roughly half of the wedgelet searching candidates.
It provides 64.6\% time reduction in average while the BD performance
is well maintained comparing with the unmodified 3D-HEVC encoder.
%%%%%%%%%%%%%%%%%%%%%%%%%%%%%%%%%%%%%%%%%%%%
% Table examples
%%%%%%%%%%%%%%%%%%%%%%%%%%%%%%%%%%%%%%%%%%%%
%\begin{table}
%    \centering
%
%    \begin{tabular}{|l||r|r|r|c|}
%        \hline
%        Name & Exam1 & Exam2 & Exam3 & Grade\\
%        \hline\hline
%        John & 19 & 28 & 33 & C\\
%        \hline
%        Jane & 49 & 35 & 60 & B\\
%        \hline
%        Jim & 76 & 38 & 59 & A\\
%        \hline
%    \end{tabular}
%
%    \caption{Math 500 Grades}
%    \label{math500grades}
%\end{table}
%
%\begin{table}
%    \label{tab:treatments}
%    \centering
%    \begin{tabular}{c r @{.} l}
%        Pi expression       &
%        \multicolumn{2}{c}{Value} \\
%        \hline
%        $\pi$               & 3&1416  \\
%        $\pi^{\pi}$         & 36&46   \\
%        $(\pi^{\pi})^{\pi}$ & 80662&7 \\
%    \end{tabular}
%    \caption{The effects of treatments X and Y on the four groups studied.}
%\end{table}
%
%\begin{table}
%    \label{tab:tabular_example1}
%    \centering
%    \begin{tabular}[t]{|r|l|}
%        \hline
%        7C0 & hexadecimal \\
%        3700 & octal \\
%        \cline{2-2} 11111000000 & binary \\
%        \hline
%        \hline
%        1984 & decimal \\
%        \hline
%    \end{tabular}
%\caption{Just an example}
%\end{table}
%
%\begin{table}
%    \label{tab:tabular_example2}
%    \centering
%    \begin{tabular}{|r|l|}
%        \hline
%        7C0 & hexadecimal \\
%        3700 & octal \\
%%        \cline{2-2}
%%        11111000000 & binary \\
%%        \hline
%%        \hline
%%        1984 & decimal \\
%        \hline
%    \end{tabular}
%\caption{Basic Usage}
%\end{table}
%
%\begin{table}
%    \label{tab:tabular_example3}
%    \centering
%    \begin{tabular}{|r|l|}
%        \hline
%        7C0 & hexadecimal \\
%        \cline{1-2}
%        3700 & octal \\
%        \cline{2-2}
%        11111000000 & binary \\
%        \cline{1-2}
%        11111000 & binary \\
%%        \hline
%%        \hline
%%        1984 & decimal \\
%        \hline
%    \end{tabular}
%\caption{horizontal lines extend over multiple columns}
%\end{table}
%
%\begin{table}
%    \label{tab:tabular_example4}
%    \centering
%    \begin{tabular}{|p{4.7cm}|}
%        \hline Welcome to Boxy's paragraph. We sincerely hope you'll all enjoy the show.\\
%        \hline
%    \end{tabular}
%    \caption{define a special type of column which will wrap-around the text as in a normal paragraph}
%\end{table}
%
%\begin{table}
%    \label{tab:tabular_example8}
%    \centering
%    \begin{tabular}{p{4.7cm}}
%        \hline
%        Welcome to Boxy's paragraph.
%        We sincerely hope you'll all enjoy the show.\\
%        \hline
%    \end{tabular}
%    \caption{define a special type of column which will wrap-around the text as in a normal paragraph}
%\end{table}
%
%\begin{table}
%    \label{tab:tabular_example5}
%    \centering
%    \begin{tabular}{@{} l @{}}
%        \hline Welcome to Boxy's paragraph. We sincerely hope you'll all enjoy the show.\\
%        \hline
%    \end{tabular}
%    \caption{define a special type of column which will wrap-around the text as in a normal paragraph}
%\end{table}
%
%\begin{table}
%    \label{tab:tabular_example6}
%    \centering
%    \begin{tabular}{l}
%        \hline Welcome to Boxy's paragraph. We sincerely hope you'll all enjoy the show.\\
%        \hline
%    \end{tabular}
%    \caption{define a special type of column which will wrap-around the text as in a normal paragraph}
%\end{table}
%
%\begin{table}
%    \label{tab:tabular_example9}
%    \centering
%    \begin{tabular}{c c} \hline \multicolumn{2}{c}{Ene} \\ \hline Mene & Muh! \\ \hline \end{tabular}
%    \caption{multicolumn command}
%\end{table}
%
%\begin{table}
%    \label{tab:tabular_example10}
%    \centering
%    \begin{tabular}{c c c c c c}
%        \hline
%        sequence name &
%        BD-BR &
%        \multicolumn{4}{c}{Ene} \\
%        \cline{3-6}
%        {} & {} & Me & Muh & Me & Mu\\
%        \hline
%        Newspaper & 0.98\% & 22 & 33 & 44 & 66\\
%    \end{tabular}
%    \caption{multicolumn command}
%\end{table}




%\begin{table}
%    \label{tab:treatments}
%    \centering
%    \begin{tabular}{c r @{.} l}
%        Pi expression       &
%        \multicolumn{2}{c}{Value} \\
%        \hline
%        $\pi$               & 3&1416  \\
%        $\pi^{\pi}$         & 36&46   \\
%        $(\pi^{\pi})^{\pi}$ & 80662&7 \\
%    \end{tabular}
%    \caption{The effects of treatments X and Y on the four groups studied.}
%\end{table}
%----------------------------------------------------------------------------------------

\section{Dissertation Outline}\label{sec:outline}




%\chapter{Introduction}\label{ch:chapter1} % For referencing the chapter elsewhere, use \ref{Chapter1}
%
%%----------------------------------------------------------------------------------------
%
%%----------------------------------------------------------------------------------------
%
%\section{Welcome and Thank You}\label{sec:welcome}
%Welcome to this \LaTeX{} Thesis Template, a beautiful and easy to use template for writing a thesis using the \LaTeX{} typesetting system.
%
%If you are writing a thesis (or will be in the future) and its subject is technical or mathematical (though it doesn't have to be), then creating it in \LaTeX{} is highly recommended as a way to make sure you can just get down to the essential writing without having to worry over formatting or wasting time arguing with your word processor.
%
%\LaTeX{} is easily able to~\parencite{RN93} professionally typeset documents that run to hundreds or thousands of pages long. With simple mark-up commands, it automatically sets out the table of contents, margins, page headers and footers and keeps the formatting consistent and beautiful. One of its main strengths is the way it can easily typeset mathematics, even \emph{heavy} mathematics. Even if those equations are the most horribly twisted and most difficult mathematical problems that can only be solved on a super-computer, you can at least count on \LaTeX{} to make them look stunning.
%
%%----------------------------------------------------------------------------------------
%
%\section{Welcome and Thanku}\label{sec:welome}
%Welcome to this \LaTeX{} Thesis Template, a beautiful and easy to use template for writing a thesis using the \LaTeX{} typesetting system.
%
%If you are writing a thesis (or will be in the future) and its subject is technical or mathematical (though it doesn't have to be), then creating it in \LaTeX{} is highly recommended as a way to make sure you can just get down to the essential writing without having to worry over formatting or wasting time arguing with your word processor.
%
%\LaTeX{} is easily able to professionally typeset documents that run to hundreds or thousands of pages long. With simple mark-up commands, it automatically sets out the table of contents, margins, page headers and footers and keeps the formatting consistent and beautiful. One of its main strengths is the way it can easily typeset mathematics, even \emph{heavy} mathematics. Even if those equations are the most horribly twisted and most difficult mathematical problems that can only be solved on a super-computer, you can at least count on \LaTeX{} to make them look stunning.
%
%%----------------------------------------------------------------------------------------
%
%\section{Welcome and ThYou}\label{sec:weome}
%Welcome to this \LaTeX{} Thesis Template~\parencite{Reference1}, a beautiful and easy to use template for writing a thesis using the \LaTeX{} typesetting system.
%
%If you are writing a thesis (or will be in the future) and its subject is technical or mathematical (though it doesn't have to be), then creating it in \LaTeX{} is highly recommended as a way to make sure you can just get down to the essential writing without having to worry over formatting or wasting time arguing with your word processor.
%
%\LaTeX{} is easily able to professionally typeset documents that run to hundreds or thousands of pages long. With simple mark-up commands, it automatically sets out the table of contents, margins, page headers and footers and keeps the formatting consistent and beautiful. One of its main strengths is the way it can easily typeset mathematics, even \emph{heavy} mathematics. Even if those equations are the most horribly twisted and most difficult mathematical problems that can only be solved on a super-computer, you can at least count on \LaTeX{} to make them look stunning.
%
%%----------------------------------------------------------------------------------------
%
%\section{Welcome and Thau}\label{sec:welcoe}
%Welcome to this \LaTeX{} Thesis Template, a beautiful and easy to use template for writing a thesis using the \LaTeX{} typesetting system.
%
%If you are
%\begin{table}
%
%    \label{tab:treatments}
%    \centering
%%    \begin{tabular}{l l l}
%%        \toprule
%%        \tabhead{Groups} & \tabhead{Treatment X} & \tabhead{Treatment Y} \\
%%        \midrule
%%        1 & 0.2 & 0.8\\
%%        2 & 0.17 & 0.7\\
%%        3 & 0.24 & 0.75\\
%%        4 & 0.68 & 0.3\\
%%        \bottomrule\\
%%    \end{tabular}
%    \begin{tabular}{c r @{.} l}
%        Pi expression       &
%        \multicolumn{2}{c}{Value} \\
%        \hline
%        $\pi$               & 3&1416  \\
%        $\pi^{\pi}$         & 36&46   \\
%        $(\pi^{\pi})^{\pi}$ & 80662&7 \\
%    \end{tabular}
%    \caption{The effects of treatments X and Y on the four groups studied.}
%\end{table}
%writing a thesis (or will be in the future) and its subject is technical or mathematical (though it doesn't have to be), then creating it in \LaTeX{} is highly recommended as a way to make sure you can just get down to the essential writing without having to worry over formatting or wasting time arguing with your word processor.
%
%\LaTeX{} is easily able to professionally typeset documents that run to hundreds or thousands of pages long. With simple mark-up commands, it automatically sets out the table of contents, margins, page headers and footers and keeps the formatting consistent and beautiful. One of its main strengths is the way it can easily typeset mathematics, even \emph{heavy} mathematics. Even if those equations are the most horribly twisted and most difficult mathematical problems that can only be solved on a super-computer, you can at least count on \LaTeX{} to make them look stunning.
%
%%----------------------------------------------------------------------------------------
%
%\section{Welcome and Tnk You}\label{sec:wlcome}
%Welcome to this \LaTeX{} Thesis Template, a beautiful and easy to use template for writing a thesis using the \LaTeX{} typesetting system.
%
%If you are writing a thesis.
%
%%\begin{verbatim}
%\begin{figure}
%    \centering
%    \includegraphics{Figures/Electron}
%    %    \decoRule
%    \caption[An Electron]{An electron (artist's impression).}
%    \label{fig:Electron}
%\end{figure}
%%\end{verbatim}
%(or will be in the future) and its subject is technical or mathematical (though it doesn't have to be), then creating it in \LaTeX{} is highly recommended as a way to make sure you can just get down to the essential writing without having to worry over formatting or wasting time arguing with your word processor.
%
%\LaTeX{} is easily able to professionally typeset documents that run to hundreds or thousands of pages long. With simple mark-up commands, it automatically sets out the table of contents, margins, page headers and footers and keeps the formatting consistent and beautiful. One of its main strengths is the way it can easily typeset mathematics, even \emph{heavy} mathematics. Even if those equations are the most horribly twisted and most difficult mathematical problems that can only be solved on a super-computer, you can at least count on \LaTeX{} to make them look stunning.
%
%%----------------------------------------------------------------------------------------
