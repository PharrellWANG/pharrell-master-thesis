% Stanford University PhD thesis style -- modifications to the report style
% This is unofficial so you should always double check against the
% Registrar's office rules
% See http://library.stanford.edu/research/bibliography-management/latex-and-bibtex
% 
% Example of use below
% See the suthesis-2e.sty file for documentation
%
\documentclass{report}
\usepackage{suthesis-2e}
\usepackage{graphicx}
\usepackage{verbatim} % for block comment
\usepackage{color}   % May be necessary if you want to color links
\usepackage{hyperref}
\usepackage{textcomp}
\usepackage[backend=bibtex,style=ieee,natbib=true]{biblatex} % Use the bibtex backend with the authoryear citation style (which resembles APA)
\addbibresource{../src/mybib.bib} % The filename of the bibliography
\hypersetup{
colorlinks=false, %set true if you want colored links
linktoc=all,     %set to all if you want both sections and subsections linked
linkcolor=black,  %choose some color if you want links to stand out
}
\dept{Electronic and Information Engineering}

\begin{document}
    \title{Fast Depth Coding in 3D-HEVC\\
    Using Deep Learning}
    \author{Zhen-xiang WANG}
    \principaladviser{Yui-Lam Chan}
    \beforepreface
    \prefacesection{Abstract}
    The 3D Extension of the High Efficiency Video Coding standard (3D-HEVC),
    which has been finalized by the Joint Collaborative Team on Video Coding
    (JCT-VC) in February 2015, is the new industry standard for 3D applications.
    The 3D-HEVC provides plenty of advanced coding tools specifically
    for addressing the coding of auto-stereoscopic videos which have the format
    of multiple texture views along with the depth maps which are responsible
    for synthesising intermediate views with sufficient quality for
    auto-stereoscopic display.
    The provided tools take advantage of the statistical redundancies amongst
    texture views and depth maps in the video sequences, as well as the unique
    characteristics of depth maps to significantly shrink the bit-rate
    while preserving the objective visual quality of the
    3D videos.
    However, those tools with high capability in terms of compression come
    with the high complexity of computation which has made the encoding time
    of the 3D video sequences much longer than ever by traversing a lot more
    candidates, calculating time-consuming RD Cost for each of them,
    especially in the wedgelet searching process for depth maps.
    While this full-search style method can promise to find the best
    candidate in depth intra mode decision, the time cost is expensive.\\
    \newline
    In this dissertation we address the time cost issue by presenting a new
    intra mode decision method for depth maps, making use of the deep
    convolutional neural networks to predict the possible modes for the
    depth blocks.
    The predictions from the learned models are capable of
    helping the encoders to reduce the number of mode candidates by half
    both in the generic angular modes which are applicable to textures
    and Depth Modelling Modes specifically for depth maps.
    The size of the neural network has been carefully designed to balance
    the trade-off between the cost of prediction time and the prediction
    accuracy.
    The confusion matrix has been used to monitor the training process.
    The top-16 criteria has been employed for the prediction.
    We have integrated the learned models into the reference software of
    3D-HEVC for the experiments.
    The compiled executable binaries are able to harness
    the power of the simultaneous computation of CPU, as well as the power of
    the parrallel computation of GPU to accelerate the predictions.
    The simulation results show that the proposed algorithm powered by
    deep neural nets provides 56.3\% time reduction in average while the
    BD performance has no decrease comparing with the implementation of
    the 3D-HEVC standard.

    \prefacesection{Acknowledgments}
    The work in this dissertation.....
    \afterpreface

    % !TEX root = ../main.tex
\chapter{Introduction}\label{ch:chapter1} % For referencing the chapter elsewhere, use \ref{Chapter1}

%----------------------------------------------------------------------------------------
Video is the medium to record, copy, playback, broadcast
and display the motion images in an electronic style~\parencite{RN190}.
Watching videos is becoming an important way for human entertainment as well
as education.
The high definition (HD) and ultra high definition (UHD) videos
are increasingly demanding nowadays.
People prefer videos with higher resolution than those with lower
resolution because HD videos provide better watching experience.
However, challenges emerged for delivering videos with high definition.
HD/UHD videos typically contain much more information in every
picture frame than videos with lower resolution.
More data needs to be squeezed into the same capacity for transmission.
For example, the uncompressed video with the dimension \(720\times480\) at 30 frames
per second requires 0.03 gigabytes per second, while the uncompressed video with
the dimension \(2880\times2048\) at 120 frames per second requires 2.12 gigabytes per
second.
Since bitrate is proportional to system bandwidth for
transmission~\parencite{RN191}, and heavily expanding the
bandwidth is usually expensive, the significantly increased bitrate
for transmitting the video data is becoming one of the
major obstacles for HD video services.

To cope with the growing need for higher compression of moving
pictures~\parencite{RN193}, Joint Collaborative Team on Video
Coding (JCT-VC)~\parencite{RN192} has finalized the High Efficiency Video
Coding which is the newest international video coding standard for
substantially ameliorate the compression performance against the previous
standards.
Comparing with H.264 Advanced Video Compression Standard~\parencite{RN194},
H.265 High Efficiency Video Coding Standard provides a reduction 
of fifty percent in terms of bitrate while maintaining the objective
video quality at the same level.

Three-dimensional (3D) video has been introduced to market via lots of ways,
including Blu-Ray disc, cable and satellite transmission, terrestrial
broadcast, and streaming or downloading from the Internet~\parencite{RN118}.
3D video provides the perception of depth information which augments
the vividness of video contents.
Currently most 3D videos in the market are using stereo display technology.
Two similar views, one for left eye, the other for right eye, are presented
at the same time with the multiplexing techniques enabling the
adjustments of video geometry information~\parencite{RN196} to provide
the 3D effect.
Figure~\ref{fig:stereo-display} illustrates the typical system structure for
transmitting videos targeting stereo display.
\begin{figure}
    \centering
    \includegraphics[
        width=\textwidth,
        height=\textheight,
        keepaspectratio
        ]{Figures/StereoDisplay.pdf}
    \caption[System Structure for transmitting videos targeting 
    stereo display]{System Structure for transmitting videos
     targeting stereo display.}\label{fig:stereo-display}
\end{figure}
It can be observed that there exists a displacement between
two views.
The green vertical left margins of the red rectangles in two views
at encoder side are different with each other.
Such a displacement is the visual disparity for 3D perception.
Stereoscopic videos~\parencite{RN153} have
achieved great profitability for movie theatres in recent years.
For example, IMAX 3D has became the most popular one that offering
the immersing multimedia experiences around the world.
Special 3D glasses are needed for watching IMAX 3D movies.
The current 3D film industry is very successful in terms of attracting
customers, however, it is not the end of the story.
Myopic people do not like to wear one more pair of glasses when
watching 3D movies.
Some people will experience discomfort after wearing 3D glasses 
for hours.
To get rid of the undesired 3D glasses,
autostereoscopic multi-view technology~\parencite{RN153} is coming to
the rescue.
The two major different characteristics between stereo display and
autostereoscopic display are
listed in Table~\ref{tab:diff-stereo-autostereo}~\parencite{RN44}.
\begin{table}[b]
    \caption{Comparing characteristics of stereoscopic display and autostereoscopic display}
    \bigskip\label{tab:diff-stereo-autostereo}
    \centering
    \begin{tabular}{c c c}
        \toprule
        Characteristic & Stereo Display & Autostereoscopic Display\\
        \midrule
        Glass-Free & No & Yes \\
        Multiple Stereo Pairs & No & Yes \\
        \bottomrule
    \end{tabular}
\end{table}
The impact of available view amount for autostereoscopic display is shown in
Table~\ref{tab:autostereo-less-views-more-views}~\parencite{RN44}.
\begin{table}
    \caption{The impact of available view amount for autostereoscopic display}
    \bigskip\label{tab:autostereo-less-views-more-views}
    \centering
    \begin{tabular}{c c c}
        \toprule
        Characteristic & Small Number of Views & Large Number of Views \\
        \midrule
        Seamless View Transition  & No & Yes \\
        High Quality of Scene Depth & No & Yes \\
        \bottomrule
    \end{tabular}
\end{table}
Comparative ease can be brought to the 3D video audience
since they do not need to wear 3D glasses for watching autostereoscopic videos.
At each different view position, scenes with minor differences are available
from multiple stereo pairs which are provided by autostereoscopic
display~\parencite{RN44}.
As a result, when audience make moves for various view positions, scenes
not viewable from the previous locations are revealed during the movements.
The autostereoscopic multi-view display demands more than two views.
With a sufficient amount of views present in autostereoscopic display, the
disparities between every two adjacent views can be small enough to offer
seamless transitions from scene to scene, such that when multiple views
meet eyes sequentially, the scenes as a whole can be gorgeous.
The visual quality of the autostereoscopic display is highly proportional to
the number of available views.
Due to limited available bandwidth, transmitting arbitrary number of views
is not practical.
Researchers have proposed a new format which only requires limited number
of views and their associated depth maps for the capability of
generating arbitrary amount of views.
The typical system structure using this new format to compress and supply 3D video
resources is shown in Figure~\ref{fig:SS-MVD}.
An enormous number of views in medium positions which are able to
guarantee the high quality of 3D display can be synthesized from
\begin{figure*}[!b]
    \centering
    \includegraphics[width=\textwidth,height=\textheight,keepaspectratio]{Figures/SystemStructureOf3DEncoder}
%        \decoRule
    \caption[System Structure for transmitting videos of Multi-view 
    Plus Depth format]{System Structure for transmitting videos of 
    Multi-view Plus Depth format.}\label{fig:SS-MVD}
\end{figure*}
decoded texture frames in combination with decoded depth maps.
%The multi-view plus depth format provides the functionality of synthesizing
%required number of views from texture views and associated depth maps.\\

To employ multi-view plus depth format for 3D video, efficient compressing
methods are needed, which leads to the 3D Extension of
High Efficiency Video Coding Standard (3D-HEVC) by the Joint Collaborative Team
on 3D Video Coding Extension Development (JCT-3V)~\parencite{RN195}.
The 3D Extension of HEVC standard provides extra coding efficiency
for encoding texture views along with the corresponding depth maps by
using new tools. 
Those new tools exploit statistical redundancies between
texture views and depth maps, and pay attention to the unique characteristics of
depth maps, such as large homogeneous
regions separated by sharp boundaries~\parencite{RN47}.

% The distance between distant views
% and nearby views from a static viewpoint,
% can be expressed in the format of depth map.
Depth map vividly conveys the distance between distant views
and nearby views.
\begin{figure*}[!t]
    \centering
    \includegraphics[width=\textwidth,height=\textheight,keepaspectratio]{Figures/wedgelet}
%        \decoRule
    \caption[Wedgelet partition illustration]
    {Example of wedgelet partition in a block of size 
    \(16\times16\) in a depth map
    from Shark video sequence.
    The tiny block highlighted by transparent blue color
    is magnified then shown in the right bottom corner
    of the depth map.
    Wedgelet partitions are straight lines
    that tries their best to fit the sharp edges in CU blocks.
    }\label{fig:wedgelet-partition}
\end{figure*}
Instead of presenting depth maps as 
intermediate views directly to audience, views in the medium
positions are generated by Depth-Image-Based Rendering (DIBR) technique.
The quality of depth maps is vital to the results produced by
DIBR process.
Corona artifacts (a.k.a.\ ringing artifacts)~\parencite{RN44}
can be discovered in synthesized
views if sharpness of edge in depth maps can not be well
preserved.
Therefore, retaining edge sharpness in depth maps is the key to avoid the
artifacts in synthesized views.
In 3D-HEVC, new intra-picture prediction tools and new residual coding methods
have been applied to adapt encoding to special properties of depth maps.
Depth Modelling Mode (DMM), which is one of the new intra-picture
prediction tools, is capable of conserving sharp edges in the shape
of straight line.
It provides a very dense set of straight partitions
which will be looped through in order to find the most suitable candidate
for each in the CU blocks in depth maps.
Figure~\ref{fig:wedgelet-partition} presents an example of wedgelet
partition in a depth map from Shark video sequence.
The small block highlighted by blue color amongst the blocks
separated by the red grid is magnified at the right-bottom position.
Straight lines are used for the partitions in wedgelet mode.
Figure~\ref{fig:contour-partition} shows a sample of the contour partition
from the same depth map shown in Figure~\ref{fig:wedgelet-partition}.
The partition pattern comprises contour line instead of
straight line.
Wedgelet partition and contour partition for depth maps
are enabled by DMM1 and DMM4 separately.
%The wedgelet partition is shown in Figure.
%The contour partition is shown in Figure.
%~\parencite{RN197}.
\begin{figure}
    \centering
    \includegraphics[width=\textwidth,height=\textheight,keepaspectratio]{Figures/contour}
%        \decoRule
    \caption[Contour partition illustration]
    {Example of contour partition in a block of size \(16\times16\) in a depth map
    from Shark video sequence.
    The tiny block highlighted by transparent blue color
    is magnified then shown in the right bottom corner
    of the depth map.
    Contour partitions are irregular lines
    that tries their best to fit the sharp edges in CU blocks.
    }\label{fig:contour-partition}
\end{figure}
%introduce a little about depth map and their usage.
%mentioning iphonex true depth camera.
%draw the picture

%----------------------------------------------------------------------------------------

\section{Motivation}\label{sec:motivation_and_contribution}
The idea of this work originates from the discovery of computational
complexity inside the process for finding the best wedgelet in DMM1.
The immense intricacy for searching the best wedgelet candidate results in
a massive increase of encoding time.
The time consumed for compressing a single depth map in 3D-HEVC encoder is
roughly a sixfold increase relevant to the encoding time of a single texture
frame wherein All-Intra configuration in \(HTM16.2\) is used.
To reduce the heavy time required in depth map coding, 
a computational model, which is a deep convolutional neural network,
has been designed and trained to predict the most probable angular mode
for each CU block.
The predicted angular mode is utilized to further predict the 
most probable wedgelet candidates.
% Thus a computational model has which has been trained
% for predicting the most probable angular mode, which in turn 
% can help with reducing the complexity in DMM1.
The learned models exhibit 91.9\% to 97.0\% top-15 precision for various
block sizes.
It has been integrated into the reference software
\(HTM16.2\) of 3D-HEVC\@.
The learned models can reduce roughly half of the wedgelet candidates.
It provides 64.6\% time reduction in average while the BD performance
has a negligible decrease comparing with the original
implementation of 3D-HEVC encoder.

\textbf{Motivation for Wedgelet Candidates Reduction:} The time
consumed by the encoder from \(HTM16.2\) for
each view can be observed from command line outputs.

Figure~\ref{fig:encoding-time-example} shows a piece of command line outputs
from the encoding process of Shark sequence.
\begin{figure}
    \centering
    \includegraphics[width=\textwidth,height=\textheight,keepaspectratio]{Figures/EncodingTimeEg}
%        \decoRule
    \caption[An example showing a piece of the command line outputs during
    the encoding process for Shark sequence]
    {An example showing a piece of the command line outputs during the
    encoding process for Shark sequence.
    }\label{fig:encoding-time-example}
\end{figure}
\begin{figure*}[!b]
    \centering
    \includegraphics[width=\textwidth,height=\textheight,keepaspectratio]{Figures/major-time-spent-in-recursive-xcompresscu}
%        \decoRule
    \caption[A screen capture of the time profiling information for Newspaper sequence]
    {A screen capture of the time profiling information for Newspaper sequence.
    }\label{fig:major-time-spent-in-recursive-comresscu}
\end{figure*}
The numbers in red blocks stands for the encoding time of certain views, while
the corresponding layer Id and Picture Order Count (POD) are in the green
blocks.
A repetitive pattern of the encoding time for each view can be observed
every six numbers vertically.
A simple calculation using six numbers within the top-most
red block, \((90+69+69)/(90+69+69+14*3) \approx 0.84\), shows that
approximately 84\% total encoding time is busy with encoding
depth maps.
Similarly, it is reported in~\parencite{RN111} that the 
encoding for depth maps
consumes nearly 86\% total 3D-HEVC encoding time.
A trial of time profiling for 3D-HEVC encoder is performed using Instruments
which is an application available on macOS\@.
After encoding the Newspaper sequence for more than one hour,
Figure~\ref{fig:major-time-spent-in-recursive-comresscu} clearly shows
97.8\% time is used to compress the CUs recursively.
The first recursive \textbf{xCompressCU} function 
(denoted as \textbf{XC1} thereafter) is
for CUs of size \(64\times64\), 
the second one
(denoted as \textbf{XC2} thereafter) is targeting 
CUs of size \(32\times32\),
the third one (denoted as \textbf{XC3} thereafter) is dedicated
to CUs of size \(16\times16\), 
and the last one 
(denoted as \textbf{XC4} thereafter) is bound
to CUs of size \(8\times8\).
It can observed from 
Figure~\ref{fig:major-time-spent-in-recursive-comresscu} 
that the most time consuming part
during the process of
compressing depth CUs is DMM1 searching.
The time percentage that DMM1 searching time has in total time occupied 
by \textbf{xCompressCU} is summarized in
Table~\ref{tab:dmm1-searching-time-percent-summary} wherein the summary
for \textbf{XC1} is omitted since DMM1 is not 
applicable to CUs of size \(64\times64\) in
\(HTM16.2\).
\begin{table}[t]
    \caption{The percentages that DMM1 searching time has in total time occupied 
    by \textbf{xCompressCU}}
    \bigskip\label{tab:dmm1-searching-time-percent-summary}
    \centering
    \begin{tabular}{c c c}
        \toprule
        size & xCompressCU & Percentage\\
        \midrule
%        Overall Display Resolution & High & Low \\
        \(32\times32\)  & \textbf{XC2} & 30.0\% \\
        \(16\times16\) & \textbf{XC3} & 25.6\% \\
        \(8\times8\) & \textbf{XC4} & 18.8\% \\
        \bottomrule
    \end{tabular}
\end{table}[t]
The major reason leading to the time consuming property of DMM1 searching is the
View Synthesis Optimization (VSO) Method for improving quality of
synthesized views~\parencite{RN124}, in which the Synthesized View Distortion
Change (SVDC) is computed.
The time percentage that VSO has in DMM1 searching are summarized in
Table~\ref{tab:vso-in-dmm1-searching-time-percent-summary}.
\begin{table}[t]
    \caption{The time percentage that VSO has in DMM1 searching}
    \bigskip\label{tab:vso-in-dmm1-searching-time-percent-summary}
    \centering
    \begin{tabular}{c c c}
        \toprule
        size & Process & Percentage\\
        \midrule
%        Overall Display Resolution & High & Low \\
        \(32\times32\)  & VSO in DMM1 searching from \textbf{XC2} & 80.1\% \\
        \(16\times16\) & VSO in DMM1 searching from \textbf{XC3} & 83.7\% \\
        \(8\times8\) & VSO in DMM1 searching from \textbf{XC4} & 78.8\% \\
        \bottomrule
    \end{tabular}
\end{table}
In \(HTM16.2\), a large number of wedgelet candidates are 
evaluated using VSO which
introduces computational complexity.
Intuitively, evaluating less wedgelet candidates can help with relieving
the heavy burden of computation bared by encoder,
thereby time reduction can be achieved.

\textbf{Motivation for Using Deep Learning:} Deep learning 
is a sub field of representation learning, which
is in turn a major subset of machine learning~\parencite{RN158}.
Machine learning~\parencite{RN198}
has been applied to many scenarios in the domain of Artificial Intelligence (AI).
Deep learning was found hard to proceed further
in the late 1980s~\parencite{RN199}.
However, starting from 2012, it kicks off its
glorious comeback.
The deep Convolutional Neural Network (CNN) has won the ImageNet
Large Scale Visual Recognition Challenge (ILSVRC)
from 2012 to 2015, with the CNN architecture going deeper
and deeper.
The great achievements have attracted attentions from 
people all over the world and
have made deep learning the most popular topic in our daily lives.
Inspired by the fact that supervised deep learning can learn multiple layers of
abstract representations in the visual recognition tasks, it should
be applicable to recognize the angular modes of the intra-picture
prediction in 3D-HEVC\@.
The final DMM1 candidates selected in depth map coding
are essentially determined by the angle pattern of depth blocks.
If we can make use of deep learning to predict the most probable angles of the
target pixel block, a large amount of angular modes and DMM1
wedgelet candidates can be naturally skipped by which the time saving can be
achieved without sacrificing much of the coding performance.

Motivated by the discussions above, we adopt deep learning approach with
deep convolutional neural network to accelerate depth map coding in
3D-HEVC\@.

\section{Contribution and Dissertation Outline}\label{sec:outline}
We accelerate depth map coding using deep learning.
The contributions of the dissertation are:
\begin{itemize}
  \item A deep convolutional neural network with 32 layers 
  comprising ResNet units~\parencite{RN67}
  has been designed and trained to recognize the
  most probable angular mode of coding units (CUs) 
  in intra-picture prediction in 3D-HEVC
  encoder.
  Each learned model has high top-15 precision which works 
  well on tasks of recognizing intra angular patterns in 3D-HEVC\@.
  \item A way of integrating the learned model into \(HTM16.2\) encoder has
  been suggested.
  By making use of Bazel~\parencite{RN200} to compile the encoder binary, the
  data level parallelism (instead of concurrency) functionality in CPU
  as well as the parallel architecture in GPU are fully utilized for
  efficient matrix computations.
  \item An algorithm for fast
  depth map coding, which is shown
  in Figure~\ref{fig:proposed-fast-depth-coding-algorithm}
  on page~\pageref{fig:proposed-fast-depth-coding-algorithm},
  has been proposed and implemented.
  The simulation results show that the proposed algorithm is capable of
  reducing 64.6\% time in wedgelet searching during 3D-HEVC encoding process
  while the BD performance only has a trivial decrease.
\end{itemize}
The first two contributions lay the foundation for the third one, which is the
main objective of this work: to accelerate the depth map encoding process in
3D-HEVC\@.

\textbf{Chapter~\ref{ch:chapter2}} supplies the background of video
coding history, video coding standards, and deep learning using artificial
neural networks.
Prior arts in video coding are surveyed in this chapter.

\textbf{Chapter~\ref{ch:chapter3}} describes the algorithm which has been
implemented to collect the data to be used in deep learning.
Data pre-processing steps are described in details along with
the reasons behind the scene. 
A rich set of visualizations are presented to illustrate 
the properties of the collected data.
% Collected data are visualized to
% understand their properties which can further 

\textbf{Chapter~\ref{ch:chapter4}} presents a deep convolutional
neural network which has been applied in our deep learning.
Both architecture and settings
are covered in details.
The stopping criteria are shown with the training results.
At the end of the chapter, the evaluation results for learned models
are presented.

\textbf{Chapter~\ref{ch:chapter5}} 
shows the methods
used to integrate learned models
into the reference software of 3D-HEVC\@.
The problem encountered during the integration is described,
and the solution for the problem is presented.
Simulation results comparing with the original \(HTM16.2\) are given in this
chapter.

\textbf{Chapter~\ref{ch:chapter6}} concludes the thesis.

    % !TEX root = ../main.tex
\chapter{Background}\label{ch:chapter2} % For referencing the chapter elsewhere, use \ref{Chapter1}
%
%%----------------------------------------------------------------------------------------
%
To start with, we bring up what is video coding, why it is needed, and
its challenges.
Next we discuss what is deep learning, the history of deep learning and how
it works for vision tasks.
Furthermore, we introduce how we plan to apply deep learning to optimize video
coding tasks and why it should work.
In the end, a survey of related works on the topic of video coding is given.

%3D Video applications are attracting more interests
%%----------------------------------------------------------------------------------------
%
\section{Video Coding}\label{sec:video-coding}
Video playback is the most straightforward way for human to perceive dynamic
scenes that exist across a time series.
More than half of the neurons in human brain are born to process the visual
information which is supplied by human eyes.
It becomes effortless for human to understand things presented by
the video playback instead of a long paragraph of words.
Videos are made up of consecutive sets of image frames, which in turn
are made up of pixel matrices.
Visual information of a cosmic scale is first stored by various methods
then delivered during a period of video playback.

In 1950s, video tapes were employed to store the videos.
Video tape is able to serve for about eight to twelve years
before the video quality starts to degrade.
In 1970s, laser disc appeared in the US market as an alternative of video tapes.
Start from laser disc, the video storage started its new era in digital world.
In 1990s, DVDs were released after laser disc.
Data is stored in spiralling tracks on the disc.
A laser beam can be utilized to read the data.
In addition, hard drives, flash drives and SD cards were also starting to
become popular in the late 90s.
Nowadays, the cloud storage is very common in daily lives.
It is capable of storing data on the servers which are
accessible from any devices via internet connections.

Although so many formats are available for video storage, they share a common
feature: the more storage you use, the more cost it will be.
Let's take the cloud storage as an example.
Google cloud is one of the most popular cloud services in our daily lives.
It provides cloud storage with a price
of \$0.026 per GB/month~\parencite{RN202}
(this price is observed on 21 Nov 2017, it may change in the future).
If a 4K video with a resolution of 4096*2160,
at 120 frames per second,
8 bits for each of the RGB component, needs to be stored without
any compression in Google cloud,
we need to pay a monthly fee:
\((4096*2160*120*60*90*3*0.026)/(1024*1024*1024) \approx 416.47\) \$.
Without doubt, this figure is relatively not acceptable for just
storing the video.
High compression is needed to store the videos in a practical way.

From the other perspective, let us take the bandwidth into consideration.
To deliver the uncompressed 4K video which has been mentioned in
the previous paragraph, we need a bandwidth of:
\((4096*2160*120*3)/(1024*1024*1024) \approx 2.97\) Gigabytes per second.
The maximum bandwidth of Wireless 802.11ac, which is one of the common
internet access technologies, is 1.3 Gigabytes per second~\parencite{RN203}.
Apparently, the wireless connection is not able to deliver such kind of
4K videos.
High compression is desired to deliver the video through the internet.

Despite the fact that raw videos usually contain a large amount of data,
a lot of redundancies exist.
For every video sequence, two types of redundancies are ubiquitous: Spacial
Redundancy and Temporal Redundancy.
Video coding technologies are taking advantages of those redundancies to
achieve the efficient compression for video data.
Many of the useful video coding technologies have been adopted by the
international video coding standards, such as MPEG-4, H.264, H.265, etc.

Figure~\ref{fig:video-std-brief-history} shows the brief history of the
video coding standards.
\begin{figure}
    \centering
    \includegraphics[width=\textwidth,height=\textheight,keepaspectratio]{Figures/video-std-brief-history.pdf}
%        \decoRule
    \caption[The brief history of the video coding standards]
    {The brief history of the video coding standards}
    \label{fig:video-std-brief-history}
\end{figure}
In 1980s, the COST211 video codec, built on top of Differential
Pulse Code Modulation (DPCM), was standardized under H.120 standard by CCITT
(now known as ITU-T).
In late 1989, the H.261 was completed and its success marked a milestone for
video coding at low bit rate with fairly good quality~\parencite{RN181}.
The Motion Picture Experts Group (MPEG) kicked off the exploration of video
storage, such as CD-ROMs.
Their objective was to achieve a competitive performance with cassette
recorders in terms of compression of videos which have rich motions.
The framework of H.261 had been used to start the codec design of MPEG-1.
MPEG-2 was one generation after the MPEG-1.
It featured higher capabilities when handling videos with
high bit rates and high resolutions.
In MPEG-2, the encoder is allowed to make its own decision on the
the number of bi-directionally predicted pictures according to a
suitable coding delay.
ITU-T found this technique applicable to telecommunication applications, as
a result MPEG-2 has been adopted as H.262 for telecommunications.
Right after the MPEG-2 standard, MPEG-3 was designed mainly for coding of
high definition videos.
However, MPEG-3 was discarded due to the versatility of MPEG-2, which
can be used to encode videos of any resolutions.
In the late 1998, MPEG-4 was introduced as a way of defining compression of
both audio and visual digital data.
Later on MPEG-4 was divided into several parts during its continuously evolving.
Among its sub-parts, MPEG-4 part 10 (a.k.a. Advanced Video Coding) is mainly
for the video compression.
With the rising popularity of the high definition videos, the new standard
termed High Efficiency Video Coding (HEVC) for compressing videos in a more
efficient way comparing with previous standards, such as H.264/AVC, has
emerged under the efforts from the Joint Collaborative Team on Video
Coding (JCT-VC).
In the meanwhile, five extensions of the HEVC standard, comprising
Format Range Extension (RExt), Scalability Extension (SHVC),
Multi-view Extension (MV-HEVC), 3D Extension (3D-HEVC),
Screen Content Coding Extension (SCC),  have been finalized
from 2014 to 2016 to fulfill extra requirements in various video coding
scenarios.

In this work, we focus on the depth map coding in 3D-HEVC\@.
The 35 angular modes and depth modeling modes have been embraced in the
depth map coding tools in 3D-HEVC\@.
The DMM1 mode introduces an huge increase for the encoding time of 3D videos.
Acceleration of the depth map coding is needed.

\section{Deep Learning}\label{sec:deep-learning}
Deep learning is an approach of representation learning
(a.k.a. feature learning), which is essentially a method to
learn from data.
Numerous layers of computational units together with appropriate activating
mechanism comprise the basic architecture for deep learning.
Multitudinous data sets are needed for those computational architectures
to learn data abstractions
for tasks such as image classification, speech recognition,
object detection, etc.
Each layer learns a level of abstraction from the data sets using
back-propagation algorithm~\parencite{RN96}.
Making use of those learned abstractions, the computational architectures are
able to solve complex problems which are typically non-linear and normally hard
to solve by using specific rules that are designed in advance.

Deep learning has been attracting wide attention from all over the world
in recent years, not only because of the great achievements it has
made in various application scenarios, but also due to the promise of an
intelligent future it gives.
Such a learning methodology makes people believe it is possible
for the formation of wise machines
that they have long dreamed to possess.
The growing data accessibility provides rich examples for deep computational
architectures to adjust their internal weights and bias until their
predictions have low error rate.
On the other hand, the computational devices are relatively
affordable than in the previous years by the society, with the help of which,
accelerations of learning processes has been achieved, hence a bunch of
time consuming deep learning architectures can be tried within acceptable
periods.

In the ILSVRC-2012 competition~\parencite{RN205}, AlexNet~\parencite{RN65}
received the championship with the 15.3\% top-5 error rate, compared to
26.2\% achieved by the runner-up.
Such a large margin of error rate claimed a breakthrough in
object recognition history.
It kicked off a blistering pace of trying out deep learning by both academia
and industry, which in turn led to an increase of the convolutional
neural networks' submissions to ILSVRC-2013, in which ZF Net~\parencite{RN66}
was the winner.
It fine-turned the architecture of AlexNet based on the
gorgeous visualizations of trained models.
Both AlexNet and ZF Net are of the same structure which is built up
by simply stacking computational layers while GoogLeNet~\parencite{RN60}
is composed of Inception
modules.
This new architecture was the most successful candidate in ILSVRC-2014.
It has not only set the new height of object recognition but also started to
optimize the computational resources of the network by design.
It consists of 22 layers, which was deeper than all the previous
networks in ILSVRC\@.
However, it is still not deep enough.
In ILSVRC-2015, Residual Neural Network (ResNet)~\parencite{RN67} with
152 layers won the championships in all the five main tracks.
ResNet introduced a brand new notion into the neural network architecture
named identity mapping.
The shortcut connection in the identity mapping prevents the degradation of
training accuracy when the network goes deeper.
Besides, the converging speed of ResNet is faster than the network built up
with Inception modules when both are of the similar size.

Despite the fact that neural networks built up from Inception modules
converge slower than those built up from ResNet modules, it is still
worth it for a brief review of the valuable insights residing in
the Inception networks.
A typical incarnation of the first generation of Inception networks is named
GoogLeNet~\parencite{RN60}.
It was intricately carved with a responsibility to win computer vision
tasks in ILSVRC-2014, on which it performed better than all the other
deep neural network architectures.
There exist philosophical reflections which are intend to serve as guidelines
for the construction of Inception networks.
Two major downsides of a enlarged neural network have been discussed
in~\parencite{RN60}.
One is the higher chances of overfitting while the other is
the strikingly increased requirements of computational resources with the
enlarged network size.
For handling those drawbacks, based on the new ideas which were introduced
in~\parencite{RN207} about how to construct the reasonable architecture of
neural networks, new experiments orienting sparse network structure have
been tried out.
One year later after GoogleNet hold the championship of ILSVRC-2014,
a method named Batch Normalization~\parencite{RN61} has been
proposed by Google researchers to accelerate and ease the
training of deep neural networks.
The core idea behind Batch Normalization is to normalize
the inputs to each layer for every batch of training data.
More importantly, based on the observation that the normalization process
essentially is matrix multiplications followed by adding biases, the Batch
Normalization is implemented as additional layers which makes it part of
the network architecture.
This fairly novel method started a new chapter for the training of deep
neural networks.
With the adoption of Batch Normalization, higher learning rates no longer
impede the convergence of the deep networks, oppositely faster
training speed is brought to scene which can achieve a
better accuracy of prediction with considerably less time.
Additionally, in some cases, it can even replace the Dropout~\parencite{RN70}
which is an effective method to prevent overfitting.
The incorporation of Batch Normalization into the first generation of
Inception network architecture led to the formation of Inception-v2, which
improved the best accuracy on ImageNet classification with less training steps.
%more advanced accuracy on ILSVRC 2012
%classification challenge validation set.
In the same year, Inception-v3~\parencite{RN62} joined the party, the objective of which was
to effectively leverage the power of additional computation by factorizing
to smaller size convolutions and regularizing the classifier layer with
the estimation of minor effect of label-dropout in the training process.
The network architectures were scaled up in Inception-v3, which consequently
imposed higher requirements of available computational resources.
With the ResNet~\parencite{RN67} stealing the show in ILSVRC-2015,
the influence of the
identity connections in residual units on the learning process
has been investigated in~\parencite{RN63}.
The filter concatenation stage of in Inception-v3 is replaced using identify
connections which led to the layout of a new model named Inception-ResNet-v1.
A more advanced version which was named Inception-ResNet-v2 has a larger
network size than the first version.
Besides the mixed architectures of Inception-ResNet, a pure Inception
incarnation named Inception-v4 was also presented with comparison to
Inception-ResNet-v2.
Both Inception-v4 and Inception-ResNet-v2 have significant gain of performance
mainly benefiting from the enlarged size of network.

\section{Related Work}\label{sec:related-work}
In this section, the prior arts working on optimizations for video
coding are reviewed.

Before the occurrence of Depth Modeling Mode (DMM) and
View Synthesis Optimization (VSO) in~\parencite{RN208}, a lot of literature
on depth map coding which have been published are mainly
focusing on improving the effectiveness of depth map coding.
%Although in this thesis the researching focus is 3D-HEVC oriented, it is
%still helpful to know the related works in HEVC\@.
%We start with a review of fast intra coding for HEVC,
%after that fast depth coding for 3D-HEVC\@.
%\subsection{Fast Intra Coding in HEVC}\label{subsec:fast-HEVC}
%
%\subsection{Fast depth Coding in 3D-HEVC}\label{subsec:fast-3D-HEVC}
Based on the observation that the depth map is characterized by vast
smooth regions separated by sharp edges, an algorithm to effectively
encode homogeneous regions has been proposed in~\parencite{RN120}.
It improves coding performance for depth maps by copying pixel values for
homogeneous blocks from values of neighboring reference pixels.
In~\parencite{RN123}, Depth Lookup Table (DLT) has been proposed for
encoding the depth maps in 3D-HEVC standard.
It offers the benefits of 1.3\% bit-rate reduction.
To further improve the coding performance for depth map, more
dedicated tools for depth map coding are needed.
Depth Modeling Modes (DMM) and View Synthesis Optimization (VSO) are proposed
in~\parencite{RN208}.
VSO provides 17\% bit rate reduction in average while DMM provides 6\% savings
on bit rate.
Although the introduction of DMM and VSO have brought the effectiveness of
depth map coding into a new level, the computational complexity has increased
a lot due to the complex nature of VSO\@.
Consequently the time cost of depth map encoding becomes fairly expensive.

The computational complexity of depth map coding
raised the question of whether it is possible to reduce the computational
complexity for saving encoding time.
In~\parencite{RN76}, a fast wedgelet searching scheme achieves significant
reduction for computational complexity with minor BD-rate increase.
It takes advantage of the result from
Sum of Absolute Transform Difference (SATD) to reduce the wedgelet searching
candidates.
Rough RD cost from Rough Mode Decision is used as mode selection threshold
in~\parencite{RN90} to speed up the bi-partition modes decision.
A two-step fast searching approach for wedgelet partition
appears in~\parencite{RN126}.
It features a coarse search in conjunction with a further refinement step.
Another fast approach for wedgelet searching~\parencite{RN79}
is to make use of the Most Probable Mode (MPM) to reduce wedgelet
searching candidates.
Since intra angular modes will lead to ringing artifacts
when utilized for depth map coding, the idea of skipping intra
angular prediction by making use of edge detector is shown in~\parencite{RN89}.
Bayesian classifier is used in~\parencite{RN102} to alleviate the computational
complexity of intra mode decision in 3D-HEVC\@.
The optimal mode of the parent prediction unit (PU)
in the hierarchical quad-tree coding structure has been utilized to select
the mode for child prediction unit (PU) in~\parencite{RN131}, and early
decision for segment-wise DC coding is used together to achieve faster
depth intra coding.
Edge classification in Hadamard transform domain is used in~\parencite{RN86}
to skip the DMM decision process conditionally.
The minimum RD cost of the candidates in the full-RD searching list is taken
as a threshold to bypass DMM decision based on the comparisons with the
header rates in~\parencite{RN93}.
Most probable region for DMM1 mode decision is identified with the help
of sharp edges in~\parencite{RN209}, and DMM3 is skipped when depth
prediction unit (PU) does not match with co-located texture counterpart.
Variance is utilized in~\parencite{RN210} to estimate the most promising
sub-region for DMM1.
Corner point is used for fast quad-tree decision of depth intra coding
in~\parencite{RN211}.
Variance distribution is studied in~\parencite{RN111}, based on which the
method termed Squared Euclidean distance of variances (SEDV) is
proposed to substitute the long-standing View Synthesis Optimization (VSO)
process.
Besides, a new scheme termed probability-based early depth intra mode
decision (PBED) is employed to skip modes and the RD cost in
Rough Mode Decision (RMD) is used to terminate
segment-wise depth coding (SDC)~\parencite{RN123}
as early as possible.
The correlation between depth maps and texture views are explored
in~\parencite{RN94} to alleviate the complexity of the
compression for depth map.
In~\parencite{RN212}, comparing RD cost with pre-calculated threshold for fast
intra mode decision together with early decision for the CU depth are used
to accelerate the encoding process.
Making use of RD cost results of the angular modes, only the most promising
DMMs are evaluated in~\parencite{RN87} and, moreover, an innovative
method using golden ratio to further improve the
depth map coding is proposed.
The characteristics of depth map are studied in~\parencite{RN91},
as a result only four conventional intra modes are used for
depth map intra coding and only six directions are used in DMM1 searching.
Block edge along with the border gradient are used together
in~\parencite{RN114} to accelerate the depth map coding.
Information of neighbouring blocks and threshold which is derived from
lots of experiments are used in~\parencite{RN85} for improving depth
map coding.

Despite the aforementioned works which are using heuristic approaches,
machine learning approaches are also applied to optimize the
video coding process.
In~\parencite{RN74}, the decision for the depth of the coding unit
in High Efficiency Video Coding (HEVC) is modeled as a classification
problem which is solved by machine learning approach.
A shallow convolutional neural network (CNN) is
used in~\parencite{RN78} to determine coding unit depth in
High Efficiency Video Coding (HEVC) while
in~\parencite{DBLP:journals-corr-abs-1710-01218}, a deeper convolutional
neural network together with long- and short-term memory (LSTM) network
are employed to address the same issue.
In additional to the works which are targeting the
coding unit depth decision using machine learning approaches, it is found
in~\parencite{RN73} that deep learning is used for the intra mode
selection in Screen Content Coding (SCC)
Extension of High Efficiency Video Coding (HEVC).

The researching focus in this thesis is the same as~\parencite{RN73}.
However, there exist three important differences.
Firstly, the network in this thesis comprising 32 layers is
much deeper than the 4-layer network in~\parencite{RN73}.
Secondly, unlike the server-client setup in~\parencite{RN73}, we have
managed to integrate the learned models into the codebase of \(HTM16.2\).
Executable binary can be obtained which is totally self-contained in the
sense that they are not relying on the remote server to do the prediction,
just the binary itself is capable of doing prediction
for mode selection in depth maps.
Thirdly, in this work the deep learning is for depth map coding in
Three Dimension Extension of High Efficiency Video Coding (3D-HEVC)
while in~\parencite{RN73} the learning is for Screen Content Coding (SCC)
Extension of High Efficiency Video Coding (HEVC).
%Since the angular modes are designed for preserving texture
%patterns, it does not work well on keeping the fidelity of depth maps.
%Simply using angular mode to encode depth maps,
%inging artifacts~\parencite{RN44} occur in the sharp edges on depth maps
%which tend to cause distortions on synthesized views.
%Depth maps are not directly presented to audience while synthesized views are.
%Apparently attention is needed for improvements.

% ====== can be used for literature review =====
%AlexNet contains five convolutional layers and three fully-connected
%layers.
%The Rectified Linear Units (ReLU)~\parencite{RN206}, Local Response
%Normalization and Overlapping Pooling were adopted.
%The methodology of multiple GPU training was used to make the learning fast.
%Data Augmentation and Dropout were chosen to overcome the problem of
%Overfitting.
%Stochastic gradient descent was adopted.
% ====== can be used for literature review =====




%Welcome to this \LaTeX{} Thesis Template, a beautiful and easy to use template for writing a thesis using the \LaTeX{} typesetting system.
%
%If you are writing a thesis (or will be in the future) and its subject is technical or mathematical (though it doesn't have to be), then creating it in \LaTeX{} is highly recommended as a way to make sure you can just get down to the essential writing without having to worry over formatting or wasting time arguing with your word processor.
%
%\LaTeX{} is easily able to~\parencite{RN93} professionally typeset documents that run to hundreds or thousands of pages long. With simple mark-up commands, it automatically sets out the table of contents, margins, page headers and footers and keeps the formatting consistent and beautiful. One of its main strengths is the way it can easily typeset mathematics, even \emph{heavy} mathematics. Even if those equations are the most horribly twisted and most difficult mathematical problems that can only be solved on a super-computer, you can at least count on \LaTeX{} to make them look stunning.
%
%%----------------------------------------------------------------------------------------
%
%\section{Welcome and Thanku}\label{sec:welome}
%Welcome to this \LaTeX{} Thesis Template, a beautiful and easy to use template for writing a thesis using the \LaTeX{} typesetting system.
%
%If you are writing a thesis (or will be in the future) and its subject is technical or mathematical (though it doesn't have to be), then creating it in \LaTeX{} is highly recommended as a way to make sure you can just get down to the essential writing without having to worry over formatting or wasting time arguing with your word processor.
%
%\LaTeX{} is easily able to professionally typeset documents that run to hundreds or thousands of pages long. With simple mark-up commands, it automatically sets out the table of contents, margins, page headers and footers and keeps the formatting consistent and beautiful. One of its main strengths is the way it can easily typeset mathematics, even \emph{heavy} mathematics. Even if those equations are the most horribly twisted and most difficult mathematical problems that can only be solved on a super-computer, you can at least count on \LaTeX{} to make them look stunning.
%
%%----------------------------------------------------------------------------------------
%
%\section{Welcome and ThYou}\label{sec:weome}
%Welcome to this \LaTeX{} Thesis Template~\parencite{Reference1}, a beautiful and easy to use template for writing a thesis using the \LaTeX{} typesetting system.
%
%If you are writing a thesis (or will be in the future) and its subject is technical or mathematical (though it doesn't have to be), then creating it in \LaTeX{} is highly recommended as a way to make sure you can just get down to the essential writing without having to worry over formatting or wasting time arguing with your word processor.
%
%\LaTeX{} is easily able to professionally typeset documents that run to hundreds or thousands of pages long. With simple mark-up commands, it automatically sets out the table of contents, margins, page headers and footers and keeps the formatting consistent and beautiful. One of its main strengths is the way it can easily typeset mathematics, even \emph{heavy} mathematics. Even if those equations are the most horribly twisted and most difficult mathematical problems that can only be solved on a super-computer, you can at least count on \LaTeX{} to make them look stunning.
%
%%----------------------------------------------------------------------------------------
%
%\section{Welcome and Thau}\label{sec:welcoe}
%Welcome to this \LaTeX{} Thesis Template, a beautiful and easy to use template for writing a thesis using the \LaTeX{} typesetting system.
%
%If you are
%\begin{table}
%
%    \label{tab:treatments}
%    \centering
%%    \begin{tabular}{l l l}
%%        \toprule
%%        \tabhead{Groups} & \tabhead{Treatment X} & \tabhead{Treatment Y} \\
%%        \midrule
%%        1 & 0.2 & 0.8\\
%%        2 & 0.17 & 0.7\\
%%        3 & 0.24 & 0.75\\
%%        4 & 0.68 & 0.3\\
%%        \bottomrule\\
%%    \end{tabular}
%    \begin{tabular}{c r @{.} l}
%        Pi expression       &
%        \multicolumn{2}{c}{Value} \\
%        \hline
%        $\pi$               & 3&1416  \\
%        $\pi^{\pi}$         & 36&46   \\
%        $(\pi^{\pi})^{\pi}$ & 80662&7 \\
%    \end{tabular}
%    \caption{The effects of treatments X and Y on the four groups studied.}
%\end{table}
%writing a thesis (or will be in the future) and its subject is technical or mathematical (though it doesn't have to be), then creating it in \LaTeX{} is highly recommended as a way to make sure you can just get down to the essential writing without having to worry over formatting or wasting time arguing with your word processor.
%
%\LaTeX{} is easily able to professionally typeset documents that run to hundreds or thousands of pages long. With simple mark-up commands, it automatically sets out the table of contents, margins, page headers and footers and keeps the formatting consistent and beautiful. One of its main strengths is the way it can easily typeset mathematics, even \emph{heavy} mathematics. Even if those equations are the most horribly twisted and most difficult mathematical problems that can only be solved on a super-computer, you can at least count on \LaTeX{} to make them look stunning.
%
%%----------------------------------------------------------------------------------------
%
%\section{Welcome and Tnk You}\label{sec:wlcome}
%Welcome to this \LaTeX{} Thesis Template, a beautiful and easy to use template for writing a thesis using the \LaTeX{} typesetting system.
%
%If you are writing a thesis.
%
%%\begin{verbatim}
%\begin{figure}
%    \centering
%    \includegraphics{Figures/Electron}
%    %    \decoRule
%    \caption[An Electron]{An electron (artist's impression).}
%    \label{fig:Electron}
%\end{figure}
%%\end{verbatim}
%(or will be in the future) and its subject is technical or mathematical (though it doesn't have to be), then creating it in \LaTeX{} is highly recommended as a way to make sure you can just get down to the essential writing without having to worry over formatting or wasting time arguing with your word processor.
%
%\LaTeX{} is easily able to professionally typeset documents that run to hundreds or thousands of pages long. With simple mark-up commands, it automatically sets out the table of contents, margins, page headers and footers and keeps the formatting consistent and beautiful. One of its main strengths is the way it can easily typeset mathematics, even \emph{heavy} mathematics. Even if those equations are the most horribly twisted and most difficult mathematical problems that can only be solved on a super-computer, you can at least count on \LaTeX{} to make them look stunning.
%
%%----------------------------------------------------------------------------------------
%    \chapter{Conclusions}
    ...
%    \appendix
%    \chapter{A Long Proof}
    ...
    \printbibliography[heading=bibintoc]
\end{document}