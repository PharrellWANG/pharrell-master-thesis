% Stanford University PhD thesis style -- modifications to the report style
% This is unofficial so you should always double check against the
% Registrar's office rules
% See http://library.stanford.edu/research/bibliography-management/latex-and-bibtex
% 
% Example of use below
% See the suthesis-2e.sty file for documentation
%
% !TEX root = relative/or/absolute/path/to/root/file.tex
\documentclass{report}
\usepackage{suthesis-2e}
\usepackage{graphicx}
\usepackage{verbatim} % for block comment
\usepackage{color}   % May be necessary if you want to color links
\usepackage{hyperref}
\usepackage{textcomp}
\usepackage{xcolor}
\usepackage[linesnumbered,ruled,vlined]{algorithm2e}
\usepackage{dblfloatfix}
\usepackage[backend=bibtex,style=ieee,natbib=true]{biblatex} % Use the bibtex backend with the authoryear citation style (which resembles APA)
\usepackage{listings}\definecolor{listinggray}{gray}{0.9}
\newcommand\mycommfont[1]{\footnotesize\ttfamily\textcolor{blue}{#1}}
\SetCommentSty{mycommfont}
\definecolor{lbcolor}{rgb}{0.9,0.9,0.9}
\lstset{
backgroundcolor=\color{lbcolor},
tabsize=4,
%   rulecolor=,
language=[GNU]C++,
basicstyle=\scriptsize,
upquote=true,
aboveskip={1.5\baselineskip},
columns=fixed,
showstringspaces=false,
extendedchars=false,
breaklines=true,
prebreak = \raisebox{0ex}[0ex][0ex]{\ensuremath{\hookleftarrow}},
frame=single,
numbers=left,
showtabs=false,
showspaces=false,
showstringspaces=false,
identifierstyle=\ttfamily,
keywordstyle=\color[rgb]{0,0,1},
commentstyle=\color[rgb]{0.026,0.112,0.095},
stringstyle=\color[rgb]{0.627,0.126,0.941},
numberstyle=\color[rgb]{0.205, 0.142, 0.73},
%        \lstdefinestyle{C++}{language=C++,style=numbers}’.
}
\lstset{
backgroundcolor=\color{lbcolor},
tabsize=4,
language=C++,
captionpos=b,
tabsize=3,
frame=lines,
numbers=left,
numberstyle=\tiny,
numbersep=5pt,
breaklines=true,
showstringspaces=false,
basicstyle=\footnotesize,
%  identifierstyle=\color{magenta},
keywordstyle=\color[rgb]{0,0,1},
commentstyle=\color{Darkgreen},
stringstyle=\color{red}
}

%\usepackage{xcolor}
%\lstset { %
%    numbers=left,
%    numberstyle=\tiny,
%    stepnumber=1,
%    numbersep=5pt
%    language=C++,
%%    backgroundcolor=\color{black!5}, % set backgroundcolor
%%    basicstyle=\footnotesize,% basic font setting
%}

\addbibresource{../src/mybib.bib} % The filename of the bibliography
\hypersetup{
colorlinks=false, %set true if you want colored links
linktoc=all,     %set to all if you want both sections and subsections linked
linkcolor=black,  %choose some color if you want links to stand out
}
\dept{Electronic and Information Engineering}

\begin{document}
    \title{Fast Depth Coding in 3D-HEVC
    Using Deep Learning}
    \author{Zhen-xiang WANG}
    \principaladviser{Yui-Lam Chan}
    \beforepreface
    \prefacesection{Abstract}
    The 3D Extension of the High Efficiency Video Coding standard (3D-HEVC),
    which has been finalized by the Joint Collaborative Team on Video Coding
    (JCT-VC) in February 2015, is the new industry standard for 3D applications.
    The 3D-HEVC provides plenty of advanced coding tools specifically
    for addressing the coding of auto-stereoscopic videos which have the format
    of multiple texture views along with the depth maps which are responsible
    for synthesising intermediate views with sufficient quality for
    auto-stereoscopic display.
    The provided tools take advantage of the statistical redundancies amongst
    texture views and depth maps in the video sequences, as well as the unique
    characteristics of depth maps to significantly shrink the bit-rate
    while preserving the objective visual quality of the
    3D videos.
    However, those tools with high capability in terms of compression come
    with the high complexity of computation which has made the encoding time
    of the 3D video sequences much longer than ever by traversing a lot more
    candidates, calculating time-consuming RD Cost for each of them,
    especially in the wedgelet searching process for depth maps.
    While this full-search style method can promise to find the best
    candidate in depth intra mode decision, the time cost is expensive.

    In this dissertation we address the time cost by presenting a new
    intra mode decision method for depth maps, leveraging the deep
    convolutional neural networks to predict the wedgelet angles
    for the depth blocks.
    The predictions from the learned models are capable of
    reducing the number of wedgelet candidates by half as well as the
    angular modes in depth map coding.
    The size of the neural network has been carefully designed to balance
    the trade-off between the time cost of model prediction and the model prediction
    accuracy.
    Confusion matrix is used to monitor the training process.
    Top-K criteria is employed for the prediction.
    We have integrated the learned models into the reference software of
    3D-HEVC for the experiments.
    The compiled executable binaries are able to harness
    the power of the simultaneous computation of CPU, as well as
    the parrallel computation of GPU to accelerate the predictions.
    The simulation results show that the proposed algorithm
    provides 64.6\% time reduction in average while the
    BD performance has a tiny decrease comparing with the state-of-the-art 3D-HEVC
    standard.

    \prefacesection{Acknowledgments}
%    I would not be able to accomplish the work in this dissertation
%    if it were not for the help
%    from people.
    First and foremost, I would like to give sincere thanks to my supervisor,
    Dr.Yui-Lam Chan, for his
    extremely generous support, most insightful advices and innumerable yet
    constructive feedback.
    I learned from him to first identify a problem,
    by reading a vast amount of articles
    to know what people have achieved and what bottlenecks they have encountered.
    I learned how to read papers, how to organize them to
    become the inner comprehension.
    He guided me to use the machine learning approach to solve the
    problem that has been found in the first stage.
    Without his guidance I will not have the idea to learn the deep
    learning technology and apply it to optimize the video coding.
    His encyclopedic knowledge and charming personalities made him my mentor in
    both research and life.
    I wish to thank Dr.Sik-Ho Tsang, for our in-depth discussions from
    which I can always find useful clues to proceed to next step.
    His great expertise in video coding significantly benefits me during my
    intensive period of learning.
    Also I would like to thank my friends Alex
    and Jacky, for our
    extensive discussions about artificial intelligence
    and their applications.
    Finally thank you my parents, for the great love and constant
    encouragement which give me confidence to face and handle all the
    challenges at every moment.
    \afterpreface

    % !TEX root = ../main.tex
\chapter{Introduction}\label{ch:chapter1} % For referencing the chapter elsewhere, use \ref{Chapter1}

%----------------------------------------------------------------------------------------
Video is the medium to record, copy, playback, broadcast
and display the motion images in an electronic style~\parencite{RN190}.
Watching videos is becoming an important way for human entertainment as well
as education.
The high definition (HD) and ultra high definition (UHD) videos
are increasingly demanding nowadays.
People prefer videos with higher resolution than those with lower
resolution because HD videos provide better watching experience.
However, challenges emerged for delivering videos with high definition.
HD/UHD videos typically contain much more information in every
picture frame than videos with lower resolution.
More data needs to be squeezed into the same capacity for transmission.
For example, the uncompressed video with the dimension \(720\times480\) at 30 frames
per second requires 0.03 gigabytes per second, while the uncompressed video with
the dimension \(2880\times2048\) at 120 frames per second requires 2.12 gigabytes per
second.
Since bitrate is proportional to system bandwidth for
transmission~\parencite{RN191}, and heavily expanding the
bandwidth is usually expensive, the significantly increased bitrate
for transmitting the video data is becoming one of the
major obstacles for HD video services.

To cope with the growing need for higher compression of moving
pictures~\parencite{RN193}, Joint Collaborative Team on Video
Coding (JCT-VC)~\parencite{RN192} has finalized the High Efficiency Video
Coding which is the newest international video coding standard for
substantially ameliorate the compression performance against the previous
standards.
Comparing with H.264 Advanced Video Compression Standard~\parencite{RN194},
H.265 High Efficiency Video Coding Standard provides a reduction 
of fifty percent in terms of bitrate while maintaining the objective
video quality at the same level.

Three-dimensional (3D) video has been introduced to market via lots of ways,
including Blu-Ray disc, cable and satellite transmission, terrestrial
broadcast, and streaming or downloading from the Internet~\parencite{RN118}.
3D video provides the perception of depth information which augments
the vividness of video contents.
Currently most 3D videos in the market are using stereo display technology.
Two similar views, one for left eye, the other for right eye, are presented
at the same time with the multiplexing techniques enabling the
adjustments of video geometry information~\parencite{RN196} to provide
the 3D effect.
Figure~\ref{fig:stereo-display} illustrates the typical system structure for
transmitting videos targeting stereo display.
\begin{figure}
    \centering
    \includegraphics[
        width=\textwidth,
        height=\textheight,
        keepaspectratio
        ]{Figures/StereoDisplay.pdf}
    \caption[System Structure for transmitting videos targeting 
    stereo display]{System Structure for transmitting videos
     targeting stereo display.}\label{fig:stereo-display}
\end{figure}
It can be observed that there exists a displacement between
two views.
The green vertical left margins of the red rectangles in two views
at encoder side are different with each other.
Such a displacement is the visual disparity for 3D perception.
Stereoscopic videos~\parencite{RN153} have
achieved great profitability for movie theatres in recent years.
For example, IMAX 3D has became the most popular one that offering
the immersing multimedia experiences around the world.
Special 3D glasses are needed for watching IMAX 3D movies.
The current 3D film industry is very successful in terms of attracting
customers, however, it is not the end of the story.
Myopic people do not like to wear one more pair of glasses when
watching 3D movies.
Some people will experience discomfort after wearing 3D glasses 
for hours.
To get rid of the undesired 3D glasses,
autostereoscopic multi-view technology~\parencite{RN153} is coming to
the rescue.
The two major different characteristics between stereo display and
autostereoscopic display are
listed in Table~\ref{tab:diff-stereo-autostereo}~\parencite{RN44}.
\begin{table}[b]
    \caption{Comparing characteristics of stereoscopic display and autostereoscopic display}
    \bigskip\label{tab:diff-stereo-autostereo}
    \centering
    \begin{tabular}{c c c}
        \toprule
        Characteristic & Stereo Display & Autostereoscopic Display\\
        \midrule
        Glass-Free & No & Yes \\
        Multiple Stereo Pairs & No & Yes \\
        \bottomrule
    \end{tabular}
\end{table}
The impact of available view amount for autostereoscopic display is shown in
Table~\ref{tab:autostereo-less-views-more-views}~\parencite{RN44}.
\begin{table}
    \caption{The impact of available view amount for autostereoscopic display}
    \bigskip\label{tab:autostereo-less-views-more-views}
    \centering
    \begin{tabular}{c c c}
        \toprule
        Characteristic & Small Number of Views & Large Number of Views \\
        \midrule
        Seamless View Transition  & No & Yes \\
        High Quality of Scene Depth & No & Yes \\
        \bottomrule
    \end{tabular}
\end{table}
Comparative ease can be brought to the 3D video audience
since they do not need to wear 3D glasses for watching autostereoscopic videos.
At each different view position, scenes with minor differences are available
from multiple stereo pairs which are provided by autostereoscopic
display~\parencite{RN44}.
As a result, when audience make moves for various view positions, scenes
not viewable from the previous locations are revealed during the movements.
The autostereoscopic multi-view display demands more than two views.
With a sufficient amount of views present in autostereoscopic display, the
disparities between every two adjacent views can be small enough to offer
seamless transitions from scene to scene, such that when multiple views
meet eyes sequentially, the scenes as a whole can be gorgeous.
The visual quality of the autostereoscopic display is highly proportional to
the number of available views.
Due to limited available bandwidth, transmitting arbitrary number of views
is not practical.
Researchers have proposed a new format which only requires limited number
of views and their associated depth maps for the capability of
generating arbitrary amount of views.
The typical system structure using this new format to compress and supply 3D video
resources is shown in Figure~\ref{fig:SS-MVD}.
An enormous number of views in medium positions which are able to
guarantee the high quality of 3D display can be synthesized from
\begin{figure*}[!b]
    \centering
    \includegraphics[width=\textwidth,height=\textheight,keepaspectratio]{Figures/SystemStructureOf3DEncoder}
%        \decoRule
    \caption[System Structure for transmitting videos of Multi-view 
    Plus Depth format]{System Structure for transmitting videos of 
    Multi-view Plus Depth format.}\label{fig:SS-MVD}
\end{figure*}
decoded texture frames in combination with decoded depth maps.
%The multi-view plus depth format provides the functionality of synthesizing
%required number of views from texture views and associated depth maps.\\

To employ multi-view plus depth format for 3D video, efficient compressing
methods are needed, which leads to the 3D Extension of
High Efficiency Video Coding Standard (3D-HEVC) by the Joint Collaborative Team
on 3D Video Coding Extension Development (JCT-3V)~\parencite{RN195}.
The 3D Extension of HEVC standard provides extra coding efficiency
for encoding texture views along with the corresponding depth maps by
using new tools. 
Those new tools exploit statistical redundancies between
texture views and depth maps, and pay attention to the unique characteristics of
depth maps, such as large homogeneous
regions separated by sharp boundaries~\parencite{RN47}.

% The distance between distant views
% and nearby views from a static viewpoint,
% can be expressed in the format of depth map.
Depth map vividly conveys the distance between distant views
and nearby views.
\begin{figure*}[!t]
    \centering
    \includegraphics[width=\textwidth,height=\textheight,keepaspectratio]{Figures/wedgelet}
%        \decoRule
    \caption[Wedgelet partition illustration]
    {Example of wedgelet partition in a block of size 
    \(16\times16\) in a depth map
    from Shark video sequence.
    The tiny block highlighted by transparent blue color
    is magnified then shown in the right bottom corner
    of the depth map.
    Wedgelet partitions are straight lines
    that tries their best to fit the sharp edges in CU blocks.
    }\label{fig:wedgelet-partition}
\end{figure*}
Instead of presenting depth maps as 
intermediate views directly to audience, views in the medium
positions are generated by Depth-Image-Based Rendering (DIBR) technique.
The quality of depth maps is vital to the results produced by
DIBR process.
Corona artifacts (a.k.a.\ ringing artifacts)~\parencite{RN44}
can be discovered in synthesized
views if sharpness of edge in depth maps can not be well
preserved.
Therefore, retaining edge sharpness in depth maps is the key to avoid the
artifacts in synthesized views.
In 3D-HEVC, new intra-picture prediction tools and new residual coding methods
have been applied to adapt encoding to special properties of depth maps.
Depth Modelling Mode (DMM), which is one of the new intra-picture
prediction tools, is capable of conserving sharp edges in the shape
of straight line.
It provides a very dense set of straight partitions
which will be looped through in order to find the most suitable candidate
for each in the CU blocks in depth maps.
Figure~\ref{fig:wedgelet-partition} presents an example of wedgelet
partition in a depth map from Shark video sequence.
The small block highlighted by blue color amongst the blocks
separated by the red grid is magnified at the right-bottom position.
Straight lines are used for the partitions in wedgelet mode.
Figure~\ref{fig:contour-partition} shows a sample of the contour partition
from the same depth map shown in Figure~\ref{fig:wedgelet-partition}.
The partition pattern comprises contour line instead of
straight line.
Wedgelet partition and contour partition for depth maps
are enabled by DMM1 and DMM4 separately.
%The wedgelet partition is shown in Figure.
%The contour partition is shown in Figure.
%~\parencite{RN197}.
\begin{figure}
    \centering
    \includegraphics[width=\textwidth,height=\textheight,keepaspectratio]{Figures/contour}
%        \decoRule
    \caption[Contour partition illustration]
    {Example of contour partition in a block of size \(16\times16\) in a depth map
    from Shark video sequence.
    The tiny block highlighted by transparent blue color
    is magnified then shown in the right bottom corner
    of the depth map.
    Contour partitions are irregular lines
    that tries their best to fit the sharp edges in CU blocks.
    }\label{fig:contour-partition}
\end{figure}
%introduce a little about depth map and their usage.
%mentioning iphonex true depth camera.
%draw the picture

%----------------------------------------------------------------------------------------

\section{Motivation}\label{sec:motivation_and_contribution}
The idea of this work originates from the discovery of computational
complexity inside the process for finding the best wedgelet in DMM1.
The immense intricacy for searching the best wedgelet candidate results in
a massive increase of encoding time.
The time consumed for compressing a single depth map in 3D-HEVC encoder is
roughly a sixfold increase relevant to the encoding time of a single texture
frame wherein All-Intra configuration in \(HTM16.2\) is used.
To reduce the heavy time required in depth map coding, 
a computational model, which is a deep convolutional neural network,
has been designed and trained to predict the most probable angular mode
for each CU block.
The predicted angular mode is utilized to further predict the 
most probable wedgelet candidates.
% Thus a computational model has which has been trained
% for predicting the most probable angular mode, which in turn 
% can help with reducing the complexity in DMM1.
The learned models exhibit 91.9\% to 97.0\% top-15 precision for various
block sizes.
It has been integrated into the reference software
\(HTM16.2\) of 3D-HEVC\@.
The learned models can reduce roughly half of the wedgelet candidates.
It provides 64.6\% time reduction in average while the BD performance
has a negligible decrease comparing with the original
implementation of 3D-HEVC encoder.

\textbf{Motivation for Wedgelet Candidates Reduction:} The time
consumed by the encoder from \(HTM16.2\) for
each view can be observed from command line outputs.

Figure~\ref{fig:encoding-time-example} shows a piece of command line outputs
from the encoding process of Shark sequence.
\begin{figure}
    \centering
    \includegraphics[width=\textwidth,height=\textheight,keepaspectratio]{Figures/EncodingTimeEg}
%        \decoRule
    \caption[An example showing a piece of the command line outputs during
    the encoding process for Shark sequence]
    {An example showing a piece of the command line outputs during the
    encoding process for Shark sequence.
    }\label{fig:encoding-time-example}
\end{figure}
\begin{figure*}[!b]
    \centering
    \includegraphics[width=\textwidth,height=\textheight,keepaspectratio]{Figures/major-time-spent-in-recursive-xcompresscu}
%        \decoRule
    \caption[A screen capture of the time profiling information for Newspaper sequence]
    {A screen capture of the time profiling information for Newspaper sequence.
    }\label{fig:major-time-spent-in-recursive-comresscu}
\end{figure*}
The numbers in red blocks stands for the encoding time of certain views, while
the corresponding layer Id and Picture Order Count (POD) are in the green
blocks.
A repetitive pattern of the encoding time for each view can be observed
every six numbers vertically.
A simple calculation using six numbers within the top-most
red block, \((90+69+69)/(90+69+69+14*3) \approx 0.84\), shows that
approximately 84\% total encoding time is busy with encoding
depth maps.
Similarly, it is reported in~\parencite{RN111} that the 
encoding for depth maps
consumes nearly 86\% total 3D-HEVC encoding time.
A trial of time profiling for 3D-HEVC encoder is performed using Instruments
which is an application available on macOS\@.
After encoding the Newspaper sequence for more than one hour,
Figure~\ref{fig:major-time-spent-in-recursive-comresscu} clearly shows
97.8\% time is used to compress the CUs recursively.
The first recursive \textbf{xCompressCU} function 
(denoted as \textbf{XC1} thereafter) is
for CUs of size \(64\times64\), 
the second one
(denoted as \textbf{XC2} thereafter) is targeting 
CUs of size \(32\times32\),
the third one (denoted as \textbf{XC3} thereafter) is dedicated
to CUs of size \(16\times16\), 
and the last one 
(denoted as \textbf{XC4} thereafter) is bound
to CUs of size \(8\times8\).
It can observed from 
Figure~\ref{fig:major-time-spent-in-recursive-comresscu} 
that the most time consuming part
during the process of
compressing depth CUs is DMM1 searching.
The time percentage that DMM1 searching time has in total time occupied 
by \textbf{xCompressCU} is summarized in
Table~\ref{tab:dmm1-searching-time-percent-summary} wherein the summary
for \textbf{XC1} is omitted since DMM1 is not 
applicable to CUs of size \(64\times64\) in
\(HTM16.2\).
\begin{table}[t]
    \caption{The percentages that DMM1 searching time has in total time occupied 
    by \textbf{xCompressCU}}
    \bigskip\label{tab:dmm1-searching-time-percent-summary}
    \centering
    \begin{tabular}{c c c}
        \toprule
        size & xCompressCU & Percentage\\
        \midrule
%        Overall Display Resolution & High & Low \\
        \(32\times32\)  & \textbf{XC2} & 30.0\% \\
        \(16\times16\) & \textbf{XC3} & 25.6\% \\
        \(8\times8\) & \textbf{XC4} & 18.8\% \\
        \bottomrule
    \end{tabular}
\end{table}[t]
The major reason leading to the time consuming property of DMM1 searching is the
View Synthesis Optimization (VSO) Method for improving quality of
synthesized views~\parencite{RN124}, in which the Synthesized View Distortion
Change (SVDC) is computed.
The time percentage that VSO has in DMM1 searching are summarized in
Table~\ref{tab:vso-in-dmm1-searching-time-percent-summary}.
\begin{table}[t]
    \caption{The time percentage that VSO has in DMM1 searching}
    \bigskip\label{tab:vso-in-dmm1-searching-time-percent-summary}
    \centering
    \begin{tabular}{c c c}
        \toprule
        size & Process & Percentage\\
        \midrule
%        Overall Display Resolution & High & Low \\
        \(32\times32\)  & VSO in DMM1 searching from \textbf{XC2} & 80.1\% \\
        \(16\times16\) & VSO in DMM1 searching from \textbf{XC3} & 83.7\% \\
        \(8\times8\) & VSO in DMM1 searching from \textbf{XC4} & 78.8\% \\
        \bottomrule
    \end{tabular}
\end{table}
In \(HTM16.2\), a large number of wedgelet candidates are 
evaluated using VSO which
introduces computational complexity.
Intuitively, evaluating less wedgelet candidates can help with relieving
the heavy burden of computation bared by encoder,
thereby time reduction can be achieved.

\textbf{Motivation for Using Deep Learning:} Deep learning 
is a sub field of representation learning, which
is in turn a major subset of machine learning~\parencite{RN158}.
Machine learning~\parencite{RN198}
has been applied to many scenarios in the domain of Artificial Intelligence (AI).
Deep learning was found hard to proceed further
in the late 1980s~\parencite{RN199}.
However, starting from 2012, it kicks off its
glorious comeback.
The deep Convolutional Neural Network (CNN) has won the ImageNet
Large Scale Visual Recognition Challenge (ILSVRC)
from 2012 to 2015, with the CNN architecture going deeper
and deeper.
The great achievements have attracted attentions from 
people all over the world and
have made deep learning the most popular topic in our daily lives.
Inspired by the fact that supervised deep learning can learn multiple layers of
abstract representations in the visual recognition tasks, it should
be applicable to recognize the angular modes of the intra-picture
prediction in 3D-HEVC\@.
The final DMM1 candidates selected in depth map coding
are essentially determined by the angle pattern of depth blocks.
If we can make use of deep learning to predict the most probable angles of the
target pixel block, a large amount of angular modes and DMM1
wedgelet candidates can be naturally skipped by which the time saving can be
achieved without sacrificing much of the coding performance.

Motivated by the discussions above, we adopt deep learning approach with
deep convolutional neural network to accelerate depth map coding in
3D-HEVC\@.

\section{Contribution and Dissertation Outline}\label{sec:outline}
We accelerate depth map coding using deep learning.
The contributions of the dissertation are:
\begin{itemize}
  \item A deep convolutional neural network with 32 layers 
  comprising ResNet units~\parencite{RN67}
  has been designed and trained to recognize the
  most probable angular mode of coding units (CUs) 
  in intra-picture prediction in 3D-HEVC
  encoder.
  Each learned model has high top-15 precision which works 
  well on tasks of recognizing intra angular patterns in 3D-HEVC\@.
  \item A way of integrating the learned model into \(HTM16.2\) encoder has
  been suggested.
  By making use of Bazel~\parencite{RN200} to compile the encoder binary, the
  data level parallelism (instead of concurrency) functionality in CPU
  as well as the parallel architecture in GPU are fully utilized for
  efficient matrix computations.
  \item An algorithm for fast
  depth map coding, which is shown
  in Figure~\ref{fig:proposed-fast-depth-coding-algorithm}
  on page~\pageref{fig:proposed-fast-depth-coding-algorithm},
  has been proposed and implemented.
  The simulation results show that the proposed algorithm is capable of
  reducing 64.6\% time in wedgelet searching during 3D-HEVC encoding process
  while the BD performance only has a trivial decrease.
\end{itemize}
The first two contributions lay the foundation for the third one, which is the
main objective of this work: to accelerate the depth map encoding process in
3D-HEVC\@.

\textbf{Chapter~\ref{ch:chapter2}} supplies the background of video
coding history, video coding standards, and deep learning using artificial
neural networks.
Prior arts in video coding are surveyed in this chapter.

\textbf{Chapter~\ref{ch:chapter3}} describes the algorithm which has been
implemented to collect the data to be used in deep learning.
Data pre-processing steps are described in details along with
the reasons behind the scene. 
A rich set of visualizations are presented to illustrate 
the properties of the collected data.
% Collected data are visualized to
% understand their properties which can further 

\textbf{Chapter~\ref{ch:chapter4}} presents a deep convolutional
neural network which has been applied in our deep learning.
Both architecture and settings
are covered in details.
The stopping criteria are shown with the training results.
At the end of the chapter, the evaluation results for learned models
are presented.

\textbf{Chapter~\ref{ch:chapter5}} 
shows the methods
used to integrate learned models
into the reference software of 3D-HEVC\@.
The problem encountered during the integration is described,
and the solution for the problem is presented.
Simulation results comparing with the original \(HTM16.2\) are given in this
chapter.

\textbf{Chapter~\ref{ch:chapter6}} concludes the thesis.

    % !TEX root = ../main.tex
\chapter{Background}\label{ch:chapter2} % For referencing the chapter elsewhere, use \ref{Chapter1}
%
%%----------------------------------------------------------------------------------------
%
To start with, we bring up what is video coding, why it is needed, and
its challenges.
Next we discuss what is deep learning, the history of deep learning and how
it works for vision tasks.
Furthermore, we introduce how we plan to apply deep learning to optimize video
coding tasks and why it should work.
In the end, a survey of related works on the topic of video coding is given.

%3D Video applications are attracting more interests
%%----------------------------------------------------------------------------------------
%
\section{Video Coding}\label{sec:video-coding}
Video playback is the most straightforward way for human to perceive dynamic
scenes that exist across a time series.
More than half of the neurons in human brain are born to process the visual
information which is supplied by human eyes.
It becomes effortless for human to understand things presented by
the video playback instead of a long paragraph of words.
Videos are made up of consecutive sets of image frames, which in turn
are made up of pixel matrices.
Visual information of a cosmic scale is first stored by various methods
then delivered during a period of video playback.

In 1950s, video tapes were employed to store the videos.
Video tape is able to serve for about eight to twelve years
before the video quality starts to degrade.
In 1970s, laser disc appeared in the US market as an alternative of video tapes.
Start from laser disc, the video storage started its new era in digital world.
In 1990s, DVDs were released after laser disc.
Data is stored in spiralling tracks on the disc.
A laser beam can be utilized to read the data.
In addition, hard drives, flash drives and SD cards were also starting to
become popular in the late 90s.
Nowadays, the cloud storage is very common in daily lives.
It is capable of storing data on the servers which are
accessible from any devices via internet connections.

Although so many formats are available for video storage, they share a common
feature: the more storage you use, the more cost it will be.
Let's take the cloud storage as an example.
Google cloud is one of the most popular cloud services in our daily lives.
It provides cloud storage with a price
of \$0.026 per GB/month~\parencite{RN202}
(this price is observed on 21 Nov 2017, it may change in the future).
If a 4K video with a resolution of 4096*2160,
at 120 frames per second,
8 bits for each of the RGB component, needs to be stored without
any compression in Google cloud,
we need to pay a monthly fee:
\((4096*2160*120*60*90*3*0.026)/(1024*1024*1024) \approx 416.47\) \$.
Without doubt, this figure is relatively not acceptable for just
storing the video.
High compression is needed to store the videos in a practical way.

From the other perspective, let us take the bandwidth into consideration.
To deliver the uncompressed 4K video which has been mentioned in
the previous paragraph, we need a bandwidth of:
\((4096*2160*120*3)/(1024*1024*1024) \approx 2.97\) Gigabytes per second.
The maximum bandwidth of Wireless 802.11ac, which is one of the common
internet access technologies, is 1.3 Gigabytes per second~\parencite{RN203}.
Apparently, the wireless connection is not able to deliver such kind of
4K videos.
High compression is desired to deliver the video through the internet.

Despite the fact that raw videos usually contain a large amount of data,
a lot of redundancies exist.
For every video sequence, two types of redundancies are ubiquitous: Spacial
Redundancy and Temporal Redundancy.
Video coding technologies are taking advantages of those redundancies to
achieve the efficient compression for video data.
Many of the useful video coding technologies have been adopted by the
international video coding standards, such as MPEG-4, H.264, H.265, etc.

Figure~\ref{fig:video-std-brief-history} shows the brief history of the
video coding standards.
\begin{figure}
    \centering
    \includegraphics[width=\textwidth,height=\textheight,keepaspectratio]{Figures/video-std-brief-history.pdf}
%        \decoRule
    \caption[The brief history of the video coding standards]
    {The brief history of the video coding standards}
    \label{fig:video-std-brief-history}
\end{figure}
In 1980s, the COST211 video codec, built on top of Differential
Pulse Code Modulation (DPCM), was standardized under H.120 standard by CCITT
(now known as ITU-T).
In late 1989, the H.261 was completed and its success marked a milestone for
video coding at low bit rate with fairly good quality~\parencite{RN181}.
The Motion Picture Experts Group (MPEG) kicked off the exploration of video
storage, such as CD-ROMs.
Their objective was to achieve a competitive performance with cassette
recorders in terms of compression of videos which have rich motions.
The framework of H.261 had been used to start the codec design of MPEG-1.
MPEG-2 was one generation after the MPEG-1.
It featured higher capabilities when handling videos with
high bit rates and high resolutions.
In MPEG-2, the encoder is allowed to make its own decision on the
the number of bi-directionally predicted pictures according to a
suitable coding delay.
ITU-T found this technique applicable to telecommunication applications, as
a result MPEG-2 has been adopted as H.262 for telecommunications.
Right after the MPEG-2 standard, MPEG-3 was designed mainly for coding of
high definition videos.
However, MPEG-3 was discarded due to the versatility of MPEG-2, which
can be used to encode videos of any resolutions.
In the late 1998, MPEG-4 was introduced as a way of defining compression of
both audio and visual digital data.
Later on MPEG-4 was divided into several parts during its continuously evolving.
Among its sub-parts, MPEG-4 part 10 (a.k.a. Advanced Video Coding) is mainly
for the video compression.
With the rising popularity of the high definition videos, the new standard
termed High Efficiency Video Coding (HEVC) for compressing videos in a more
efficient way comparing with previous standards, such as H.264/AVC, has
emerged under the efforts from the Joint Collaborative Team on Video
Coding (JCT-VC).
In the meanwhile, five extensions of the HEVC standard, comprising
Format Range Extension (RExt), Scalability Extension (SHVC),
Multi-view Extension (MV-HEVC), 3D Extension (3D-HEVC),
Screen Content Coding Extension (SCC),  have been finalized
from 2014 to 2016 to fulfill extra requirements in various video coding
scenarios.

In this work, we focus on the depth map coding in 3D-HEVC\@.
The 35 angular modes and depth modeling modes have been embraced in the
depth map coding tools in 3D-HEVC\@.
The DMM1 mode introduces an huge increase for the encoding time of 3D videos.
Acceleration of the depth map coding is needed.

\section{Deep Learning}\label{sec:deep-learning}
Deep learning is an approach of representation learning
(a.k.a. feature learning), which is essentially a method to
learn from data.
Numerous layers of computational units together with appropriate activating
mechanism comprise the basic architecture for deep learning.
Multitudinous data sets are needed for those computational architectures
to learn data abstractions
for tasks such as image classification, speech recognition,
object detection, etc.
Each layer learns a level of abstraction from the data sets using
back-propagation algorithm~\parencite{RN96}.
Making use of those learned abstractions, the computational architectures are
able to solve complex problems which are typically non-linear and normally hard
to solve by using specific rules that are designed in advance.

Deep learning has been attracting wide attention from all over the world
in recent years, not only because of the great achievements it has
made in various application scenarios, but also due to the promise of an
intelligent future it gives.
Such a learning methodology makes people believe it is possible
for the formation of wise machines
that they have long dreamed to possess.
The growing data accessibility provides rich examples for deep computational
architectures to adjust their internal weights and bias until their
predictions have low error rate.
On the other hand, the computational devices are relatively
affordable than in the previous years by the society, with the help of which,
accelerations of learning processes has been achieved, hence a bunch of
time consuming deep learning architectures can be tried within acceptable
periods.

In the ILSVRC-2012 competition~\parencite{RN205}, AlexNet~\parencite{RN65}
received the championship with the 15.3\% top-5 error rate, compared to
26.2\% achieved by the runner-up.
Such a large margin of error rate claimed a breakthrough in
object recognition history.
It kicked off a blistering pace of trying out deep learning by both academia
and industry, which in turn led to an increase of the convolutional
neural networks' submissions to ILSVRC-2013, in which ZF Net~\parencite{RN66}
was the winner.
It fine-turned the architecture of AlexNet based on the
gorgeous visualizations of trained models.
Both AlexNet and ZF Net are of the same structure which is built up
by simply stacking computational layers while GoogLeNet~\parencite{RN60}
is composed of Inception
modules.
This new architecture was the most successful candidate in ILSVRC-2014.
It has not only set the new height of object recognition but also started to
optimize the computational resources of the network by design.
It consists of 22 layers, which was deeper than all the previous
networks in ILSVRC\@.
However, it is still not deep enough.
In ILSVRC-2015, Residual Neural Network (ResNet)~\parencite{RN67} with
152 layers won the championships in all the five main tracks.
ResNet introduced a brand new notion into the neural network architecture
named identity mapping.
The shortcut connection in the identity mapping prevents the degradation of
training accuracy when the network goes deeper.
Besides, the converging speed of ResNet is faster than the network built up
with Inception modules when both are of the similar size.

Despite the fact that neural networks built up from Inception modules
converge slower than those built up from ResNet modules, it is still
worth it for a brief review of the valuable insights residing in
the Inception networks.
A typical incarnation of the first generation of Inception networks is named
GoogLeNet~\parencite{RN60}.
It was intricately carved with a responsibility to win computer vision
tasks in ILSVRC-2014, on which it performed better than all the other
deep neural network architectures.
There exist philosophical reflections which are intend to serve as guidelines
for the construction of Inception networks.
Two major downsides of a enlarged neural network have been discussed
in~\parencite{RN60}.
One is the higher chances of overfitting while the other is
the strikingly increased requirements of computational resources with the
enlarged network size.
For handling those drawbacks, based on the new ideas which were introduced
in~\parencite{RN207} about how to construct the reasonable architecture of
neural networks, new experiments orienting sparse network structure have
been tried out.
One year later after GoogleNet hold the championship of ILSVRC-2014,
a method named Batch Normalization~\parencite{RN61} has been
proposed by Google researchers to accelerate and ease the
training of deep neural networks.
The core idea behind Batch Normalization is to normalize
the inputs to each layer for every batch of training data.
More importantly, based on the observation that the normalization process
essentially is matrix multiplications followed by adding biases, the Batch
Normalization is implemented as additional layers which makes it part of
the network architecture.
This fairly novel method started a new chapter for the training of deep
neural networks.
With the adoption of Batch Normalization, higher learning rates no longer
impede the convergence of the deep networks, oppositely faster
training speed is brought to scene which can achieve a
better accuracy of prediction with considerably less time.
Additionally, in some cases, it can even replace the Dropout~\parencite{RN70}
which is an effective method to prevent overfitting.
The incorporation of Batch Normalization into the first generation of
Inception network architecture led to the formation of Inception-v2, which
improved the best accuracy on ImageNet classification with less training steps.
%more advanced accuracy on ILSVRC 2012
%classification challenge validation set.
In the same year, Inception-v3~\parencite{RN62} joined the party, the objective of which was
to effectively leverage the power of additional computation by factorizing
to smaller size convolutions and regularizing the classifier layer with
the estimation of minor effect of label-dropout in the training process.
The network architectures were scaled up in Inception-v3, which consequently
imposed higher requirements of available computational resources.
With the ResNet~\parencite{RN67} stealing the show in ILSVRC-2015,
the influence of the
identity connections in residual units on the learning process
has been investigated in~\parencite{RN63}.
The filter concatenation stage of in Inception-v3 is replaced using identify
connections which led to the layout of a new model named Inception-ResNet-v1.
A more advanced version which was named Inception-ResNet-v2 has a larger
network size than the first version.
Besides the mixed architectures of Inception-ResNet, a pure Inception
incarnation named Inception-v4 was also presented with comparison to
Inception-ResNet-v2.
Both Inception-v4 and Inception-ResNet-v2 have significant gain of performance
mainly benefiting from the enlarged size of network.

\section{Related Work}\label{sec:related-work}
In this section, the prior arts working on optimizations for video
coding are reviewed.

Before the occurrence of Depth Modeling Mode (DMM) and
View Synthesis Optimization (VSO) in~\parencite{RN208}, a lot of literature
on depth map coding which have been published are mainly
focusing on improving the effectiveness of depth map coding.
%Although in this thesis the researching focus is 3D-HEVC oriented, it is
%still helpful to know the related works in HEVC\@.
%We start with a review of fast intra coding for HEVC,
%after that fast depth coding for 3D-HEVC\@.
%\subsection{Fast Intra Coding in HEVC}\label{subsec:fast-HEVC}
%
%\subsection{Fast depth Coding in 3D-HEVC}\label{subsec:fast-3D-HEVC}
Based on the observation that the depth map is characterized by vast
smooth regions separated by sharp edges, an algorithm to effectively
encode homogeneous regions has been proposed in~\parencite{RN120}.
It improves coding performance for depth maps by copying pixel values for
homogeneous blocks from values of neighboring reference pixels.
In~\parencite{RN123}, Depth Lookup Table (DLT) has been proposed for
encoding the depth maps in 3D-HEVC standard.
It offers the benefits of 1.3\% bit-rate reduction.
To further improve the coding performance for depth map, more
dedicated tools for depth map coding are needed.
Depth Modeling Modes (DMM) and View Synthesis Optimization (VSO) are proposed
in~\parencite{RN208}.
VSO provides 17\% bit rate reduction in average while DMM provides 6\% savings
on bit rate.
Although the introduction of DMM and VSO have brought the effectiveness of
depth map coding into a new level, the computational complexity has increased
a lot due to the complex nature of VSO\@.
Consequently the time cost of depth map encoding becomes fairly expensive.

The computational complexity of depth map coding
raised the question of whether it is possible to reduce the computational
complexity for saving encoding time.
In~\parencite{RN76}, a fast wedgelet searching scheme achieves significant
reduction for computational complexity with minor BD-rate increase.
It takes advantage of the result from
Sum of Absolute Transform Difference (SATD) to reduce the wedgelet searching
candidates.
Rough RD cost from Rough Mode Decision is used as mode selection threshold
in~\parencite{RN90} to speed up the bi-partition modes decision.
A two-step fast searching approach for wedgelet partition
appears in~\parencite{RN126}.
It features a coarse search in conjunction with a further refinement step.
Another fast approach for wedgelet searching~\parencite{RN79}
is to make use of the Most Probable Mode (MPM) to reduce wedgelet
searching candidates.
Since intra angular modes will lead to ringing artifacts
when utilized for depth map coding, the idea of skipping intra
angular prediction by making use of edge detector is shown in~\parencite{RN89}.
Bayesian classifier is used in~\parencite{RN102} to alleviate the computational
complexity of intra mode decision in 3D-HEVC\@.
The optimal mode of the parent prediction unit (PU)
in the hierarchical quad-tree coding structure has been utilized to select
the mode for child prediction unit (PU) in~\parencite{RN131}, and early
decision for segment-wise DC coding is used together to achieve faster
depth intra coding.
Edge classification in Hadamard transform domain is used in~\parencite{RN86}
to skip the DMM decision process conditionally.
The minimum RD cost of the candidates in the full-RD searching list is taken
as a threshold to bypass DMM decision based on the comparisons with the
header rates in~\parencite{RN93}.
Most probable region for DMM1 mode decision is identified with the help
of sharp edges in~\parencite{RN209}, and DMM3 is skipped when depth
prediction unit (PU) does not match with co-located texture counterpart.
Variance is utilized in~\parencite{RN210} to estimate the most promising
sub-region for DMM1.
Corner point is used for fast quad-tree decision of depth intra coding
in~\parencite{RN211}.
Variance distribution is studied in~\parencite{RN111}, based on which the
method termed Squared Euclidean distance of variances (SEDV) is
proposed to substitute the long-standing View Synthesis Optimization (VSO)
process.
Besides, a new scheme termed probability-based early depth intra mode
decision (PBED) is employed to skip modes and the RD cost in
Rough Mode Decision (RMD) is used to terminate
segment-wise depth coding (SDC)~\parencite{RN123}
as early as possible.
The correlation between depth maps and texture views are explored
in~\parencite{RN94} to alleviate the complexity of the
compression for depth map.
In~\parencite{RN212}, comparing RD cost with pre-calculated threshold for fast
intra mode decision together with early decision for the CU depth are used
to accelerate the encoding process.
Making use of RD cost results of the angular modes, only the most promising
DMMs are evaluated in~\parencite{RN87} and, moreover, an innovative
method using golden ratio to further improve the
depth map coding is proposed.
The characteristics of depth map are studied in~\parencite{RN91},
as a result only four conventional intra modes are used for
depth map intra coding and only six directions are used in DMM1 searching.
Block edge along with the border gradient are used together
in~\parencite{RN114} to accelerate the depth map coding.
Information of neighbouring blocks and threshold which is derived from
lots of experiments are used in~\parencite{RN85} for improving depth
map coding.

Despite the aforementioned works which are using heuristic approaches,
machine learning approaches are also applied to optimize the
video coding process.
In~\parencite{RN74}, the decision for the depth of the coding unit
in High Efficiency Video Coding (HEVC) is modeled as a classification
problem which is solved by machine learning approach.
A shallow convolutional neural network (CNN) is
used in~\parencite{RN78} to determine coding unit depth in
High Efficiency Video Coding (HEVC) while
in~\parencite{DBLP:journals-corr-abs-1710-01218}, a deeper convolutional
neural network together with long- and short-term memory (LSTM) network
are employed to address the same issue.
In additional to the works which are targeting the
coding unit depth decision using machine learning approaches, it is found
in~\parencite{RN73} that deep learning is used for the intra mode
selection in Screen Content Coding (SCC)
Extension of High Efficiency Video Coding (HEVC).

The researching focus in this thesis is the same as~\parencite{RN73}.
However, there exist three important differences.
Firstly, the network in this thesis comprising 32 layers is
much deeper than the 4-layer network in~\parencite{RN73}.
Secondly, unlike the server-client setup in~\parencite{RN73}, we have
managed to integrate the learned models into the codebase of \(HTM16.2\).
Executable binary can be obtained which is totally self-contained in the
sense that they are not relying on the remote server to do the prediction,
just the binary itself is capable of doing prediction
for mode selection in depth maps.
Thirdly, in this work the deep learning is for depth map coding in
Three Dimension Extension of High Efficiency Video Coding (3D-HEVC)
while in~\parencite{RN73} the learning is for Screen Content Coding (SCC)
Extension of High Efficiency Video Coding (HEVC).
%Since the angular modes are designed for preserving texture
%patterns, it does not work well on keeping the fidelity of depth maps.
%Simply using angular mode to encode depth maps,
%inging artifacts~\parencite{RN44} occur in the sharp edges on depth maps
%which tend to cause distortions on synthesized views.
%Depth maps are not directly presented to audience while synthesized views are.
%Apparently attention is needed for improvements.

% ====== can be used for literature review =====
%AlexNet contains five convolutional layers and three fully-connected
%layers.
%The Rectified Linear Units (ReLU)~\parencite{RN206}, Local Response
%Normalization and Overlapping Pooling were adopted.
%The methodology of multiple GPU training was used to make the learning fast.
%Data Augmentation and Dropout were chosen to overcome the problem of
%Overfitting.
%Stochastic gradient descent was adopted.
% ====== can be used for literature review =====




%Welcome to this \LaTeX{} Thesis Template, a beautiful and easy to use template for writing a thesis using the \LaTeX{} typesetting system.
%
%If you are writing a thesis (or will be in the future) and its subject is technical or mathematical (though it doesn't have to be), then creating it in \LaTeX{} is highly recommended as a way to make sure you can just get down to the essential writing without having to worry over formatting or wasting time arguing with your word processor.
%
%\LaTeX{} is easily able to~\parencite{RN93} professionally typeset documents that run to hundreds or thousands of pages long. With simple mark-up commands, it automatically sets out the table of contents, margins, page headers and footers and keeps the formatting consistent and beautiful. One of its main strengths is the way it can easily typeset mathematics, even \emph{heavy} mathematics. Even if those equations are the most horribly twisted and most difficult mathematical problems that can only be solved on a super-computer, you can at least count on \LaTeX{} to make them look stunning.
%
%%----------------------------------------------------------------------------------------
%
%\section{Welcome and Thanku}\label{sec:welome}
%Welcome to this \LaTeX{} Thesis Template, a beautiful and easy to use template for writing a thesis using the \LaTeX{} typesetting system.
%
%If you are writing a thesis (or will be in the future) and its subject is technical or mathematical (though it doesn't have to be), then creating it in \LaTeX{} is highly recommended as a way to make sure you can just get down to the essential writing without having to worry over formatting or wasting time arguing with your word processor.
%
%\LaTeX{} is easily able to professionally typeset documents that run to hundreds or thousands of pages long. With simple mark-up commands, it automatically sets out the table of contents, margins, page headers and footers and keeps the formatting consistent and beautiful. One of its main strengths is the way it can easily typeset mathematics, even \emph{heavy} mathematics. Even if those equations are the most horribly twisted and most difficult mathematical problems that can only be solved on a super-computer, you can at least count on \LaTeX{} to make them look stunning.
%
%%----------------------------------------------------------------------------------------
%
%\section{Welcome and ThYou}\label{sec:weome}
%Welcome to this \LaTeX{} Thesis Template~\parencite{Reference1}, a beautiful and easy to use template for writing a thesis using the \LaTeX{} typesetting system.
%
%If you are writing a thesis (or will be in the future) and its subject is technical or mathematical (though it doesn't have to be), then creating it in \LaTeX{} is highly recommended as a way to make sure you can just get down to the essential writing without having to worry over formatting or wasting time arguing with your word processor.
%
%\LaTeX{} is easily able to professionally typeset documents that run to hundreds or thousands of pages long. With simple mark-up commands, it automatically sets out the table of contents, margins, page headers and footers and keeps the formatting consistent and beautiful. One of its main strengths is the way it can easily typeset mathematics, even \emph{heavy} mathematics. Even if those equations are the most horribly twisted and most difficult mathematical problems that can only be solved on a super-computer, you can at least count on \LaTeX{} to make them look stunning.
%
%%----------------------------------------------------------------------------------------
%
%\section{Welcome and Thau}\label{sec:welcoe}
%Welcome to this \LaTeX{} Thesis Template, a beautiful and easy to use template for writing a thesis using the \LaTeX{} typesetting system.
%
%If you are
%\begin{table}
%
%    \label{tab:treatments}
%    \centering
%%    \begin{tabular}{l l l}
%%        \toprule
%%        \tabhead{Groups} & \tabhead{Treatment X} & \tabhead{Treatment Y} \\
%%        \midrule
%%        1 & 0.2 & 0.8\\
%%        2 & 0.17 & 0.7\\
%%        3 & 0.24 & 0.75\\
%%        4 & 0.68 & 0.3\\
%%        \bottomrule\\
%%    \end{tabular}
%    \begin{tabular}{c r @{.} l}
%        Pi expression       &
%        \multicolumn{2}{c}{Value} \\
%        \hline
%        $\pi$               & 3&1416  \\
%        $\pi^{\pi}$         & 36&46   \\
%        $(\pi^{\pi})^{\pi}$ & 80662&7 \\
%    \end{tabular}
%    \caption{The effects of treatments X and Y on the four groups studied.}
%\end{table}
%writing a thesis (or will be in the future) and its subject is technical or mathematical (though it doesn't have to be), then creating it in \LaTeX{} is highly recommended as a way to make sure you can just get down to the essential writing without having to worry over formatting or wasting time arguing with your word processor.
%
%\LaTeX{} is easily able to professionally typeset documents that run to hundreds or thousands of pages long. With simple mark-up commands, it automatically sets out the table of contents, margins, page headers and footers and keeps the formatting consistent and beautiful. One of its main strengths is the way it can easily typeset mathematics, even \emph{heavy} mathematics. Even if those equations are the most horribly twisted and most difficult mathematical problems that can only be solved on a super-computer, you can at least count on \LaTeX{} to make them look stunning.
%
%%----------------------------------------------------------------------------------------
%
%\section{Welcome and Tnk You}\label{sec:wlcome}
%Welcome to this \LaTeX{} Thesis Template, a beautiful and easy to use template for writing a thesis using the \LaTeX{} typesetting system.
%
%If you are writing a thesis.
%
%%\begin{verbatim}
%\begin{figure}
%    \centering
%    \includegraphics{Figures/Electron}
%    %    \decoRule
%    \caption[An Electron]{An electron (artist's impression).}
%    \label{fig:Electron}
%\end{figure}
%%\end{verbatim}
%(or will be in the future) and its subject is technical or mathematical (though it doesn't have to be), then creating it in \LaTeX{} is highly recommended as a way to make sure you can just get down to the essential writing without having to worry over formatting or wasting time arguing with your word processor.
%
%\LaTeX{} is easily able to professionally typeset documents that run to hundreds or thousands of pages long. With simple mark-up commands, it automatically sets out the table of contents, margins, page headers and footers and keeps the formatting consistent and beautiful. One of its main strengths is the way it can easily typeset mathematics, even \emph{heavy} mathematics. Even if those equations are the most horribly twisted and most difficult mathematical problems that can only be solved on a super-computer, you can at least count on \LaTeX{} to make them look stunning.
%
%%----------------------------------------------------------------------------------------
    % !TEX root = ../main.tex
\chapter{Prepare the Data for Deep Learning}\label{ch:chapter3} % For referencing the chapter elsewhere, use \ref{Chapter1}

The success of a deep learning procedure heavily relies
on the size of available data.
It only works when a considerably large set of data can be provided.
Moreover, since we are using supervised learning which is the most
popular form of deep learning for the time being, our datasets
must be well labeled.
A large dataset can contain enormous and different classes,
with each class has its own label.
The labels are used to adjust the inner parameters of a pre-constructed
deep model according to the pre-defined loss function during the training
process.
We need to prepare a large set of labeled data before starting the
model training.
In this chapter, we start with the data collection, in which the data source
and method for collecting data are shown in detail.
After that the necessary pre-processing for the collected data are described.
Furthermore, plenty of visualizations for the collected raw data are shown with
discussion explaining the reason for the data pre-processing.

\section{Data Collection}\label{sec:data-collection}
To collect the data for training a deep model to predict the most probable
intra angular directions for depth blocks, we need to identify two questions:
Firstly, where does the data come from?
Secondly, how to collect the data from the source?
In this section, the two questions are answered one by one.

\subsection{Source of Data}\label{subsec:source-of-data}
The data are collected from four video sequences as shown in
Table~\ref{tab:data-source}.
\begin{table}[!htbp]
    \caption{Source of data for deep learning}
    \bigskip
    \label{tab:data-source}
    \centering
    \begin{tabular}{c c c c c}
        \hline
        \# & Name of the Sequence & Resolution & Usage & Number of Frames\\
        \hline
        1 & Balloons & $1024\times768$ & train,test,validate & 300\\
        2 & Kendo & $1024\times768$ & train,test,validate & 300\\
        3 & Poznan Street & $1920\times1088$ & train,test,validate & 250\\
        4 & Undo Dancer & $1920\times1088$ & train,test,validate & 250\\
        \hline
    \end{tabular}
\end{table}
Balloons sequence and Kendo sequence are of the resolution 1024 by 768 while
Poznan Street sequence and Undo Dancer sequence are of the resolution 1920
by 1088.
Both of the former two sequences have 300 frames, all of which are used to
collect the data.
The latter two sequences both have 250 frames, and all the frames are involved
in the data collection.
The collected data for each sequence will be separated into three sets
for training, testing and validating.
The training data sets are used for the deep model to learn the best
representations by the back-propagation algorithm~\parencite{RN204},
during which the inner parameters typically weights and bias are adjusted
along the gradient as instructed by the back-propagation.
After the learned model will be obtained, the validating datasets are used to
fine turn the hyper-parameters.
With the reasonably adjusted hyper-parameters, the training process will be
performed again.
After certain loops of the train-validate circle, the testing datasets will
be used to evaluate the final learned model which by then
will not be further turned any more.
After learned model will be applied to the testing datasets, the performance
results of evaluation can indicate the generalization of the learning model.

\subsection{Algorithm for Collecting Data}\label{subsec:collecting-method}
In this subsection, the algorithm used for collecting data are presented.
We collect data by encoding four video sequences shown in
Table~\ref{tab:data-source} from the coding unit (CU) level.
Most of the hybrid video coding features from HEVC remain unchanged
in 3D-HEVC, including the sizes allowed for each type of block.
There are totally four types of block, namely coding tree unit (CTU),
coding unit (CU), prediction unit (PU) and transform unit (TU).
Table~\ref{tab:allowed-sizes-of-each-type-of-block} shows the allowed sizes
for four types of block.
\begin{table}[!htbp]
    \caption{Allowed sizes of 
    each type of block}\label{tab:allowed-sizes-of-each-type-of-block}.
    \bigskip
    \centering
    \begin{tabular}{l c c c c c}
        \toprule
        Block Type & \multicolumn{5}{c}{Allowed Sizes}\\
        \midrule
        CTU & & & $16\times16$ & $32\times32$ & $64\times64$\\
        CU  & & $8\times8$ & $16\times16$ & $32\times32$ & $64\times64$\\
        PU  & $4\times4$ & $8\times8$ & $16\times16$ & $32\times32$ & $64\times64$\\
        TU  & $4\times4$ & $8\times8$ & $16\times16$ & $32\times32$ & \\
        \bottomrule
    \end{tabular}
\end{table}
A CTU itself can be used as a single CU while in some scenarios it can
be split into multiple CUs~\parencite{RN46}.
CU sits on top the PU partition structure.
CU can be partitioned to form TUs recursively in residual coding.
The maximum coding unit (CU) size in 3D-HEVC is 64.
The quad-tree splitting syntax allows CUs of largest size to be further
split into smaller sizes.
The encoder chooses the minimum allowed CU size based on the syntax in
the sequence parameter sets (SPS).
For luma CU samples, the minimum allowed size is larger than or equal to
$8\times8$.
The encoder first needs to make the basic decision of whether to code a
block with inter-picture or intra-picture prediction at CU level.
After that the best mode of the intra-picture prediction is
obtained at PU level.
The maximum block size allowed for DMM1 is $32\times32$ while the
minimum block size allowed is $4\times4$.
Hence we need to collect data for each block size from 
$4\times4$, \(8\times8\), $16\times16$ to $32\times32$.
To collect the data, we first need to identify
to which part in the reference software
we should insert the data collecting module.
Since supervised learning is chosen,
labels for data samples are also required to be collected.
The label is the best mode that has been chosen by the HTM16.2 encoder.
The diagram in Figure~\ref{fig:data-collection-diagram} illustrates
the relationships among the core modules for data collection.
The rectangular blocks with light blue background are the modules from
HTM16.2 while others are the newly added modules for data collection.
In the module of \emph{compressCtu}, the encoder recursively invokes
anther module named \emph{xCompressCtu} which is not on the diagram to
try different kinds of CU, PU, TU and to decide the best mode for them.
Once the \emph{compressCtu} module will be finished, all the needed
data can be obtained in the \emph{encodeCtu Before SAO} module.
\begin{figure}
    \centering
    \includegraphics[width=\textwidth,height=\textheight,keepaspectratio]{Figures/thesis-data-collecting-diagram.pdf}
    \caption[Data collecting diagram]{Data collecting diagram.}
    \label{fig:data-collection-diagram}
\end{figure}
\begin{figure}
    \centering
    \includegraphics[width=\textwidth,height=\textheight,keepaspectratio]{Figures/flattern-pixels-into-single-line.pdf}
    \caption[Flattern Luma samples into one line and append best mode at the end]{Flattern Luma samples into one line and append best mode at the end.}
    \label{fig:flattern-data-into-one-dimension}
\end{figure}
During the data collecting process, only Luma samples in depth blocks
are used since we are trying to
reduce the computational complexity of DMM1.
Moreover, as shown in Figure~\ref{fig:flattern-data-into-one-dimension} 
on page~\pageref{fig:flattern-data-into-one-dimension},
the Luma samples are flattened from rectangle CU blocks into
a single row which will be subsequently written into associated CSV file.

The detailed implementation for the module of \emph{depth data collection}
is shown in Algorithm~\ref{algo:collect-data} 
on page~\pageref{algo:collect-data}.
\begin{algorithm}[!b]
    \SetKwData{pcCU}{pcCU}
    \SetKwData{Left}{left}
    \SetKwData{uiAbsPartIdx}{uiAbsPartIdx}
    \SetKwData{uiDepth}{uiDepth}
    \SetKwData{DISFlag}{DISFlag}
    \SetKwData{iPartNum}{iPartNum}
    \SetKwData{sizeOfNByN}{sizeOfNByN}
    \SetKwData{maxCUWidth}{maxCUWidth}
    \SetKwData{pcPic}{pcPic}
    \SetKwData{pcSlice}{pcSlice}
    \SetKwData{pOrg}{pOrg}
    \SetKwData{iStride}{iStride}
    \SetKwData{uiCuSize}{uiCuSize}
    \SetKwData{pOrgPel}{pOrgPel}
    \SetKwData{sizeOfSubBlk}{sizeOfSubBlk}
    \SetKwData{uiTPelY}{uiTPelY}
    \SetKwData{uiLPelX}{uiLPelX}
    \SetKwData{yStartPos}{yStartPos}
    \SetKwData{yEndPos}{yEndPos}
    \SetKwData{xStartPos}{xStartPos}
    \SetKwData{xEndPos}{xEndPos}
    \SetKwData{iDir}{iDir}
    \SetKwData{partitionMode}{partitionMode}
    \SetKwFunction{getCUSize}{getCUSize}
    \SetKwFunction{getSizeOfSubBlk}{getSizeOfSubBlk}
    \SetKwFunction{FindCompress}{FindCompress}
    \SetKwFunction{getIntraDir}{getIntraDir}
    \SetKwFunction{getPic}{getPic}
    \SetKwFunction{getSlice}{getSlice}
    \SetKwFunction{getAddr}{getAddr}
    \SetKwFunction{getStride}{getStride}
    \SetKwFunction{getYPelCU}{getYPelCU}
    \SetKwFunction{getPartitionSize}{getPartitionSize}
    \DontPrintSemicolon % Some LaTeX compilers require you to use \dontprintsemicolon instead
    \KwIn{CU data structure \pcCU,
    absolute partition index of CU \uiAbsPartIdx,
    quad-tree depth \uiDepth}
    \KwOut{Flattened luma pixel values of each block together with
    the index of its best intra mode in each row of the output csv file}
    \Begin{
    \For{each CU in depth maps}{
    \uiCuSize$\leftarrow$\getCUSize{\pcCU, \uiDepth}\;
    \pOrgPel$\leftarrow$\getYPelCU{\pcCU}\;
    \If{\DISFlag $\equiv 0$}{
    %  \partitionMode$\leftarrow$ \getPartitionSize{$Im[i,j-1]$}\;
    \partitionMode$\leftarrow$\getPartitionSize{\pcCU, \uiAbsPartIdx}\;

    \eIf{\partitionMode $\equiv$ \sizeOfNByN}{
    \iPartNum$\leftarrow 4$\;
    }{
    \iPartNum$\leftarrow 1$\;
    }

    \For{$j\leftarrow 0$ \KwTo \iPartNum}{
    $iDir[j]$ $\leftarrow$ \getIntraDir{\pcCU, \uiAbsPartIdx}\;
    }
    \eIf{\iPartNum $\equiv 1$}{
    %      \tcc{collect luma values and the best mode for a single block}
    Create a new csv file, append the value of \uiDepth at the end of the name of the new csv file\;
    \For{$y\leftarrow 0$ \KwTo \uiCuSize}{
    \For{$x\leftarrow 0$ \KwTo \uiCuSize}{
    Write $pOrgPel[x]$ into $row_m$ in csv file\;
    }
    \pOrgPel $\leftarrow$ \pOrgPel + \iStride\;
    }
    Write $iDir[0]$ into the end of $row_m$ in the csv file\;
    }{
    %      \tcc{collect luma values and the best modes for each sub parts}
    Create a new csv file, append the value of (\uiDepth + $1$) at the end of the name of the new csv file\;
    \sizeOfSubBlk$\leftarrow$\getSizeOfSubBlk{\pcCU, \uiDepth}\;
    \For{$j\leftarrow 0$ \KwTo \iPartNum}{
    \uIf{$j\equiv0$}{
    \yStartPos $\leftarrow 0$
    \& \xStartPos $\leftarrow 0$
    \& \yEndPos $\leftarrow$ \sizeOfSubBlk\
    \& \xEndPos $\leftarrow$ \sizeOfSubBlk\;
    }
    \uElseIf{$j\equiv1$}{
    \yStartPos $\leftarrow 0$
    \& \xStartPos $\leftarrow$ \sizeOfSubBlk
    \& \yEndPos $\leftarrow$ \sizeOfSubBlk
    \& \xEndPos $\leftarrow \sizeOfSubBlk \times 2$\;
    }
    \uElseIf{$j\equiv2$}{
    \yStartPos $\leftarrow$ \sizeOfSubBlk
    \& \xStartPos $\leftarrow 0$
    \& \yEndPos $\leftarrow \sizeOfSubBlk \times 2$
    \& \xEndPos $\leftarrow$ \sizeOfSubBlk\;
    }
    \uElseIf{$j\equiv3$}{
    \yStartPos $\leftarrow$ \sizeOfSubBlk
    \& \xStartPos $\leftarrow$ \sizeOfSubBlk
    \& \yEndPos $\leftarrow \sizeOfSubBlk \times 2$
    \& \xEndPos $\leftarrow \sizeOfSubBlk \times 2$\;
    }
    }
    \For{$y\leftarrow$ \yStartPos \KwTo \yEndPos}{
    \For{$x\leftarrow$ \xStartPos \KwTo \xEndPos}{
    %   \For{$y\leftarrow 0$ \KwTo \yEndPos}{
    %        \For{$x\leftarrow 0$ \KwTo \xEndPos}{
    Write $pOrgPel[x]$ into $row_m$ in the csv file\;
    }
    \pOrgPel $\leftarrow$ \pOrgPel+ \iStride\;
    }
    Write $iDir[j]$ into the end of $row_m$ in the csv file\;
    }}}}
 \caption{Collect data}\label{algo:collect-data}
\end{algorithm}
The inputs to the data collecting workflow are exactly the inputs to
the module of \emph{encodeCtu}, namely the CU data structure,
the absolute partition index and the corresponding quad-tree depth.
Firstly, the partition mode of the CU is obtained which can further indicate
the partition number to either be one or four.
After that, based on the partition number, the best modes for blocks
are stored into an four dimensional array of integer values within the range
from 0 (inclusive) to 36 (inclusive).
If the partition number is equal to one, which means the current CU has not
been split and it has its own best mode.
The luma pixel samples of this block are collected into a single row in the
CSV file.
In the end of the same row, the best mode of the CU is signaled.
If the partition number is equal to four, which means the current CU has been
split to form four smaller sub-blocks with each sub-block has its
own best mode.
The luma samples for each sub-block are collected into four different rows
in the CSV file with the last value in each row to be the best mode for each
sub-block.
HTM16.2~\parencite{RN214} is used in this work, which is the newest
version of the reference software of 3D-HEVC\@.
Considering we are focusing on the intra prediction,
All-Intra configuration is used.

\section{Data Visualization}\label{sec:data-visu}
Visualizing data is a good way for human to better understand
and memorize information. 
Complex patterns hidden in the data are easier to be discovered when they 
are aesthetically presented via the graphical format.
% We visualize the collected data to understand them
% more clearly.
The collected data are visualized with the hope of
discovering more information from them.
% They shall be sufficient to support the discussions which are
% ;Showing visualizations for four sets of 
% Due to the large quantity of the data from the visualization perspective,
% based on the observations.
Discussions based on the observations of the visualized blocks
are given, which convince us about the necessity of data 
pre-processing in Section~\ref{sec:data-preprocessing} before 
training the deep models.

\subsection{Visualized Data}\label{subsec:see-data-visu}
We have four sets of data which are from blocks of size
\(4\times4\), \(8\times8\), \(16\times16\) to \(32\times32\).
% However, only the part of the visualization for data of block 
% size \(8\\times8\) are presented in this section due to the 
% below considerations:
The intra prediction modes in 3D-HEVC include
DC mode, Planar mode, 33 angular modes, DMM1 and DMM4.
In this thesis, several indices are assigned to each mode:
0 for DC mode, 1 for Planar mode, {[2,34]} for 33 angular modes, 35
for DMM1 and 36 for DMM4.

After all the four sets of data have been visualized,
it is found that four corresponding sets of visualizations
give the same hints as discussed in
Subsection~\ref{subsec:discussion-about-data-visu}.
Besides, to avoid too much visualizations' occupation of the thesis,
only the visualizations of blocks of size $8\times8$
are shown.
There are totally 37 figures presented,
from Figure~\ref{fig:size8_mode0}, Figure~\ref{fig:size8_mode1},
\ldots to Figure~\ref{fig:size8_mode36}.

\begin{figure}[H]
    
        \vspace*{1cm} % vertical separation
    
        \begin{minipage}{0.49\textwidth}
            \includegraphics[width=\linewidth]{Figures/visu-size8x8/8-0}
            \caption[Visualizations for blocks tagged with intra DC]{Visualizations for blocks tagged with intra DC.}
            \label{fig:size8_mode0}
        \end{minipage}
        \hspace{\fill} % note: no blank line here
        \begin{minipage}{0.49\textwidth}
            \includegraphics[width=\linewidth]{Figures/visu-size8x8/8-1}
            \caption[Visualizations for blocks tagged with intra PLANAR]{Visualizations for blocks tagged with intra PLANAR.}
            \label{fig:size8_mode1}
        \end{minipage}
        
        \vspace*{1cm} % vertical separation
    
        \begin{minipage}{0.49\textwidth}
            \includegraphics[width=\linewidth]{Figures/visu-size8x8/8-2.jpeg}
            \caption[Visualizations for blocks tagged with intra mode 2]{Visualizations for blocks tagged with intra mode 2.}
            \label{fig:size8_mode2}
        \end{minipage}
        \hspace{\fill} % note: no blank line here
        \begin{minipage}{0.49\textwidth}
            \includegraphics[width=\linewidth]{Figures/visu-size8x8/8-3}
            \caption[Visualizations for blocks tagged with intra mode 3]{Visualizations for blocks tagged with intra mode 3.}
            \label{fig:size8_mode3}
        \end{minipage}
    \end{figure}
    
    \begin{figure}
    
        \vspace*{1cm} % vertical separation
        
        \begin{minipage}{0.49\textwidth}
            \includegraphics[width=\linewidth]{Figures/visu-size8x8/8-4}
            \caption[Visualizations for blocks tagged with intra mode 4]{Visualizations for blocks tagged with intra mode 4.}
            \label{fig:size8_mode4}
        \end{minipage}
        \hspace{\fill} % note: no blank line here
        \begin{minipage}{0.49\textwidth}
            \includegraphics[width=\linewidth]{Figures/visu-size8x8/8-5}
            \caption[Visualizations for blocks tagged with intra mode 5]{Visualizations for blocks tagged with intra mode 5.}
            \label{fig:size8_mode5}
        \end{minipage}
    
        \vspace*{1cm} % vertical separation
    
        \begin{minipage}{0.49\textwidth}
            \includegraphics[width=\linewidth]{Figures/visu-size8x8/8-6}
            \caption[Visualizations for blocks tagged with intra mode 6]{Visualizations for blocks tagged with intra mode 6.}
            \label{fig:size8_mode6}
        \end{minipage}
        \hspace{\fill} % note: no blank line here
        \begin{minipage}{0.49\textwidth}
            \includegraphics[width=\linewidth]{Figures/visu-size8x8/8-7}
            \caption[Visualizations for blocks tagged with intra mode 7]{Visualizations for blocks tagged with intra mode 7.}
            \label{fig:size8_mode7}
        \end{minipage}
        
        \vspace*{1cm} % vertical separation
    
        \begin{minipage}{0.49\textwidth}
            \includegraphics[width=\linewidth]{Figures/visu-size8x8/8-8}
            \caption[Visualizations for blocks tagged with intra mode 8]{Visualizations for blocks tagged with intra mode 8.}
            \label{fig:size8_mode8}
        \end{minipage}
        \hspace{\fill} % note: no blank line here
        \begin{minipage}{0.49\textwidth}
            \includegraphics[width=\linewidth]{Figures/visu-size8x8/8-9}
            \caption[Visualizations for blocks tagged with intra mode 9]{Visualizations for blocks tagged with intra mode 9.}
            \label{fig:size8_mode9}
        \end{minipage}
    % \caption{Figure caption goes here}\label{fig:see-data-visu}
    \end{figure}
    
    \begin{figure}
    
        \vspace*{1cm} % vertical separation
    
        \begin{minipage}{0.49\textwidth}
            \includegraphics[width=\linewidth]{Figures/visu-size8x8/8-10}
            \caption[Visualizations for blocks tagged with intra mode 10]{Visualizations for blocks tagged with intra mode 10.}
            \label{fig:size8_mode4}
        \end{minipage}
        \hspace{\fill} % note: no blank line here
        \begin{minipage}{0.49\textwidth}
            \includegraphics[width=\linewidth]{Figures/visu-size8x8/8-11}
            \caption[Visualizations for blocks tagged with intra mode 11]{Visualizations for blocks tagged with intra mode 11.}
            \label{fig:size8_mode11}
        \end{minipage}
        
        \vspace*{1cm} % vertical separation
        
        \begin{minipage}{0.49\textwidth}
            \includegraphics[width=\linewidth]{Figures/visu-size8x8/8-12}
            \caption[Visualizations for blocks tagged with intra mode 12]{Visualizations for blocks tagged with intra mode 12.}
            \label{fig:size8_mode12}
        \end{minipage}
        \hspace{\fill} % note: no blank line here
        \begin{minipage}{0.49\textwidth}
            \includegraphics[width=\linewidth]{Figures/visu-size8x8/8-13}
            \caption[Visualizations for blocks tagged with intra mode 13]{Visualizations for blocks tagged with intra mode 13.}
            \label{fig:size8_mode13}
        \end{minipage}
        
        \vspace*{1cm} % vertical separation
    
        \begin{minipage}{0.49\textwidth}
            \includegraphics[width=\linewidth]{Figures/visu-size8x8/8-14}
            \caption[Visualizations for blocks tagged with intra mode 14]{Visualizations for blocks tagged with intra mode 14.}
            \label{fig:size8_mode14}
        \end{minipage}
        \hspace{\fill} % note: no blank line here
        \begin{minipage}{0.49\textwidth}
            \includegraphics[width=\linewidth]{Figures/visu-size8x8/8-15}
            \caption[Visualizations for blocks tagged with intra mode 15]{Visualizations for blocks tagged with intra mode 15.}
            \label{fig:size8_mode15}
        \end{minipage}
    % \caption{Figure caption goes here}\label{fig:see-data-visu}
    \end{figure}
    
    \begin{figure}
    
        \vspace*{1cm} % vertical separation
    
        \begin{minipage}{0.49\textwidth}
            \includegraphics[width=\linewidth]{Figures/visu-size8x8/8-16}
            \caption[Visualizations for blocks tagged with intra mode 16]{Visualizations for blocks tagged with intra mode 16.}
            \label{fig:size8_mode16}
        \end{minipage}
        \hspace{\fill} % note: no blank line here
        \begin{minipage}{0.49\textwidth}
            \includegraphics[width=\linewidth]{Figures/visu-size8x8/8-17}
            \caption[Visualizations for blocks tagged with intra mode 17]{Visualizations for blocks tagged with intra mode 17.}
            \label{fig:size8_mode17}
        \end{minipage}
    
        \vspace*{1cm} % vertical separation
    
        \begin{minipage}{0.49\textwidth}
            \includegraphics[width=\linewidth]{Figures/visu-size8x8/8-18}
            \caption[Visualizations for blocks tagged with intra mode 18]{Visualizations for blocks tagged with intra mode 18.}
            \label{fig:size8_mode18}
        \end{minipage}
        \hspace{\fill} % note: no blank line here
        \begin{minipage}{0.49\textwidth}
            \includegraphics[width=\linewidth]{Figures/visu-size8x8/8-19}
            \caption[Visualizations for blocks tagged with intra mode 19]{Visualizations for blocks tagged with intra mode 19.}
            \label{fig:size8_mode19}
        \end{minipage}
        
        \vspace*{1cm} % vertical separation
    
        \begin{minipage}{0.49\textwidth}
            \includegraphics[width=\linewidth]{Figures/visu-size8x8/8-20}
            \caption[Visualizations for blocks tagged with intra mode 20]{Visualizations for blocks tagged with intra mode 20.}
            \label{fig:size8_mode20}
        \end{minipage}
        \hspace{\fill} % note: no blank line here
        \begin{minipage}{0.49\textwidth}
            \includegraphics[width=\linewidth]{Figures/visu-size8x8/8-21}
            \caption[Visualizations for blocks tagged with intra mode 21]{Visualizations for blocks tagged with intra mode 21.}
            \label{fig:size8_mode21}
        \end{minipage}
    
    % \caption{Figure caption goes here}\label{fig:see-data-visu}
    \end{figure}
    
    \begin{figure}
    
        \vspace*{1cm} % vertical separation
    
        \begin{minipage}{0.49\textwidth}
            \includegraphics[width=\linewidth]{Figures/visu-size8x8/8-22}
            \caption[Visualizations for blocks tagged with intra mode 22]{Visualizations for blocks tagged with intra mode 22.}
            \label{fig:size8_mode22}
        \end{minipage}
        \hspace{\fill} % note: no blank line here
        \begin{minipage}{0.49\textwidth}
            \includegraphics[width=\linewidth]{Figures/visu-size8x8/8-23}
            \caption[Visualizations for blocks tagged with intra mode 23]{Visualizations for blocks tagged with intra mode 23.}
            \label{fig:size8_mode23}
        \end{minipage}
    
        \vspace*{1cm} % vertical separation
    
        \begin{minipage}{0.49\textwidth}
            \includegraphics[width=\linewidth]{Figures/visu-size8x8/8-24}
            \caption[Visualizations for blocks tagged with intra mode 24]{Visualizations for blocks tagged with intra mode 24.}
            \label{fig:size8_mode24}
        \end{minipage}
        \hspace{\fill} % note: no blank line here
        \begin{minipage}{0.49\textwidth}
            \includegraphics[width=\linewidth]{Figures/visu-size8x8/8-25}
            \caption[Visualizations for blocks tagged with intra mode 25]{Visualizations for blocks tagged with intra mode 25.}
            \label{fig:size8_mode25}
        \end{minipage}
        
        \vspace*{1cm} % vertical separation
    
        \begin{minipage}{0.49\textwidth}
            \includegraphics[width=\linewidth]{Figures/visu-size8x8/8-26}
            \caption[Visualizations for blocks tagged with intra mode 26]{Visualizations for blocks tagged with intra mode 26.}
            \label{fig:size8_mode26}
        \end{minipage}
        \hspace{\fill} % note: no blank line here
        \begin{minipage}{0.49\textwidth}
            \includegraphics[width=\linewidth]{Figures/visu-size8x8/8-27}
            \caption[Visualizations for blocks tagged with intra mode 27]{Visualizations for blocks tagged with intra mode 27.}
            \label{fig:size8_mode27}
        \end{minipage}
        % \caption{Figure caption goes here}\label{fig:see-data-visu}
    \end{figure}
    
    \begin{figure}
    
        \vspace*{1cm} % vertical separation
    
        \begin{minipage}{0.49\textwidth}
            \includegraphics[width=\linewidth]{Figures/visu-size8x8/8-28}
            \caption[Visualizations for blocks tagged with intra mode 28]{Visualizations for blocks tagged with intra mode 28.}
            \label{fig:size8_mode28}
        \end{minipage}
        \hspace{\fill} % note: no blank line here
        \begin{minipage}{0.49\textwidth}
            \includegraphics[width=\linewidth]{Figures/visu-size8x8/8-29}
            \caption[Visualizations for blocks tagged with intra mode 29]{Visualizations for blocks tagged with intra mode 29.}
            \label{fig:size8_mode29}
        \end{minipage}
    
        \vspace*{1cm} % vertical separation
    
        \begin{minipage}{0.49\textwidth}
            \includegraphics[width=\linewidth]{Figures/visu-size8x8/8-30}
            \caption[Visualizations for blocks tagged with intra mode 30]{Visualizations for blocks tagged with intra mode 30.}
            \label{fig:size8_mode30}
        \end{minipage}
        \hspace{\fill} % note: no blank line here
        \begin{minipage}{0.49\textwidth}
            \includegraphics[width=\linewidth]{Figures/visu-size8x8/8-31}
            \caption[Visualizations for blocks tagged with intra mode 31]{Visualizations for blocks tagged with intra mode 31.}
            \label{fig:size8_mode31}
        \end{minipage}
        
        \vspace*{1cm} % vertical separation
    
        \begin{minipage}{0.49\textwidth}
            \includegraphics[width=\linewidth]{Figures/visu-size8x8/8-32}
            \caption[Visualizations for blocks tagged with intra mode 32]{Visualizations for blocks tagged with intra mode 32.}
            \label{fig:size8_mode32}
        \end{minipage}
        \hspace{\fill} % note: no blank line here
        \begin{minipage}{0.49\textwidth}
            \includegraphics[width=\linewidth]{Figures/visu-size8x8/8-33}
            \caption[Visualizations for blocks tagged with intra mode 33]{Visualizations for blocks tagged with intra mode 33.}
            \label{fig:size8_mode33}
        \end{minipage}
    % \caption{Figure caption goes here}\label{fig:see-data-visu}
    \end{figure}
    \begin{figure}
        \vspace*{1cm} % vertical separation
        \begin{minipage}{0.49\textwidth}
            \includegraphics[width=\linewidth]{Figures/visu-size8x8/8-34}
            \caption[Visualizations for blocks tagged with intra mode 34]{Visualizations for blocks tagged with intra mode 34.}
            \label{fig:size8_mode34}
        \end{minipage}
        \hspace{\fill} % note: no blank line here
        \begin{minipage}{0.49\textwidth}
            \includegraphics[width=\linewidth]{Figures/visu-size8x8/8-35}
            \caption[Visualizations for blocks tagged with DMM1]{Visualizations for blocks tagged with DMM1}
            \label{fig:size8_mode35}
        \end{minipage}
    
        \vspace*{1cm} % vertical separation
    
        \begin{minipage}{0.49\textwidth}
            \includegraphics[width=\linewidth]{Figures/visu-size8x8/8-36}
            \caption[Visualizations for blocks tagged with DMM4]{Visualizations for blocks tagged with DMM4}
            \label{fig:size8_mode36}
        \end{minipage}
        % \caption{Figure caption goes here}\label{fig:see-data-visu}
    \end{figure}

\subsection{Discussion}\label{subsec:discussion-about-data-visu}
From the visualizations of blocks of size $8\times8$,
it is found that within blocks of each single angular mode,
part of them have sharp edges that can 
be easily perceived by human sights while the others 
are very smooth such that their edges can not be clearly seem
unless they are enlarged by multiple times.
When blocks with such a mixed style are fed into the 
convolutional neural networks, the prediction accuracy
of the constructed computational model
is always not ideal enough to be used
inside the reference software of 3D-HEVC\@.
However, the same model learns well on other 
benchmark datasets such as 
MNIST~\parencite{XRN001} and CIFAR~\parencite{XRN002}.
There exist two explanations to this phenomenon.
One is that the mix of the extreme smoothness and
clear sharpness inside each single intra prediction mode
yields impure datasets such that the neural networks
are not able to figure out the intrinsic abstractions
layer by layer.
The other is that our network size is not deep enough
to have the capability of learning representations 
for such a mixed style, and meanwhile, the size of 
the dataset cannot satisfy larger networks
due to the limited features available in the 
training dataset.
The limited size of the collected data means that
there is no chance of trying to train the deep learning
model with a much more rich datasets currently.
Moreover, larger neural networks require more 
computational power which can be very expensive.
Combining the two considerations above, 
eliminating extreme smoothness is the way to go,
by which the blocks with vague edges are removed
from the datasets.

Special attentions need to be given to
mode DC, PLANAR, DMM1 and DMM4.
For DC mode and PLANAR mode, since most of their blocks
still have weak edges with various patterns, 
intuitively it seems not practical to require 
neural network to learn to 
distinguish them from angular modes.
For DMM1 which is specially designed for depth maps, 
it is noticed that lots of their
blocks contain sharp edges with arbitrary angles.
Hence the learned model may predict DMM1 into any
of the 33 angular modes according to the angle of
the partition line in DMM1 blocks.
And DMM4 which is another dedicated mode for depth maps,
most of their blocks feature contour partitions instead of
straight lines while still some contours cannot show 
their clear characteristics that can discriminate themselves
from angular modes with some curvilinear distortions.

In fact, it has been tried to train the deep
neural networks which work well on 
benchmark datasets by using 37 classes, 
including modes [DC, PLANAR, 2, \ldots, 34, DMM1, DMM4].
During the training process, the validations
are performed in a fixed frequency to monitor 
the performance of the learned model.
Confusion matrix~\parencite{RN216} is obtained after 
every validation process.
Figure~\ref{fig:cm-after-12-epochs},
Figure~\ref{fig:cm-after-24-epochs},
Figure~\ref{fig:cm-after-36-epochs}, and
Figure~\ref{fig:cm-after-48-epochs}
on page~\pageref{fig:cm-after-12-epochs}
show the confusion matrices
after 12, 24, 36 and 48 epochs of model training
separately.
The color thickness of block with coordinates $(i,j)$
expresses the frequency with which a block with best 
mode \emph{i} is predicted as \emph{j}.
For each vertical line in the matrix, 
the probabilities of all blocks sum up to $100\%$.
From epoch 12 to epoch 48, most of the thicknesses 
in matrices have gradually been aggregated to 
the main diagonal, but there are nevertheless
four exceptional classes that predictions for them
have never been right in all the confusion matrices.
Those four classes are exactly mode DC, PLANAR, 
DMM1 and DMM4 which are tagged with mode index 
0, 1, 35 and 36 separately.
It turns out the neural networks will misclassify 
them into angular modes instead of giving 
predictions that are identical to their label 
of ground truth.
To cure this illness, it has been decided to
remove those four modes from target classes.

\begin{figure}
    \begin{minipage}{0.49\textwidth}
        \includegraphics[width=\textwidth,height=\textheight,keepaspectratio]{Figures/confusion-matrix/ckpt-10342.eps}
        \caption[Confusion matrix obtained after 12 epochs of model training]
        {Confusion matrix obtained after 12 epochs of model training}
        \label{fig:cm-after-12-epochs}
    \end{minipage}
    \hspace{\fill} % note: no blank line here
    \begin{minipage}{0.49\textwidth}
        \includegraphics[width=\textwidth,height=\textheight,keepaspectratio]{Figures/confusion-matrix/ckpt-20622.eps}
        \caption[Confusion matrix obtained after 24 epochs of model training]
        {Confusion matrix obtained after 24 epochs of model training}
        \label{fig:cm-after-24-epochs}
    \end{minipage}

    \vspace*{1cm} % vertical separation

    \begin{minipage}{0.49\textwidth}
        \includegraphics[width=\textwidth,height=\textheight,keepaspectratio]{Figures/confusion-matrix/ckpt-30757.eps}
        \caption[Confusion matrix obtained after 36 epochs of model training]
        {Confusion matrix obtained after 36 epochs of model training}
        \label{fig:cm-after-36-epochs}
    \end{minipage}
    \hspace{\fill} % note: no blank line here
    \begin{minipage}{0.49\textwidth}
        \includegraphics[width=\textwidth,height=\textheight,keepaspectratio]{Figures/confusion-matrix/ckpt-40884.eps}
        \caption[Confusion matrix obtained after 48 epochs of model training]
        {Confusion matrix obtained after 48 epochs of model training}
        \label{fig:cm-after-48-epochs}
    \end{minipage}
    % \caption{Figure caption goes here}\label{fig:see-data-visu}
\end{figure}

Different from the object recognition problems,
when convolutional neural networks are employed to 
predict the intra patterns, some popular data pre-processing
steps such as random crop, vertical flip
are not applicable anymore.
For example, angular mode 18 in Figure~\ref{fig:size8_mode18}
on page~\pageref{fig:size8_mode18} and angular mode 34 in
Figure~\ref{fig:size8_mode34} on page~\pageref{fig:size8_mode34}
would be identical to each other in terms of the edge angle
if either is vertically flipped.

Angular mode 2 and angular mode 34 are
on the same diagonal such that they cannot 
be treat as two separate classes in our case.
For this reason, collected blocks of angular mode 34
is removed from the datasets.
Instead of directly predicting angular mode 34,
both angular mode 2 and angular mode 34 will
be deemed as equal
when the prediction result is angular mode 2.
Further comparisons between the two modes will
be preformed using conventional encoder decision.

\section{Data Pre-processing}\label{sec:data-preprocessing}
According to the discussions in 
Subsection~\ref{subsec:discussion-about-data-visu}
on page~\pageref{subsec:discussion-about-data-visu}
which are based on the observations
of the visualized data in
Subsection~\ref{subsec:see-data-visu}
on page~\pageref{subsec:see-data-visu},
several pre-processing actions need
to be taken to clean up the collected data for deep learning.
The pre-processing steps are very crucial to the success
of the learning process for the convolutional neural networks.
Without pre-processing the collected data would be inappropriate 
for achieving required model performance for the prediction
activities.
In this section, the data pre-processing details in this work
are explained in details.

\begin{table}[H]
    % \begin{table}[!htbp]
    \caption{Information of the datasets after merging}
    \bigskip\label{tab:datasets-after-first-step}
    \centering
    \begin{tabular}{c c c c c}
        \toprule
        \# & Name of the file & Size & Samples & Usage\\
        \midrule
        1 & size04.csv & 206 MegaBytes & 3675428 & train,test,validate\\
        2 & size08.csv & 513 MegaBytes & 2372324 & train,test,validate\\
        3 & size16.csv & 1.25 GigaBytes & 1439773 & train,test,validate\\
        4 & size32.csv & 2.02 GigaBytes & 567554 & train,test,validate\\
        \bottomrule
    \end{tabular}
\end{table}
Right after encoding the four video sequences for data 
collection, four CSV files
would be obtained for each of them.
Every CSV file contains two unique identifications,
one is the \emph{size of blocks} collected, the other is the 
\emph{name of the source video sequence}.
For example, one of CSV files shall include depth 
Luma data of blocks of size \(16\times16\) from
\emph{Newspaper} video sequence.
In the first step of data-preprocessing, all
the CSV files having the same identification of 
\emph{size of blocks} shall be merged into a single dataset
which will be further divided into training dataset,
testing dataset and validating dataset in the subsequent 
processing steps.
The specifications of the four CSV files after merging 
are shown in Table~\ref{tab:datasets-after-first-step}
on page~\pageref{tab:datasets-after-first-step}.
There is no data for blocks of size \(64\times64\)
since the largest block size for DMM modes is \(32\times32\).
As the size of block 
increases, the total volume of the collected data is growing
while the total number of samples of the collected data is
decreasing.
This phenomenon is reasonable in the sense that
the decreasing speed of the number of samples is slower 
than the increasing speed of the volume of every single
record.
More specifically, from block size \(4\times4\) to \(8\times8\),
the number of samples is decreased by 1.55 times; however,
there exists a fourfold increase of the volume size in 
each sample.

The statistics of the datasets after the first step of merging are
shown in Table~\ref{tab:unsorted-distribution-after-first-step}
on page~\pageref{tab:unsorted-distribution-after-first-step} and 
Table~\ref{tab:sorted-distribution-after-first-step}
on page~\pageref{tab:sorted-distribution-after-first-step}.
The statistics are unsorted or sorted in terms of the percentages
of each mode.
The statistics of each mode can be quickly found 
in Table~\ref{tab:unsorted-distribution-after-first-step}
on page~\pageref{tab:unsorted-distribution-after-first-step}
by looking at the first column.
From Table~\ref{tab:sorted-distribution-after-first-step}
on page~\pageref{tab:sorted-distribution-after-first-step},
it can be observed that the size differences of collected samples 
for each mode vary in a large range.
For example, the blocks of mode 0 are 96.9 times as many
the blocks of mode 16
while it is only 1.27 times as many the blocks of mode 35
for blocks of size \(32\times32\).
This is introducing the topic of imbalanced 
learning~\parencite{RN215} in which
the data sizes of each class are very different 
from each other.

According to the discussions presented in 
Subsection~\ref{subsec:discussion-about-data-visu},
mode 0, 1, 34, 35 and 36 will be removed from 
the datasets in the second step.
The specifications of the data after the second step
are shown in Table~\ref{tab:datasets-after-second-step}.
More than half of the collected data are removed.
\begin{table}[H]
    % \begin{table}[!htbp]
    \caption{Information of the datasets after removing mode 0, 1, 34, 35 and 36}
    \bigskip\label{tab:datasets-after-second-step}
    \centering
    \begin{tabular}{c c c c c c}
        \toprule
        \# & Name of the file & Size & Samples & Usage & Percent of removed data (\%) \\
        \midrule
        1 & msize04.csv & 75.7 MegaBytes & 3675428 & train,test,validate & 64 \\
        2 & msize08.csv & 130.8 MegaBytes & 2372324 & train,test,validate & 74 \\
        3 & msize16.csv & 377.3 MegaBytes & 1439773 & train,test,validate & 70 \\
        4 & msize32.csv & 708.7 MegaBytes & 567554 & train,test,validate & 65 \\
        % \# & Name of the file & Size & Samples & Usage\\
        % \midrule
        % 1 & msize04.csv & 75.7 MegaBytes & 3675428 & train,test,validate \\
        % 2 & msize08.csv & 130.8 MegaBytes & 2372324 & train,test,validate \\
        % 3 & msize16.csv & 377.3 MegaBytes & 1439773 & train,test,validate \\
        % 4 & msize32.csv & 708.7 MegaBytes & 567554 & train,test,validate \\
        \bottomrule
    \end{tabular}
\end{table}
In the third step, we start to remove the smooth blocks.
The reasons have been discussed in 
Subsection~\ref{subsec:discussion-about-data-visu}.
Obviously the first thing in this step is to 
define the level of the smooth.
% To tackle this issue, we will truncate a large 
% amount of data from the datasets in the subsequent steps.

\begin{table}[H]
    % \begin{table}[!htbp]
    \caption{Unsorted statistics of datasets obtained after merging}
    \bigskip\label{tab:unsorted-distribution-after-first-step}
    \centering
    \resizebox{\textwidth}{!}
    {\begin{tabular}{c c c c c c c c c}
        \toprule
        Block Size & \multicolumn{2}{c}{\(4\times4\)} & \multicolumn{2}{c}{\(8\times8\)} & \multicolumn{2}{c}{\(16\times16\)} & \multicolumn{2}{c}{\(32\times32\)} \\
        Mode Idx & Samples & Percent (\%) & Samples & Percent (\%) & Samples & Percent (\%) & Samples & Percent (\%)\\
        \midrule
        0 & 717,274 & 19.52 & 642,520 & 27.08 & 459,291 & 31.90 & 158,824 & 27.98 \\ 
        1 & 482,776 & 13.14 & 249,061 & 10.50 & 164,050 & 11.39 & 57,528 & 10.14 \\ 
        2 & 97,629 &  2.66 & 25,101 &  1.06 & 12,868 &  0.89 & 4,561 &  0.80 \\ 
        3 & 17,991 &  0.49 & 12,489 &  0.53 & 7,946 &  0.55 & 2,863 &  0.50 \\ 
        4 & 14,375 &  0.39 & 11,688 &  0.49 & 8,642 &  0.60 & 2,175 &  0.38 \\ 
        5 & 15,849 &  0.43 & 13,428 &  0.57 & 8,829 &  0.61 & 2,164 &  0.38 \\ 
        6 & 17,144 &  0.47 & 15,318 &  0.65 & 9,768 &  0.68 & 2,958 &  0.52 \\ 
        7 & 18,187 &  0.49 & 17,238 &  0.73 & 15,988 &  1.11 & 6,625 &  1.17 \\ 
        8 & 19,146 &  0.52 & 15,785 &  0.67 & 20,357 &  1.41 & 11,642 &  2.05 \\ 
        9 & 23,462 &  0.64 & 12,362 &  0.52 & 18,207 &  1.26 & 18,195 &  3.21 \\ 
       10 & 42,752 &  1.16 & 9,740 &  0.41 & 7,978 &  0.55 & 18,972 &  3.34 \\ 
       11 & 23,727 &  0.65 & 12,836 &  0.54 & 17,696 &  1.23 & 21,142 &  3.73 \\ 
       12 & 21,992 &  0.60 & 17,837 &  0.75 & 23,143 &  1.61 & 13,262 &  2.34 \\ 
       13 & 24,613 &  0.67 & 20,254 &  0.85 & 19,260 &  1.34 & 6,740 &  1.19 \\ 
       14 & 22,620 &  0.62 & 17,784 &  0.75 & 13,851 &  0.96 & 2,995 &  0.53 \\ 
       15 & 21,169 &  0.58 & 18,268 &  0.77 & 12,834 &  0.89 & 2,073 &  0.37 \\ 
       16 & 20,289 &  0.55 & 15,418 &  0.65 & 10,214 &  0.71 & 1,639 &  0.29 \\ 
       17 & 21,869 &  0.60 & 17,501 &  0.74 & 10,010 &  0.70 & 1,977 &  0.35 \\ 
       18 & 52,552 &  1.43 & 19,889 &  0.84 & 9,862 &  0.68 & 1,998 &  0.35 \\ 
       19 & 23,871 &  0.65 & 16,171 &  0.68 & 9,797 &  0.68 & 2,170 &  0.38 \\ 
       20 & 22,992 &  0.63 & 15,656 &  0.66 & 10,227 &  0.71 & 1,925 &  0.34 \\ 
       21 & 25,416 &  0.69 & 16,706 &  0.70 & 11,978 &  0.83 & 2,504 &  0.44 \\ 
       22 & 27,593 &  0.75 & 16,446 &  0.69 & 12,251 &  0.85 & 2,925 &  0.52 \\ 
       23 & 33,250 &  0.90 & 16,783 &  0.71 & 12,744 &  0.89 & 3,707 &  0.65 \\ 
       24 & 40,677 &  1.11 & 17,262 &  0.73 & 12,257 &  0.85 & 4,457 &  0.79 \\ 
       25 & 36,018 &  0.98 & 13,841 &  0.58 & 7,812 &  0.54 & 5,123 &  0.90 \\ 
       26 & 404,933 & 11.02 & 70,417 &  2.97 & 30,896 &  2.15 & 17,897 &  3.15 \\ 
       27 & 48,843 &  1.33 & 17,062 &  0.72 & 12,205 &  0.85 & 7,414 &  1.31 \\ 
       28 & 34,217 &  0.93 & 24,051 &  1.01 & 16,725 &  1.16 & 6,169 &  1.09 \\ 
       29 & 37,756 &  1.03 & 23,486 &  0.99 & 15,647 &  1.09 & 5,436 &  0.96 \\ 
       30 & 30,910 &  0.84 & 20,972 &  0.88 & 12,782 &  0.89 & 3,945 &  0.69 \\ 
       31 & 35,102 &  0.95 & 20,499 &  0.86 & 13,331 &  0.93 & 3,475 &  0.61 \\ 
       32 & 25,756 &  0.70 & 18,811 &  0.79 & 12,254 &  0.85 & 3,289 &  0.58 \\ 
       33 & 33,270 &  0.91 & 19,088 &  0.80 & 11,943 &  0.83 & 3,526 &  0.62 \\ 
       34 & 73,107 &  1.99 & 31,114 &  1.31 & 15,848 &  1.10 & 5,382 &  0.95 \\ 
       35 & 789,662 & 21.48 & 710,089 & 29.93 & 299,368 & 20.79 & 126,427 & 22.28 \\ 
       36 & 276,639 &  7.53 & 139,353 &  5.87 & 70,914 &  4.93 & 23,450 &  4.13 \\ 
        \bottomrule
    \end{tabular}
    }
\end{table}

\begin{table}[H]
    % \begin{table}[!htbp]
    \caption{Sorted statistics of datasets obtained after merging}
    \bigskip\label{tab:sorted-distribution-after-first-step}
    \centering
    \resizebox{\textwidth}{!}
    {\begin{tabular}{c c c c c c c c c c c c c}
        \toprule
        Block Size & \multicolumn{3}{c}{\(4\times4\)} & \multicolumn{3}{c}{\(8\times8\)} & \multicolumn{3}{c}{\(16\times16\)} & \multicolumn{3}{c}{\(32\times32\)} \\
        Row idx  & Mode idx & Samples & Percent (\%) & Mode idx & Samples & Percent (\%) & Mode idx & Samples & Percent (\%) & Mode idx & Samples & Percent (\%)\\
        \midrule
        0 & 4  & 14,375 &  0.39 & 10  & 9,740 &  0.41 & 25  & 7,812 &  0.54 & 16  & 1,639 &  0.29 \\ 
        1 & 5  & 15,849 &  0.43 & 4  & 11,688 &  0.49 & 3  & 7,946 &  0.55 & 20  & 1,925 &  0.34 \\ 
        2 & 6  & 17,144 &  0.47 & 9  & 12,362 &  0.52 & 10  & 7,978 &  0.55 & 17  & 1,977 &  0.35 \\ 
        3 & 3  & 17,991 &  0.49 & 3  & 12,489 &  0.53 & 4  & 8,642 &  0.60 & 18  & 1,998 &  0.35 \\ 
        4 & 7  & 18,187 &  0.49 & 11  & 12,836 &  0.54 & 5  & 8,829 &  0.61 & 15  & 2,073 &  0.37 \\ 
        5 & 8  & 19,146 &  0.52 & 5  & 13,428 &  0.57 & 6  & 9,768 &  0.68 & 5  & 2,164 &  0.38 \\ 
        6 & 16  & 20,289 &  0.55 & 25  & 13,841 &  0.58 & 19  & 9,797 &  0.68 & 19  & 2,170 &  0.38 \\ 
        7 & 15  & 21,169 &  0.58 & 6  & 15,318 &  0.65 & 18  & 9,862 &  0.68 & 4  & 2,175 &  0.38 \\ 
        8 & 17  & 21,869 &  0.60 & 16  & 15,418 &  0.65 & 17  & 10,010 &  0.70 & 21  & 2,504 &  0.44 \\ 
        9 & 12  & 21,992 &  0.60 & 20  & 15,656 &  0.66 & 16  & 10,214 &  0.71 & 3  & 2,863 &  0.50 \\ 
       10 & 14  & 22,620 &  0.62 & 8  & 15,785 &  0.67 & 20  & 10,227 &  0.71 & 22  & 2,925 &  0.52 \\ 
       11 & 20  & 22,992 &  0.63 & 19  & 16,171 &  0.68 & 33  & 11,943 &  0.83 & 6  & 2,958 &  0.52 \\ 
       12 & 9  & 23,462 &  0.64 & 22  & 16,446 &  0.69 & 21  & 11,978 &  0.83 & 14  & 2,995 &  0.53 \\ 
       13 & 11  & 23,727 &  0.65 & 21  & 16,706 &  0.70 & 27  & 12,205 &  0.85 & 32  & 3,289 &  0.58 \\ 
       14 & 19  & 23,871 &  0.65 & 23  & 16,783 &  0.71 & 22  & 12,251 &  0.85 & 31  & 3,475 &  0.61 \\ 
       15 & 13  & 24,613 &  0.67 & 27  & 17,062 &  0.72 & 32  & 12,254 &  0.85 & 33  & 3,526 &  0.62 \\ 
       16 & 21  & 25,416 &  0.69 & 7  & 17,238 &  0.73 & 24  & 12,257 &  0.85 & 23  & 3,707 &  0.65 \\ 
       17 & 32  & 25,756 &  0.70 & 24  & 17,262 &  0.73 & 23  & 12,744 &  0.89 & 30  & 3,945 &  0.69 \\ 
       18 & 22  & 27,593 &  0.75 & 17  & 17,501 &  0.74 & 30  & 12,782 &  0.89 & 24  & 4,457 &  0.79 \\ 
       19 & 30  & 30,910 &  0.84 & 14  & 17,784 &  0.75 & 15  & 12,834 &  0.89 & 2  & 4,561 &  0.80 \\ 
       20 & 23  & 33,250 &  0.90 & 12  & 17,837 &  0.75 & 2  & 12,868 &  0.89 & 25  & 5,123 &  0.90 \\ 
       21 & 33  & 33,270 &  0.91 & 15  & 18,268 &  0.77 & 31  & 13,331 &  0.93 & 34  & 5,382 &  0.95 \\ 
       22 & 28  & 34,217 &  0.93 & 32  & 18,811 &  0.79 & 14  & 13,851 &  0.96 & 29  & 5,436 &  0.96 \\ 
       23 & 31  & 35,102 &  0.95 & 33  & 19,088 &  0.80 & 29  & 15,647 &  1.09 & 28  & 6,169 &  1.09 \\ 
       24 & 25  & 36,018 &  0.98 & 18  & 19,889 &  0.84 & 34  & 15,848 &  1.10 & 7  & 6,625 &  1.17 \\ 
       25 & 29  & 37,756 &  1.03 & 13  & 20,254 &  0.85 & 7  & 15,988 &  1.11 & 13  & 6,740 &  1.19 \\ 
       26 & 24  & 40,677 &  1.11 & 31  & 20,499 &  0.86 & 28  & 16,725 &  1.16 & 27  & 7,414 &  1.31 \\ 
       27 & 10  & 42,752 &  1.16 & 30  & 20,972 &  0.88 & 11  & 17,696 &  1.23 & 8  & 11,642 &  2.05 \\ 
       28 & 27  & 48,843 &  1.33 & 29  & 23,486 &  0.99 & 9  & 18,207 &  1.26 & 12  & 13,262 &  2.34 \\ 
       29 & 18  & 52,552 &  1.43 & 28  & 24,051 &  1.01 & 13  & 19,260 &  1.34 & 26  & 17,897 &  3.15 \\ 
       30 & 34  & 73,107 &  1.99 & 2  & 25,101 &  1.06 & 8  & 20,357 &  1.41 & 9  & 18,195 &  3.21 \\ 
       31 & 2  & 97,629 &  2.66 & 34  & 31,114 &  1.31 & 12  & 23,143 &  1.61 & 10  & 18,972 &  3.34 \\ 
       32 & 36  & 276,639 &  7.53 & 26  & 70,417 &  2.97 & 26  & 30,896 &  2.15 & 11  & 21,142 &  3.73 \\ 
       33 & 26  & 404,933 & 11.02 & 36  & 139,353 &  5.87 & 36  & 70,914 &  4.93 & 36  & 23,450 &  4.13 \\ 
       34 & 1  & 482,776 & 13.14 & 1  & 249,061 & 10.50 & 1  & 164,050 & 11.39 & 1  & 57,528 & 10.14 \\ 
       35 & 0  & 717,274 & 19.52 & 0  & 642,520 & 27.08 & 35  & 299,368 & 20.79 & 35  & 126,427 & 22.28 \\ 
       36 & 35  & 789,662 & 21.48 & 35  & 710,089 & 29.93 & 0  & 459,291 & 31.90 & 0  & 158,824 & 27.98 \\ 
      
        \bottomrule
    \end{tabular}
    }
\end{table}
%Welcome to this \LaTeX{} Thesis Template, a beautiful and easy to use template for writing a thesis using the \LaTeX{} typesetting system.
%
%If you are writing a thesis (or will be in the future) and its subject is technical or mathematical (though it doesn't have to be), then creating it in \LaTeX{} is highly recommended as a way to make sure you can just get down to the essential writing without having to worry over formatting or wasting time arguing with your word processor.
%
%\LaTeX{} is easily able to~\parencite{RN93} professionally typeset documents that run to hundreds or thousands of pages long. With simple mark-up commands, it automatically sets out the table of contents, margins, page headers and footers and keeps the formatting consistent and beautiful. One of its main strengths is the way it can easily typeset mathematics, even \emph{heavy} mathematics. Even if those equations are the most horribly twisted and most difficult mathematical problems that can only be solved on a super-computer, you can at least count on \LaTeX{} to make them look stunning.
%
%%----------------------------------------------------------------------------------------
%
%\section{Welcome and Thanku}\label{sec:welome}
%Welcome to this \LaTeX{} Thesis Template, a beautiful and easy to use template for writing a thesis using the \LaTeX{} typesetting system.
%
%If you are writing a thesis (or will be in the future) and its subject is technical or mathematical (though it doesn't have to be), then creating it in \LaTeX{} is highly recommended as a way to make sure you can just get down to the essential writing without having to worry over formatting or wasting time arguing with your word processor.
%
%\LaTeX{} is easily able to professionally typeset documents that run to hundreds or thousands of pages long. With simple mark-up commands, it automatically sets out the table of contents, margins, page headers and footers and keeps the formatting consistent and beautiful. One of its main strengths is the way it can easily typeset mathematics, even \emph{heavy} mathematics. Even if those equations are the most horribly twisted and most difficult mathematical problems that can only be solved on a super-computer, you can at least count on \LaTeX{} to make them look stunning.
%
%%----------------------------------------------------------------------------------------
%
%\section{Welcome and ThYou}\label{sec:weome}
%Welcome to this \LaTeX{} Thesis Template~\parencite{Reference1}, a beautiful and easy to use template for writing a thesis using the \LaTeX{} typesetting system.
%
%If you are writing a thesis (or will be in the future) and its subject is technical or mathematical (though it doesn't have to be), then creating it in \LaTeX{} is highly recommended as a way to make sure you can just get down to the essential writing without having to worry over formatting or wasting time arguing with your word processor.
%
%\LaTeX{} is easily able to professionally typeset documents that run to hundreds or thousands of pages long. With simple mark-up commands, it automatically sets out the table of contents, margins, page headers and footers and keeps the formatting consistent and beautiful. One of its main strengths is the way it can easily typeset mathematics, even \emph{heavy} mathematics. Even if those equations are the most horribly twisted and most difficult mathematical problems that can only be solved on a super-computer, you can at least count on \LaTeX{} to make them look stunning.
%
%%----------------------------------------------------------------------------------------
%
%\section{Welcome and Thau}\label{sec:welcoe}
%Welcome to this \LaTeX{} Thesis Template, a beautiful and easy to use template for writing a thesis using the \LaTeX{} typesetting system.
%
%\begin{table}
%
%    \label{tab:treatments}
%    \centering
%%    \begin{tabular}{l l l}
%%        \toprule
%%        \tabhead{Groups} & \tabhead{Treatment X} & \tabhead{Treatment Y} \\
%%        \midrule
%%        1 & 0.2 & 0.8\\
%%        2 & 0.17 & 0.7\\
%%        3 & 0.24 & 0.75\\
%%        4 & 0.68 & 0.3\\
%%        \bottomrule\\
%%    \end{tabular}
%    \begin{tabular}{c r @{.} l}
%        Pi expression       &
%        \multicolumn{2}{c}{c}{Value} \\
%        \hline
%        $\pi$               & 3&1416  \\
%        $\pi^{\pi}$         & 36&46   \\
%        $(\pi^{\pi})^{\pi}$ & 80662&7 \\
%    \end{tabular}
%    \caption{The effects of treatments X and Y on the four groups studied.}
%\end{table}
%writing a thesis (or will be in the future) and its subject is technical or mathematical (though it doesn't have to be), then creating it in \LaTeX{} is highly recommended as a way to make sure you can just get down to the essential writing without having to worry over formatting or wasting time arguing with your word processor.
%
%\LaTeX{} is easily able to professionally typeset documents that run to hundreds or thousands of pages long. With simple mark-up commands, it automatically sets out the table of contents, margins, page headers and footers and keeps the formatting consistent and beautiful. One of its main strengths is the way it can easily typeset mathematics, even \emph{heavy} mathematics. Even if those equations are the most horribly twisted and most difficult mathematical problems that can only be solved on a super-computer, you can at least count on \LaTeX{} to make them look stunning.
%
%\section{Welcome and Tnk You}\label{sec:wlcome}
%Welcome to this \LaTeX{} Thesis Template, a beautiful and easy to use template for writing a thesis using the \LaTeX{} typesetting system.
%
%If you are writing a thesis.
%
%%\begin{verbatim}
%\begin{figure}
%    \centering
%    \includegraphics{Figures/Electron}
%    %    \decoRule
%    \caption[An Electron]{An electron (artist's impression).}
%    \label{fig:Electron}
%\end{figure}
%%\end{verbatim}
%(or will be in the future) and its subject is technical or mathematical (though it doesn't have to be), then creating it in \LaTeX{} is highly recommended as a way to make sure you can just get down to the essential writing without having to worry over formatting or wasting time arguing with your word processor.
%
%\LaTeX{} is easily able to professionally typeset documents that run to hundreds or thousands of pages long. With simple mark-up commands, it automatically sets out the table of contents, margins, page headers and footers and keeps the formatting consistent and beautiful. One of its main strengths is the way it can easily typeset mathematics, even \emph{heavy} mathematics. Even if those equations are the most horribly twisted and most difficult mathematical problems that can only be solved on a super-computer, you can at least count on \LaTeX{} to make them look stunning.
%
%%----------------------------------------------------------------------------------------
    % !TEX root = ../main.tex
\chapter{Train the Deep Model for Prediction}\label{ch:chapter4} % For referencing the chapter elsewhere, use \ref{Chapter1}
%
%%----------------------------------------------------------------------------------------
The convolutional neural networks (CNNs) consist of neurons that have 
weights and bias, which can be trained using large datasets for solving 
computer vision tasks.
In this chapter, our architecture of
deep convolutional neural network for intra mode prediction 
is illustrated.
Then the hyper-parameters of our deep model is introduced
with explanations.
At the end of the chapter, the stopping criteria and training results 
are presented.
\begin{figure}
    \centering
    \includegraphics[width=\textwidth,height=\textheight,keepaspectratio]{Figures/cnn_illustration.pdf}
    \caption[Two types of the residual units]{
        Two types of the residual units.}\label{fig:cnn-illustration}
\end{figure}

\section{The Architecture of CNN}\label{sec:cnn}
There exist many kinds of convolutional neural networks while the 
major difference that distinguish them from each other
is the uniqueness of each architecure.
Figure~\ref{fig:cnn-illustration} on 
page~\pageref{fig:cnn-illustration}
illustrates the basic architecture
of the convolutional neural networks.
The light blue cubics represent the Luma pixel values from a single
coding unit (CU) of size \(8\times8\) while the yellow cubics
shows a kernel of size \(3\times3\) that slides over the
CU in both directions.
At each position, the kernel does a weighted sum of all its inputs,
then adds biases. The output from the
region it covers will be fed into a neuron right 
below the covered region.
Although both are artificial neural networks, 
the big difference between multilayer perceptron and 
CNN is that weights of the neurons are shared in the latter case.
For example, all the neurons in Figure~\ref{fig:cnn-illustration} on
page~\pageref{fig:cnn-illustration} are reusing the same patch of
parameters, i.e., weights and biases.
For a single kernal of size \(3\times3\), it has 9 weights.
Apparently this amount of learnable parameters is just not enough.
More degrees of freedom are needed to enable 
the learning capability of a neural network.
For this reason, multiple kernels will be used instead of one single kernal.
If many kernels are stacked in a convolutional neural network such that
the architecture looks very deep, the network is called deep neural network
instead of simply neural network which typically refers to shallow ones.
In the deep convolutional neural networks, it can be imagined that
each convolutional layer comprising multiple kernels which subsequently produces
a three dimensional volume of the outputs.
After the convolutional layer, there normally exist activating layer where
the specified activation function will be applied to each single output
inside the output volume from the convolutional layer.
Followed by the activating layer, there is a pooling layer.
In the previous years, the maximum pooling and average pooling 
are popular methods for reducing the dimension of the outputs
from convolutional layers.
Nowadays, instead of applying a conventional pooling, 
a convolutional layer of larger stride is utilized.
Such a combination of convolution-activation-pooling will
be replicated multiple times for a single convolutional neural network.
In the tail of the network, a fully-connected layer can be used
to compute the probability of each target class for the 
classification problems.

Our network is exactly built from the above description except the 
fact that we have adopted the identity mapping 
\(h(\mathbf{x}_l)=\mathbf{x}_l\) in the Residual 
Neural Network from~\parencite{RN67}.
It has been shown in~\parencite{RN68} that
the identity mapping in the residual units
can enable the direct propagation of information
from one layer to any other layers.
Such a nice property brings vital benefits
for deep neural networks such that the accuracy
saturation problem can be alleviated.
As a result, ultra-deep models are able to
learn desired representations for solving
problems.
Besides, with the identity mapping, the network
is able to converge faster hence the training time
required for a deep model can be reduced.
% The architecture of the deep 
Figure~\ref{fig:basic-resnet-structure} on 
page~\pageref{fig:basic-resnet-structure}
shows two types
of the residual units which implement the identity mapping
via shortcut connections from the beginning of the block
to the end of the same block.
\begin{figure}
    \centering
    \includegraphics[width=\textwidth,height=\textheight,keepaspectratio]{Figures/basic-resnet-structure.pdf}
    \caption[Two types of the residual units]{
        Two types of the residual units are shown here.
        BN stands for Batch Normalization while
        Relu stands for Rectified Linear Unit.
        }\label{fig:basic-resnet-structure}
\end{figure}

The architecture of the deep convolutional neural 
network used in this thesis is
shown in Figure~\ref{fig:our-architecture} on 
page~\pageref{fig:our-architecture}.

\begin{figure}
    \centering
    \includegraphics[width=\textwidth,height=\textheight,keepaspectratio]{Figures/our-neural-net-structure.pdf}
    \caption[Neural network architecture for intra mode prediction]{
        Neural network architecture for intra mode prediction.
        }\label{fig:our-architecture}
\end{figure}

\section{The Hyper-parameters of CNN}\label{sec:config}
\section{Stopping criteria and Training Results}\label{sec:training}
%Welcome to this \LaTeX{} Thesis Template, a beautiful and easy to use template for writing a thesis using the \LaTeX{} typesetting system.
%
%If you are writing a thesis (or will be in the future) and its subject is technical or mathematical (though it doesn't have to be), then creating it in \LaTeX{} is highly recommended as a way to make sure you can just get down to the essential writing without having to worry over formatting or wasting time arguing with your word processor.
%
%\LaTeX{} is easily able to~\parencite{RN93} professionally typeset documents that run to hundreds or thousands of pages long. With simple mark-up commands, it automatically sets out the table of contents, margins, page headers and footers and keeps the formatting consistent and beautiful. One of its main strengths is the way it can easily typeset mathematics, even \emph{heavy} mathematics. Even if those equations are the most horribly twisted and most difficult mathematical problems that can only be solved on a super-computer, you can at least count on \LaTeX{} to make them look stunning.
%
%%----------------------------------------------------------------------------------------
%
%\section{Welcome and Thanku}\label{sec:welome}
%Welcome to this \LaTeX{} Thesis Template, a beautiful and easy to use template for writing a thesis using the \LaTeX{} typesetting system.
%
%If you are writing a thesis (or will be in the future) and its subject is technical or mathematical (though it doesn't have to be), then creating it in \LaTeX{} is highly recommended as a way to make sure you can just get down to the essential writing without having to worry over formatting or wasting time arguing with your word processor.
%
%\LaTeX{} is easily able to professionally typeset documents that run to hundreds or thousands of pages long. With simple mark-up commands, it automatically sets out the table of contents, margins, page headers and footers and keeps the formatting consistent and beautiful. One of its main strengths is the way it can easily typeset mathematics, even \emph{heavy} mathematics. Even if those equations are the most horribly twisted and most difficult mathematical problems that can only be solved on a super-computer, you can at least count on \LaTeX{} to make them look stunning.
%
%%----------------------------------------------------------------------------------------
%
%\section{Welcome and ThYou}\label{sec:weome}
%Welcome to this \LaTeX{} Thesis Template~\parencite{Reference1}, a beautiful and easy to use template for writing a thesis using the \LaTeX{} typesetting system.
%
%If you are writing a thesis (or will be in the future) and its subject is technical or mathematical (though it doesn't have to be), then creating it in \LaTeX{} is highly recommended as a way to make sure you can just get down to the essential writing without having to worry over formatting or wasting time arguing with your word processor.
%
%\LaTeX{} is easily able to professionally typeset documents that run to hundreds or thousands of pages long. With simple mark-up commands, it automatically sets out the table of contents, margins, page headers and footers and keeps the formatting consistent and beautiful. One of its main strengths is the way it can easily typeset mathematics, even \emph{heavy} mathematics. Even if those equations are the most horribly twisted and most difficult mathematical problems that can only be solved on a super-computer, you can at least count on \LaTeX{} to make them look stunning.
%
%%----------------------------------------------------------------------------------------
%
%\section{Welcome and Thau}\label{sec:welcoe}
%Welcome to this \LaTeX{} Thesis Template, a beautiful and easy to use template for writing a thesis using the \LaTeX{} typesetting system.
%
%If you are
%\begin{table}
%
%    \label{tab:treatments}
%    \centering
%%    \begin{tabular}{l l l}
%%        \toprule
%%        \tabhead{Groups} & \tabhead{Treatment X} & \tabhead{Treatment Y} \\
%%        \midrule
%%        1 & 0.2 & 0.8\\
%%        2 & 0.17 & 0.7\\
%%        3 & 0.24 & 0.75\\
%%        4 & 0.68 & 0.3\\
%%        \bottomrule\\
%%    \end{tabular}
%    \begin{tabular}{c r @{.} l}
%        Pi expression       &
%        \multicolumn{2}{c}{Value} \\
%        \hline
%        $\pi$               & 3&1416  \\
%        $\pi^{\pi}$         & 36&46   \\
%        $(\pi^{\pi})^{\pi}$ & 80662&7 \\
%    \end{tabular}
%    \caption{The effects of treatments X and Y on the four groups studied.}
%\end{table}
%writing a thesis (or will be in the future) and its subject is technical or mathematical (though it doesn't have to be), then creating it in \LaTeX{} is highly recommended as a way to make sure you can just get down to the essential writing without having to worry over formatting or wasting time arguing with your word processor.
%
%\LaTeX{} is easily able to professionally typeset documents that run to hundreds or thousands of pages long. With simple mark-up commands, it automatically sets out the table of contents, margins, page headers and footers and keeps the formatting consistent and beautiful. One of its main strengths is the way it can easily typeset mathematics, even \emph{heavy} mathematics. Even if those equations are the most horribly twisted and most difficult mathematical problems that can only be solved on a super-computer, you can at least count on \LaTeX{} to make them look stunning.
%
%%----------------------------------------------------------------------------------------
%
%\section{Welcome and Tnk You}\label{sec:wlcome}
%Welcome to this \LaTeX{} Thesis Template, a beautiful and easy to use template for writing a thesis using the \LaTeX{} typesetting system.
%
%If you are writing a thesis.
%
%%\begin{verbatim}
%\begin{figure}
%    \centering
%    \includegraphics{Figures/Electron}
%    %    \decoRule
%    \caption[An Electron]{An electron (artist's impression).}
%    \label{fig:Electron}
%\end{figure}
%%\end{verbatim}
%(or will be in the future) and its subject is technical or mathematical (though it doesn't have to be), then creating it in \LaTeX{} is highly recommended as a way to make sure you can just get down to the essential writing without having to worry over formatting or wasting time arguing with your word processor.
%
%\LaTeX{} is easily able to professionally typeset documents that run to hundreds or thousands of pages long. With simple mark-up commands, it automatically sets out the table of contents, margins, page headers and footers and keeps the formatting consistent and beautiful. One of its main strengths is the way it can easily typeset mathematics, even \emph{heavy} mathematics. Even if those equations are the most horribly twisted and most difficult mathematical problems that can only be solved on a super-computer, you can at least count on \LaTeX{} to make them look stunning.
%
%%----------------------------------------------------------------------------------------
    % !TEX root = ../main.tex
\chapter{Employ the Learned Deep Model}\label{ch:chapter5} % For referencing the chapter elsewhere, use \ref{Chapter1}
%
%%----------------------------------------------------------------------------------------
%
After obtaining the learned models which exhibit desired behavior, 
their integration into the reference software is carried out.
In this chapter, the time cost of the model prediction
is analyzed and optimizing methods are described.
After that we explain the details of the algorithm for integration.
In the end of the chapter the results of experiments 
are presented with discussions.

%3D Video applications are attracting more interests
%%----------------------------------------------------------------------------------------
%
\section{Analysis and Optimization for Prediction Time}\label{sec:analysis-and-optimization}
% In C++ APIs from Tensorflow version 1.1, if we want to run the prediction,
% a unique session has to be initialized in which the predicting activities 
% take place.
% It is found that the session initialization is very time expensive.
It is intuitive to think in this way: 
every time when the encoder encounters a new block in the video frames,
the learned models are required to perform the prediction for that block.
However, after this idea has been implemented, it turns out the
time cost of the encoder with neural engine even gets much higher 
than the encoder without neural engine.
We found that the idea of one prediction for one block actually
introduces a problem that impedes the integration to work as expected:
a new session is initialized for a prediction, and the initialization
for a single session in Tensorflow C++ APIs actually is very time
expensive.
The other idea which is a better way would be a single 
session for a batch of blocks.

The time cost of session initialization has been investigated
and compared between the two ideas aforementioned.
The learned model for blocks of size \(8\times8\) is loaded 
to run predictions for blocks of size \(8\times8\) from one single
frame in one video sequence with the resolution to 
be \(1024\times768\).
Totally 12288 blocks to predict.
When compiling the executable binary of the encoder
powered by neural engine, it can be compiled 
against CPU or GPU when the compilation tool 
called Bazel~\parencite{RN200} is utilized.
Even more, with Bazel, the AVX and SSE4.2 can be employed to
accelerate the parallel computations in one instruction
at single moment in CPU\@.
They offer more efficient matrix computations
to CPU\@.
Six experiments have been conducted and the results
are shown in 
Table~\ref{tab:seesion-init-plain-cpu}.
% on page~\pageref{tab:seesion-init-plain-cpu}.

\begin{table}
    \caption{Time cost of session initialization for two ideas based on three sets of compiling configurations}
    \bigskip\label{tab:seesion-init-plain-cpu}
    \centering
    \begin{tabular}{c c c c c}
        \toprule
        \# & Idea & Plain CPU & AVX and SSE4.2 & AVX, SSE4.2 and GPU \\
        \midrule
        1 & initialize 1 session for 12288 blocks & 21.37s &15.56s&2.03s \\
        2 & initialize 12288 sessions for 12288 blocks & 47.81s &33.91s&55.26s\\
        \bottomrule
    \end{tabular}
\end{table}

Obviously the best way is to compile the 
executable binary against AVX, SSE4.2 and GPU, at the same
time use the idea of one session for multiple blocks.

\section{The Integration of the Learned Model}\label{sec:integration-of-learned-model}
We have integrated two learned models into 
\(HTM16.2\) which is the newest version of
the reference software of 3D-HEVC by the time this thesis
is written\@.
More specifically, ResNet engines have been integrated to 
\textbf{\textit{TAppEncoder}} in \(HTM16.2\) for depth map 
angular modes [2, 34] prediction and the DMM1 searching process.
Wedgelet slope is employed
to reduce the number of wedgelet candidates to be evaluated in DMM1 
searching process. 
Since top-15 is used, roughly half of the candidates are skipped
during the encoding time.

The details of the proposed fast depth coding algorithm is shown 
in Figure~\ref{fig:proposed-fast-depth-coding-algorithm}
on page~\pageref{fig:proposed-fast-depth-coding-algorithm}.
Initially, before the encoding of a depth map will start, 
its Luma pixel values are accessed by the learned models to
predict best modes for all the suitable blocks inside it.
We call it \textbf{Batch Prediction}.
Next, for each depth block of 
size \(8\times8\), \(16\times16\) and \(32\times32\), 
top 15 modes together with PLANAR, DC modes are added to 
a candidate list for Rough Mode Decision (RMD).
For blocks of other sizes, the original encoder workflow
is applied.
Then if mode 2 is inside the RMD list, mode 34
will be added to RMD list due to the reason that
it is of the same angle as mode 2.
Else if mode 2 is not inside the RMD list, 
mode 34 will not be added to RMD list.
Then Sum of Absolute Transform Difference (SATD)
is used to obtain the best 3 candidates for blocks
of size \(8\times8\) and \(16\times16\), 2 candidates
for blocks of size \(32\times32\).
After SATD, the best prediction of the block
is utilized to construct a range of slopes.
Denote the number of the top-1 mode from the model 
prediction as \(N_1\), a slope range will be constructed 
using the angular modes within \([N_1-7, N_1+7]\), where
the two boundaries are inclusive.
In the further step, when performing the DMM1 coarse
searching, the wedgelets which have the slopes out of the 
constructed slope range are excluded from 
View Synthesis Optimization (VSO) which is the
source of the computational complexity in the original
encoder.
The DMM1 refinement following the coarse search is unmodified.
In the end, the best mode for the
depth map to be coded in the encoding process will
be obtained.
This algorithm can reduce the wedgelets to be evaluated by 
roughly half.
Besides, as a bonus, the conventional intra 
angular mode decision for depth maps is also
accelerated using the same prediction from 
neural networks.
% First 

\begin{figure}
    \centering
    \includegraphics[height=0.92\textheight,keepaspectratio]{Figures/proposed-fast-depth-coding-algorithm}
    \caption[Flowchart for proposed fast depth coding 
    algorithm]{Flowchart for proposed fast depth coding 
    algorithm.}\label{fig:proposed-fast-depth-coding-algorithm}
\end{figure}

\section{Results of Experiments}\label{sec:simu-results}
The experiments are conducted on four video sequences shown in
Table~\ref{tab:data-for-experiments} 
% on page~\pageref{tab:data-for-experiments} 
to make sure every sample
that will be predicted has never been seen by the learned models.
Two metrics including \textbf{BD-BR} from~\parencite{RN234} and \textbf{BD-PSNR}
from~\parencite{RN235}
are used in the evaluations for the proposed fast 
depth coding algorithm.
\textbf{BD-BR} calculates the average difference between
Rate Distortion curves of two sets of synthesized views
where one set of views are synthesized from 
the reconstructed views from original encoder while 
the other set of views are synthesized from 
the reconstructed views from the encoder powered by
deep learning.
\textbf{BD-PSNR} is employed to assess the subjective 
quality of synthesized views.
The common test condition defined
in~\parencite{common-test-condition}
is used in our experiments.
All the frames from four video sequences in experiments 
are encoded as I-Frames.
\begin{table}[!htbp]
    \caption{Video sequences used for experiments}
    \bigskip\label{tab:data-for-experiments}
    \centering
    \begin{tabular}{c c c c c}
        \toprule
        \# & Name of the Sequence & Resolution & Usage & Number of Frames\\
        \midrule
        1 & Newspaper & \(1024\times768\) & Simulation & 300\\
        2 & GhostTownFly & \(1920\times1088\) & Simulation & 250\\
        3 & PoznanHall2 & \(1920\times1088\) & Simulation & 200\\
        4 & Shark & \(1920\times1088\) & Simulation & 300\\
        \bottomrule
    \end{tabular}
\end{table}

Table~\ref{tab:ts-dmm}
% on page~\pageref{tab:ts-dmm}
shows the time saving for DMM1 wedgelet searching process together 
with the coding performance.
Table~\ref{tab:ts-total}
% on page~\pageref{tab:ts-total}
presents the time saving for total encoding and the coding performance.

\begin{table}[!htbp]
    \caption{Time saving for wedgelet searching and coding performance of proposed method}
    \bigskip\label{tab:ts-dmm}
    \centering
    \begin{tabular}{c c c c c c c}
        \toprule
         & \multicolumn{4}{c}{Time saving for DMM1} & & \\\cline{2-5}
        Sequences & QP34 & QP39 & QP42 & QP45 & BD-BR & BD-PSNR \\
        \midrule
        Newspaper       & 63.76 & 64.94 & 71.98 & 74.14 & 0.98\% & -0.02dB \\
        PoznanHall2    & 71.08 & 71.08 & 66.36 & 71.27 & 1.64\% & -0.05dB \\
        GhostTownFly       & 62.00 & 56.20 & 51.60 & 58.87 & 0.65\% & -0.02dB \\
        Shark           & 63.55 & 58.66 & 63.26 & 63.34 & 1.04\% & -0.03dB \\
    \end{tabular}
\end{table}

\begin{table}[!htbp]
    % \begin{table}[!htbp]
    \caption{Time saving for total encoding and coding performance of proposed method}
    \bigskip\label{tab:ts-total}
    \centering
    % \resizebox{\textwidth}{!}
    % {
    \begin{tabular}{c c c c c c c}
        \toprule
         & \multicolumn{4}{c}{Time saving for total encoding} & & \\\cline{2-5}
        Sequences & QP34 & QP39 & QP42 & QP45 & BD-BR & BD-PSNR \\
        \midrule
        Newspaper       & 27.33 & 27.20 & 27.78 & 27.38 & 0.98\% & -0.02dB \\
        PoznanHall2     & 29.00 & 28.34 & 28.75 & 19.04 & 1.64\% & -0.05dB \\
        GhostTownFly    & 32.52 & 31.19 & 31.25 & 32.07 & 0.65\% & -0.02dB \\
        Shark           & 38.94 & 31.26 & 37.33 & 36.86 & 1.04\% & -0.03dB \\
    \end{tabular}
    % }
\end{table}

From the experimental statistics shown in the two tables above,
our fast depth coding algorithm achieves
an average time reduction of 64.6\% in wedgelet searching 
during 3D-HEVC encoding
while the BD performance only has a trivial decrease.
It is noticed that the values of \textbf{BD-PSNR} almost stayed
unchanged comparing
with the original implementation from the reference software
of 3D-HEVC\@.
The models trained by deep learning
were trying their best to keep the originality of the
pixel blocks that they have seen during the encoding process.
However, the learned models have no idea of how to
adjust their prediction results by taking the bitrate
into consideration.
For this reason, the perceptual visual qualities of the videos
are well reserved while the performance on bitrate
decreased a little as 
a penalty.
In each single encoding activity for a video sequence, 
the time saving for the whole encoding process is lower than 
the time saving for DMM1 but still the overall results
achieved by the proposed fast depth coding algorithm
can be considered as excellent when
comparing with the original implementation of 3D-HEVC\@.

    % !TEX root = ../main.tex
\chapter{Conclusion}\label{ch:chapter6} % For referencing the chapter elsewhere, use \ref{Chapter1}
%
%%----------------------------------------------------------------------------------------
%
In this dissertation we have presented a fast
intra mode decision algorithm to
address the time cost issue of depth map coding
in 3D-HEVC, leveraging 
the learned models, which are constructed from deep
convolutional neural networks, to predict the wedgelet angles
for the CUs in the depth maps.
The predictions from the learned models are capable of
reducing the number of DMM1 wedgelet candidates by half.
The size of the neural network has been carefully designed to balance
the trade-off between complexity and accuracy of model prediction.
Validation precision and confusion matrix are 
at the core of the stopping criteria.
Top-k metric is adopted to make use of the predictions
from the learned models.
The learned models are integrated into the reference software of
3D-HEVC for the simulations.
The compiled executable binaries are able to harness
the power of simultaneous computation of CPU, as well as
parallel computation of GPU to accelerate the 
predictions.
The simulation results show that the proposed algorithm
provides 64.6\% time reduction in average 
for DMM1 wedgelet searching
in depth maps while the
BD performance has a trivial decrease comparing 
with the most recent implementation of 3D-HEVC
standard.

    \include{Chapters/chapter7}
%    \appendix
%    \chapter{A Long Proof}
    \printbibliography[heading=bibintoc]
\end{document}